\documentclass{article}

\usepackage[T2A]{fontenc}
\usepackage[utf8]{inputenc}
\usepackage[english,russian]{babel}
\usepackage[left=2cm,right=2cm,top=2cm,bottom=2cm,bindingoffset=0cm]{geometry}

\usepackage{sectsty}
\allsectionsfont{\centering}

\parindent=0cm
\begin{document}
\title{Л.П.Дыко - Беседы о фотомастерстве}
\author{Tass}
\date{2016/07/25}
\maketitle

\tableofcontents
\newpage
\section{Путь в фотоискусство}
\textbf{Художественное мастерство фотографа}. Смысл и значение художественного мастерства состоят в том, чтобы найти в жизни типичные и яркие ее проявления, воплотить жизненные наблюдения в образах, использовать все возможности и средства фотографии, для полного, глубокого раскрытия содержания произведения. Зритель увидит и оценит не световые пятна и линейные сходы, как таковые, а правдивые картины жизни, типические характеры, ярко, выпукло, живописно воспроизведенные фотографом.

\section{Искусство фотографии и его изобразительные средства}
\textbf{Изобразительные средства фотогорафии} --- линейная композиция, световое решение, тональный рисунок. Так же, возможно творческое использование технических средств --- экспозиции, выдержки, глубины резкости.

\section{Изобразительное решение темы}
\textbf{Начало формирования кадра} --- сознательный отбор жизненного материала, введение в кадр только тех деталей, которые органически связаны с выбранным сюжетом. Без ясного авторского замысла, без четкого сюжета не может быть найдено законченное изобразительное решение снимка.
\\
\textbf{Изобразительное решение темы} начинается с определения смыслового центра снимаемого сюжета. Это не просто нахождение внешнего рисунка кадра --- линейного ритма, градации тонов, характера светотени, etc., это --- осмысление материала, умение понять его и выразить в конкретном моменте развивающегося действия. Отобрав из всего материала то, что яснее всего раскрывает смысл, содержание картины, фотограф должен выделить эти элементы и изобразительно нарисовать их отчетливо, выпукло, объемно, то есть смысловой центр снимка сделать зрительным центром.

\section{Смысловой и изобразительный центр кадра}
\textbf{Отчетливому изображению главного объекта} будут способствовать:
\begin{itemize}
\item Укрупнение, изображение сюжетного центра в крупном масштабе: съемка ведется с близкого расстояния, а рамка кадра очерчивает относительно небольшое пространство.
\item Размещение главного объекта на переднем плане. В кадр могут попасть и достаточно широкие пространства, снимок может представлять собой общий план, и только главный объект находится на близком расстоянии от точки съемки.
\item Тональное различие объекта и фона. Контраст тонов помогает выявлению главного объекта изображения.
\item Световой акцент, при котором самые высокие яркости образуются на главном элементе композиции.
\item Наводка на резкость по главному объекту изображения и потеря резкости на фоне и второстепенных элементах композиции.
\item Направление основных линий в кадре, ведущих глаз зрителя к центру композиции.
\item Размещение главного объекта изображения в центре картинной плоскости или близко к нему.
\item Создание контрастного тонального рисунка на главном и мягкая градация тонов на второстепенных композиционных элементах и на фоне.
\end{itemize}

\section{О принципах заполнения картинной плоскости}
Творческий процесс получения композиционно слаженного снимка достаточно сложен и тонок, многие его звенья основаны на интуиции и таланте фотографа. Творческая мысль автора может дать совершенно различные изобразительные результаты с помощью одной и той же техники.

\section{Уравновешенные композиции}
\textbf{Принцип равновесия} --- пропорциональность отдельных частей картины.
\\
Самая строгая уравновешенность частей картины возникает в том случае, когда композиция снимка состоит из элементов, расположенных \textbf{симметрично}. 
\\
При \textbf{фронтальной композиции} зрителю видна только одна из плоскостей, ограничивающих объемные предметы, находящиеся в центре кадра. При этом, главные оси элементов картины --- фигур, предметов, сооружений, расположенных в поле зрения объектива, --- совпадают с осью всей фотографической картины. Такое расположение и проекция отдельных элементов изображения приводят к потере глубины, к ослабленной передаче объемов. Обычно, при фронтальной композиции возникает общая статичность изображения, появляются спокойствие и строгость в трактовке материала.
\\
Один из простейших способов построения уравновешенной композиции --- размещение главного объекта изображения в центре картины. Устойчивость центральной композиции ограничивает ее применение при съемке сюжетов, связанных с движением.Быстродвижущийся объект, зафиксированный в центре кадра, как бы теряет свой динамизм, движение тормозится, а то и останавливается.
\\
Равновесие в кадре может быть достигнуто не только при сопоставлении двух предметов, находящихся в разных частях картины, но и введением в общую картину композиции элементов освещения --- активного светотеневого рисунка, светового пятна, блика.
\\
Композиции, строящиеся с учетом \textbf{развивающегося движения}, по направлению движения на картинной плоскости могут иметь незначительное свободное пространство. Эта незаполненная часть кадра имеет свой смысл и значение --- мы понимаем, что в последующие моменты времени, которые не могут быть запечатлены в единичном снимке, движущийся элемент композиции будет перемещаться вдоль картинной плоскости и свободное пространство --- это как бы освобожденный для движения путь.
\\ 
Сюжетно важные элементы, в первую очередь привлекающие и задерживающие внимание зрителя, легко уравновешивают более громоздкие, но менее значимые компоненты картины.
\\
Законы заполнения картинной плоскости так же могут быть отнесены и к соотношениям верхней и нижней частей снимка. При этом, за счет более темного тонального рисунка, во многих снимках в нижней части кадра можно найти опору и основу для других компонентов картины. 
\\
Существуют композиции, где равновесие сознательно нарушается в целях достижения определенного художественного эффекта.

\section{Ритмический рисунок кадра}
Линии в фотографии существуют не в том виде, в каком мы встречаем их, например, в карандашном рисунке. Линия на фотографии не является первичным элементом, а возникает как производная от тона. Она представляет на снимке всего-лишь раздел двух тонов, иногда ясный и четкий, иногда мягкий, а порой и вовсе расплывчатый.
\\
Под \textbf{ритмом} в произведении искусства мы понимаем закономерное чередование композиционных элеметнов, их повторяемость через определенные промежутки, порядок их сочетания.
\\
Ритм линий, световых пятен и других элементов изобразительной формы отнюдь не главное в произведении фотоискусства. Конечно, они дают возможность получить иногда строгий и стройный, иногда очень нарядный рисунок изображения. Но цель ритмических построений --- выявление характерных особенностей объекта съемки, раскрытие смысла, содержания картины, живописно-художественная трактовка темы. В таком понимании ритм теряет чисто формальное значение и прочно связывается с развитием сюжета, с движением, происходящими в кадре, и пр. Вместе с другими творческими приемами, ритм придает художественной картине глубокую правдивость, поэтичность, образность.

\section{О других проблемах изобразительного решения снимка}
Для достижения \textbf{цельности композиционного рисунка}, необходимо найти прочные внутрикадровые связи всех элементов, образующих данную композицию. Они легко нарушаются в ряде случаев, о которых следует знать. Композиция распадается на две самостоятельные части, если она состоит из двух основных элементов и эти элементы отнесены к краям кадра, находятся в противоположных частях картинной плоскости и разделены свободным пространством. Композиционная цельность картины может быть так же нарушена, если четкие горизонтальные или вертикальные линии идут параллельно границам кадра. Опасность утраты цельности комопзиции особенно велика в тех случаях, когда линии совпадают с центральными осями прямоугольника кадра.
\\
\textbf{Соотношение объекта и фона} в кадре --- вопрос чрезвычайно важный и отнюдь не формальный. В одних случаях световая разработка фона, увязанная с освещением главного объекта изображения, помогает правдиво передать обстановку, среду. В других --- соотношение двух этих элементов изображения по степени резкости способствует передаче пространства. Во всех случаях, правильное решение проблемы дает возможность выпукло изобразить сюжетно важные части картины.
\\
\textbf{Лаконизм}, как средство художественной выразительности, предлагает простоту конструктивного построения снимка, использование минимума изобразительных элементов для полного и всестороннего раскрытия темы.
\\
\textbf{Границы} хорошо скомпонованного снимка занимают строго определенное положение и, как правило, обусловлены размещением материала в пределах поля кадра, его общей композицией. Очень часто при этом границы снимка как бы опираются на фигуры и предметы, находящиеся у краев кадра, они-то и удерживают линии рамки кадров в устойчивом положении.

\section{Свет в природе и фотографии}
\textbf{Источники света на натурных съемках}. Главный источник света - солнце, даже если съемка ведется в пасмурную погоду и свет достигает объекта лишь будучи рассеянным плотным слоем облаков. Со стороны источника света на предмете образуются высокие яркости - \textit{света}. С противоположной стороны возникают \textit{собственные тени}. Сам освещаемый предмет отбрасывает на окружающие его поверхности \textit{падающую тень}. Окружающие предмет поверхности отбрасывают встречный поток отраженного рассеянного света. В тени отраженный свет виден достаточно хорошо, он высветляет теневые участки. Эту подсветку называют \textit{рефлексом}. Поверхности, ограничивающие объем предмета, обращены к свету под различными углами, поэтому одни из них выглядят темнее, другие светлее. Так появляются не только тени, а еще и \textit{полутени} - мягкие переходы от ярких светов к глубоким теням. В местах, где луч света отразился от освещаемой поверхности под углом зеркального отражения, возникают \textit{блики}.
\begin{itemize}
\item \textbf{Рассеянный свет неба}. Дневное освещение может дать и очень мягкий рисунок светотени и очень контрастный, поскольку количество света меняется в зависимости от состояния неба. Чистое безоблачное небо дает наименьшее количество света. Облака повышают количество рассеянного света на освещаемом объекте --- контрасты светотени при этом смягчаются. Если небо сплошь затянуто легким полупрозрачным слоем облаков, объект съемки насыщается заполняющим светом предельно и рисунок теней тогда еле намечается.
\item \textbf{Фронтальный свет}: выразительный световой рисунок отсутствует, нет необходимой градации светов и теней, объемные формы и пространства на снимках передаются плохо, рисунок становится плоским, тени от предметов падают назад и скрываются за самими предметами. Пространство при фронтальном освещении очерчивается вяло. Фронтальным освещением можно воспользоваться, если фотограф задался целью передать собственные тона объекта, не изменяя их световым рисунком. Такое изображение получается очень тонким, нежным, как говорят иногда, пастельным.
\item \textbf{Боковой свет}. Чередование светов и теней, освещенных и затененных участков, дает представление о пространственной протяженности объекта съемки, и снимок теряет плоскостной рисунок, который часто портит изображение, сделанное при фронтальном освещении. Боковой свет очерчивает объемы и рельефы, более отчетливо передает фактуры поверхностей и предметов.
\item \textbf{Контровой светы}. Предметы, составляющие объект съемки, при контровом освещении обращены к аппарату своей теневой стороной, в кадре преобладают тени, и снимок приобретает низкую, темную тональность. Изображение становится еще более собранным в тоне, если сравнивать его с кадрами, сделанными при боковом освещении, т.к. световые блики занимают относительно небольшие площади. Пространство передается как за счет нарастания светлоты тонов глубине, так и за счет хорошей проработки воздушной среды.
\end{itemize}
Все съемочное время суток можно разделить на несколько периодов, каждый из которых имеет свои световые особенности и дает возможность получить своеобразный колорит снимка.
\begin{itemize}
\item \textbf{Сумеречное освещение}. Освещенности при сумеречном освещении невелики. Небо становится достаточно ярким фоном для затемненных наземных предметов. Так возникают силуэтные или полусилуэтные рисунки изображения, привлекающие своей собранностью в тоне, своеобразным строгим колоритом. Утреннее небо передается достаточно светлой тональностью. Вечерний закат богат красками и часто сопровождается причудливым рисунком облаков, расцвеченных лучами солнца, уже скрывшегося за горизонтом.
\item \textbf{Эффектное освещение} - угол стояния солнца не выше 15$^{\circ}$. Контрасты светотени в период эффектного освещения относительно невысоки. Количество рассеянного света велико и тени получают хорошую подсветку. Над землей часто поднимается легкая туманная дымка, тоже смягчающая контрасты и входящая сама по себе выразительным компонентом в фотокартину.
\item \textbf{Нормальное освещение} - угол стояния солнца от 15$^{\circ}$ до 60$^{\circ}$. На объекте имеется выразительная светотень, в достаточной мере смягченная рассеянным светом неба. Тени постепенно становятся более короткими, но светотеневой рисунок остается живописным и выразительным, хорошо очерчивает объемы, выявляет фактуры и пространственные положения предметов.
\item \textbf{Зенит}. Тени очень коротки, а их контрасты высоки. Меняется баланс освещения вертикальных и горизонтальных поверхностей объекта --- первые освещены значительно меньше, чем вторые. Небо приобретает повышенную яркость. При съемке в этот период времени, трудно ожидать живописных снимков.
\end{itemize}

\section{Тональное решение снимка}
Разнообразные яркие и пастельные тона при воспроизведении средствами черно-белой фотографии преобразуются в гамму черных, серых и белых тонов, различающихся между собой только по \textbf{светлоте}. Тона различной светлоты, так же как и цвета в картине, не могут распределяться на картинной плоскости произвольно, они требуют определенного согласования.
\\
Подобно тому как, анализируя художественное решение живописного полотна, говорят о золотистом колорите или о картине, выдержанной в голубых тонах, точно так же говорят и о светлой или темной тональности черно-белого снимка. Таким образом, называют тон, преобладающий, доминирующий в картине. \textbf{Доминирующий тон} является основным, и с ним согласуются все остальные тона и полутона. Они обозначаются на снимке более легкими "мазками" и не разрушают единства тонального решения общей светлой или приглушенной темной тональностью снимка.
\\
Добиваясь образного решения темы, фотограф использует то или иное тональное решение снимка, как действенное средство выполнения своих замыслов. При известном уровне мастерства, средства образования тонального рисунка становятся управляемыми и тонкая градация тонов возникает последовательно во всех звеньях процесса создания снимка.

\section{Пространственность кадра}
Поскольку снимок двухмерен, возможность прямой передачи на снимке глубины пространства отсутствует. Порой это приводит к образованию совершенно плоских изображений, где не чувствуется расстояний, отделяющих передний план от предметов удаленных.
\\
Восприятие человеком пространства связано с реальными проявлениями \textbf{линейной перспективы}. Основные ее закономерности можно сформулировать так:
\begin{itemize}
\item Фигуры и предметы кажутся тем меньшими, чем дальше они находятся.
\item Параллельные линии, уходящие в даль, обнаруживают стремление сойтись в одной точке.
\item Грани предметов, направленные по лучу зрения глаза наблюдателя, кажутся короче, чем в действительности.
\end{itemize}
На перспективу фотоизображения оказывают влияние расстояние, с которого ведется съемка; высота установки фотоаппарата; направление оптической оси объектива по отношению к снимаемому объекту. Фокусное расстояние объектива на перспективу \textbf{не влияет}.
\\
\textbf{Центральные точки} съемки не являются самыми распространенными. Перспектива таких кадров, как правило, ослаблена. Смещение точки вверх или вниз может добавить изображению перспективы.
\\
В случае съемки с \textbf{боковых точек}, основные композиционные линии устремляются к боковым точкам схода и раскрывают пространственную протяженность объекта. Эффект будет тем большим, чем больше смещена точка в сторону от ее центрального положения. Направление композиционных линий при этом постепенно приближается к диагонали прямоугольника кадра, благодаря ярко выраженному сходу линий рисунок преобретает подчеркнутую направленность и вследствие этого динамичность.
\\
\textbf{Нормальными} по высоте принято считать такие точки съемки, которые находятся на уровне глаз человека. Предметы и пространства, снятые с таких точек, рисуются такими же, как мы привыкли видеть их в жизни.
\\
При съемке с \textbf{нижней} точки изменяется привычное сопоставление переднего и дальнего плана по высоте: даже невысокие предметы, находящиеся на переднем плане, часто проецируются на фон неба или оказываются на одной высоте с большими сооружениями дальнего плана. Зрителю кажется, что предметы переднего плана стали высокими, масштабными.
\\
При \textbf{верхней} точке съемки предметы переднего плана проецируются на фон земли, кажется, что они потеряли свою высоту, принижены.
\\
Понятие \textbf{ракурс} связано со съемкой под углом, с верхних или нижних точек, которые должны быть и достаточно близкими.При съемке в ракурсе резко сокращаются по длине линии, уходящие от наблюдателя в глубину, масштабы изображения отдельных деталей объекта с их удалением от точки съемки. Ракурс в ряде случаев помогает вскрыть сущность происходящего, выявить и подчернуть характерные особенности объекта.
\\
Плоскостные решения снимка имеют право на жизнь. Часто они имеют своеобразный линейный ритм и в ряде случаев, связанных с особыми характеристиками объектов съемки, становятся основой интересных линейных построений.
\end{document}