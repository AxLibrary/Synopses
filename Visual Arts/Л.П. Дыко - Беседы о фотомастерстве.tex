\documentclass{article}
% Кодировка, поддержка русского языка
\usepackage[T2A]{fontenc}
\usepackage[utf8]{inputenc}
\usepackage[english,russian]{babel}
% Отступы от края страницы
\usepackage{geometry}
\geometry{left=3cm}
\geometry{right=3cm}
\geometry{top=2cm}
\geometry{bottom=2cm}
\geometry{bindingoffset=0cm}
% Вложенные списки
\usepackage[ampersand]{easylist}
\ListProperties(Hide=100, Hang=true, Progressive=2ex,Style*=$\bullet$ ,Style1*=$\Rightarrow$ )

\renewcommand{\title}[2]{
	\begin{center}
		\LARGE
		\baselineskip=0.5\baselineskip
		\textbf{#1}
		\\
		\rule[0.5\baselineskip]{0.7\textwidth}{0.15pt}
		\\
		\textbf{#2}
		\\\baselineskip=2\baselineskip(конспект)		
	\end{center}
}
\renewcommand{\section}[2]{
	\vspace{2em}
	\begin{flushright}
		\Large
		\baselineskip=0.5\baselineskip
		\textbf{#1}
		\\
		\rule[0.5\baselineskip]{\textwidth}{0.15pt}
		\\
		\textbf{#2}
	\end{flushright}
}
\renewcommand{\subsection}[1]{
	\begin{flushright}
		\large
		\textbf{#1}
	\end{flushright}
}
\newcommand{\note}[1]{\textit{#1}}
\newcommand{\important}[1]{\textbf{#1}}	
\newcommand{\define}[2]{
	\important{#1} --- #2
}
\begin{document}
\title{Л.П. Дыко}{Беседы о фотомастерстве}
\section{Беседа первая}{Путь в фотоискусство}
\begin{easylist}
& Художественное мастерство состоит в том, чтобы найти в жизни яркие ее проявления и воплотить жизненные наблюдения в образах.
& Зритель видит и оценивает не световые пятна и линейные сходы, а правдивые картины жизни, типические характеры, живописно воспроизведенные фотографом.
\end{easylist}
\section{Беседа вторая}{Искусство фотографии и его изобразительные средства}
\begin{easylist}
& Изобразительные средства фотографии: линейная композиция, световое решение, тональный рисунок.
& Возможно творческое использование технических средств: экспозиция, выдержка, глубина резкости
\end{easylist}
\section{Беседа третья}{Изобразительное решение темы}
\begin{easylist}
& Начало формирования кадра --- отбор материала, введение в кадр деталей, связанных с сюжетом. Без замысла, четкого сюжета, не может быть найдено решение снимка.
& Изобразительное решение темы начинается с определения смыслового центра сюжета, осмысления материала, умения понять его и выразить в конкретном моменте. 
& Отобрав из материала то, что раскрывает смысл, содержание, необходимо выделить эти элементы и изобразить отчетливо, выпукло, объемно, сделать смысловой центр снимка зрительным центром.
\end{easylist}
\section{Беседа четвертая}{Смысловой и изобразительный центр кадра}
\begin{easylist}
& Отчетливому изображению главного объекта будут способствовать:
&& Изображение сюжетного центра в крупном масштабе.
&& Размещение главного объекта на переднем плане.
&& Тональное различие объекта и фона.
&& Световой акцент, при котором самые высокие яркости образуются на главном элементе композиции.
&& Наводка на резкость по главному объекту изображения, потеря резкости на фоне.
&& Направление основных линий в кадре, ведущих глаз к центру композиции.
&& Размещение главного объекта изображения в центре или близко к нему.
&& Создание контрастного тонального рисунка на главном и мягкая градация тонов на фоне.
\end{easylist}
\section{Беседа пятая}{О принципах заполнения картинной плоскости}
\begin{easylist}
& Процесс получения композиционно слаженного снимка сложен и тонок, звенья основаны на интуиции и таланте фотографа.
& Творческая мысль может дать различные результаты с помощью одной и той же техники.
\end{easylist}
\section{Беседа шестая}{Уравновешенные композиции}
\begin{easylist}
& Самая строгая уравновешенность частей картины возникает в случае композиции, состоящей из элементов, расположенных симметрично.
& При фронтальной композиции видна только одна из плоскостей, ограничивающих объемные предметы в центре кадра. Главные оси элементов картины  совпадают с осью всей картины. Это приводит к потере глубины, ослабленной передаче объемов. Возникает общая статичность изображения, появляются спокойствие и строгость в трактовке материала.
& Один из способов построения уравновешенной композиции --- размещение главного объекта в центре картины. Устойчивость центральной композиции ограничивает ее применение при съемке сюжетов, связанных с движением, объект, зафиксированный в центре кадра, теряет динамизм, движение тормозится или останавливается.
& Равновесие может быть достигнуто введением в общую картину композиции элементов освещения --- светотеневого рисунка, светового пятна, блика.
& Композиции, строящиеся с учетом развивающегося движения, по направлению движения на картинной плоскости могут иметь незначительное свободное пространство, которое представляет освобожденный для движения путь.
& Важные элементы, привлекающие внимание зрителя, легко уравновешивают менее значимые компоненты картины.
& Законы заполнения картинной плоскости могут быть отнесены к соотношениям верхней и нижней частей снимка. За счет более темного тонального рисунка, во многих снимках в нижней части кадра можно найти опору для других компонентов картины.
& Существуют композиции, где равновесие сознательно нарушается для достижения художественного эффекта.
\end{easylist}
\section{Беседа седьмая}{Ритмический рисунок кадра}
\begin{easylist}
& Линия на фотографии не является первичным элементом, возникает как производная от тона. На снимке представляет раздел двух тонов, иногда ясный и четкий, иногда мягкий, а порой и вовсе расплывчатый.
& Ритм --- закономерное чередование композиционных элементов, их повторяемость, порядок их сочетания.
& Ритм не главное в произведении фотоискусства, хотя дает возможность получить иногда строгий и стройный, иногда очень нарядный рисунок изображения.
& Цель ритмических построений --- выявление характерных особенностей объекта съемки, раскрытие смысла, содержания картины, живописно-художественная трактовка темы.
\end{easylist}
\section{Беседа восьмая}{О других проблемах изобразительного решения снимка}
\begin{easylist}
& Цельность рисунка требует прочных внутрикадровых связей элементов, образующих композицию. 
&& Композиция распадается на части, если состоит из двух основных элементов, находящихся в противоположных частях плоскости и разделенных свободным пространством. 
&& Цельность может быть нарушена, если четкие горизонтальные или вертикальные линии идут параллельно границам кадра; особенно, если линии совпадают с центральными осями прямоугольника кадра.
& Световая разработка фона, связанная с освещением главного объекта изображения, помогает правдиво передать обстановку, среду. 
& Соотношение фона и главного объекта изображения по степени резкости способствует передаче пространства.
& Лаконизм предлагает простоту конструктивного построения снимка, использование минимума элементов для раскрытия темы.
& Границы хорошо скомпонованного снимка обусловлены размещением материала в пределах кадра; опираются на фигуры и предметы, находящиеся у краев кадра.
\end{easylist}
\section{Беседа девятая}{Свет в природе и фотографии}
\important{Источники света на натурных съемках}. Главный источник света - солнце, даже если съемка ведется в пасмурную погоду и свет достигает объекта лишь будучи рассеянным плотным слоем облаков. Со стороны источника света на предмете образуются высокие яркости - \textit{света}. С противоположной стороны возникают \textit{собственные тени}. Сам освещаемый предмет отбрасывает на окружающие его поверхности \textit{падающую тень}. Окружающие предмет поверхности отбрасывают встречный поток отраженного рассеянного света. В тени отраженный свет виден достаточно хорошо, он высветляет теневые участки. Эту подсветку называют \textit{рефлексом}. Поверхности, ограничивающие объем предмета, обращены к свету под различными углами, поэтому одни из них выглядят темнее, другие светлее. Так появляются не только тени, а еще и \textit{полутени} - мягкие переходы от ярких светов к глубоким теням. В местах, где луч света отразился от освещаемой поверхности под углом зеркального отражения, возникают \textit{блики}.
\begin{itemize}
	\item \important{Рассеянный свет неба}. Дневное освещение может дать и очень мягкий рисунок светотени и очень контрастный, поскольку количество света меняется в зависимости от состояния неба. Чистое безоблачное небо дает наименьшее количество света. Облака повышают количество рассеянного света на освещаемом объекте --- контрасты светотени при этом смягчаются. Если небо сплошь затянуто легким полу-прозрачным слоем облаков, объект съемки насыщается заполняющим светом предельно и рисунок теней тогда еле намечается.
	\item \important{Фронтальный свет}: выразительный световой рисунок отсутствует, нет необходимой градации светов и теней, объемные формы и пространства на снимках передаются плохо, рисунок становится плоским, тени от предметов падают назад и скрываются за самими предметами. Пространство при фронтальном освещении очерчивается вяло. Фронтальным освещением можно воспользоваться, если фотограф задался целью передать собственные тона объекта, не изменяя их световым рисунком. Такое изображение получается очень тонким, нежным, как говорят иногда, пастельным.
	\item \important{Боковой свет}. Чередование светов и теней, освещенных и затененных участков, дает представление о пространственной протяженности объекта съемки, и снимок теряет плоскостной рисунок, который часто портит изображение, сделанное при фронтальном освещении. Боковой свет очерчивает объемы и рельефы, более отчетливо передает фактуры поверхностей и предметов.
	\item \important{Контровой свет}. Предметы, составляющие объект съемки, при контровом освещении обращены к аппарату своей теневой стороной, в кадре преобладают тени, и снимок приобретает низкую, темную тональность. Изображение становится еще более собранным в тоне, если сравнивать его с кадрами, сделанными при боковом освещении, т.к. световые блики занимают относительно небольшие площади. Пространство передается как за счет нарастания светлоты тонов глубине, так и за счет хорошей проработки воздушной среды.
\end{itemize}
Все съемочное время суток можно разделить на несколько периодов, каждый из которых имеет свои световые особенности и дает возможность получить своеобразный колорит снимка.
\begin{itemize}
	\item \important{Сумеречное освещение}. Освещенности при сумеречном освещении невелики. Небо становится достаточно ярким фоном для затемненных наземных предметов. Так возникают силуэтные или полу-силуэтные рисунки изображения, привлекающие своей собранностью в тоне, своеобразным строгим колоритом. Утреннее небо передается достаточно светлой тональностью. Вечерний закат богат красками и часто сопровождается причудливым рисунком облаков, расцвеченных лучами солнца, уже скрывшегося за горизонтом.
	\item \important{Эффектное освещение} - угол стояния солнца не выше 15$^{\circ}$. Контрасты светотени в период эффектного освещения относительно невысоки. Количество рассеянного света велико и тени получают хорошую подсветку. Над землей часто поднимается легкая туманная дымка, тоже смягчающая контрасты и входящая сама по себе выразительным компонентом в фото-картину.
	\item \important{Нормальное освещение} - угол стояния солнца от 15$^{\circ}$ до 60$^{\circ}$. На объекте имеется выразительная светотень, в достаточной мере смягченная рассеянным светом неба. Тени постепенно становятся более короткими, но светотеневой рисунок остается живописным и выразительным, хорошо очерчивает объемы, выявляет фактуры и пространственные положения предметов.
	\item \important{Зенит}. Тени очень коротки, а их контрасты высоки. Меняется баланс освещения вертикальных и горизонтальных поверхностей объекта --- первые освещены значительно меньше, чем вторые. Небо приобретает повышенную яркость. При съемке в этот период времени, трудно ожидать живописных снимков.
\end{itemize}
\section{Беседа десятая}{Тональное решение снимка}
Разнообразные яркие и пастельные тона при воспроизведении средствами черно-белой фотографии преобразуются в гамму черных, серых и белых тонов, различающихся между собой только по \important{светлоте}. Тона различной светлоты, так же как и цвета в картине, не могут распределяться на картинной плоскости произвольно, они требуют определенного согласования.\\
Подобно тому как, анализируя художественное решение живописного полотна, говорят о золотистом колорите или о картине, выдержанной в голубых тонах, точно так же говорят и о светлой или темной тональности черно-белого снимка. Таким образом, называют тон, преобладающий, доминирующий в картине. \important{Доминирующий тон} является основным, и с ним согласуются все остальные тона и полутона. Они обозначаются на снимке более легкими "мазками" и не разрушают единства тонального решения общей светлой или приглушенной темной тональностью снимка.\\
Добиваясь образного решения темы, фотограф использует то или иное тональное решение снимка, как действенное средство выполнения своих замыслов. При известном уровне мастерства, средства образования тонального рисунка становятся управляемыми и тонкая градация тонов возникает последовательно во всех звеньях процесса создания снимка.
\section{Беседа одиннадцатая}{Пространственность кадра}
Поскольку снимок двухмерен, возможность прямой передачи на снимке глубины пространства отсутствует. Порой это приводит к образованию совершенно плоских изображений, где не чувствуется расстояний, отделяющих передний план от предметов удаленных.\\
Восприятие человеком пространства связано с реальными проявлениями \important{линейной перспективы}. Основные ее закономерности можно сформулировать так:
\begin{itemize}
	\item Фигуры и предметы кажутся тем меньшими, чем дальше они находятся.
	\item Параллельные линии, уходящие в даль, обнаруживают стремление сойтись в одной точке.
	\item Грани предметов, направленные по лучу зрения глаза наблюдателя, кажутся короче, чем в действительности.
\end{itemize}
На перспективу фотоизображения оказывают влияние расстояние, с которого ведется съемка; высота установки фотоаппарата; направление оптической оси объектива по отношению к снимаемому объекту. Фокусное расстояние объектива на перспективу \important{не влияет}.\\
\important{Центральные точки} съемки не являются самыми распространенными. Перспектива таких кадров, как правило, ослаблена. Смещение точки вверх или вниз может добавить изображению перспективы.\\
В случае съемки с \important{боковых точек}, основные композиционные линии устремляются к боковым точкам схода и раскрывают пространственную протяженность объекта. Эффект будет тем большим, чем больше смещена точка в сторону от ее центрального положения. Направление композиционных линий при этом постепенно приближается к диагонали прямоугольника кадра, благодаря ярко выраженному сходу линий рисунок приобретает подчеркнутую направленность и вследствие этого динамичность.\\
\important{Нормальными} по высоте принято считать такие точки съемки, которые находятся на уровне глаз человека. Предметы и пространства, снятые с таких точек, рисуются такими же, как мы привыкли видеть их в жизни.\\
При съемке с \important{нижней} точки изменяется привычное сопоставление переднего и дальнего плана по высоте: даже невысокие предметы, находящиеся на переднем плане, часто проецируются на фон неба или оказываются на одной высоте с большими сооружениями дальнего плана. Зрителю кажется, что предметы переднего плана стали высокими, масштабными.\\
При \important{верхней} точке съемки предметы переднего плана проецируются на фон земли, кажется, что они потеряли свою высоту, принижены.\\
Понятие \important{ракурс} связано со съемкой под углом, с верхних или нижних точек, которые должны быть и достаточно близкими.При съемке в ракурсе резко сокращаются по длине линии, уходящие от наблюдателя в глубину, масштабы изображения отдельных деталей объекта с их удалением от точки съемки. Ракурс в ряде случаев помогает вскрыть сущность происходящего, выявить и подчеркнуть характерные особенности объекта.\\
Плоскостные решения снимка имеют право на жизнь. Часто они имеют своеобразный линейный ритм и в ряде случаев, связанных с особыми характеристиками объектов съемки, становятся основой интересных линейных построений.
\section{Беседа двенадцатая}{Пленэр у фотографов}
\subsection{Воздушная перспектива}
\begin{easylist}
& \define{Тональная (воздушная) перспектива}{изменение цветов и тонов предметов, обусловленное толщиной воздушного слоя}
& Особенно заметно в случаях, когда воздух насыщен влагой, пылью, дымами и пр.
& Восприятие пространства связано с закономерностями воздушной перспективы:
&& Четкость и ясность очертаний теряются по мере удаления
&& Уменьшается насыщенность  цветов, теряется яркость в отдалении
&& Контрасты светотени в глубине смягчаются
&& Угасают блики и рефлексы
&& Глубина, дали кажутся более светлыми, чем передний план.
\end{easylist}
\subsection{Воздушные дымки}
\begin{easylist}
& Воздушная дымка образуется вследствие рассеяния света в земной атмосфере.
& Прозрачность воздуха уменьшается с увеличением толщины воздушного слоя, а также когда воздух ярко освещен.
& Чем больше в воздухе взвешено мельчайших частиц (пыль, дымы, капельки влаги), тем сильнее будет рассеиваться свет и тем плотнее будет воздушная дымка.
\end{easylist}
\subsection{Влияние характера освещения}
\begin{easylist}
& При контровом направлении света, лучи, встречаясь с капельками влаги или кристалликами льда, отражаются под углами, близкими к углам зеркального отражения. Взвешенные частицы начинают бликовать, а вся воздушная среда приобретает повышенную яркость.
& Так же, при контровом освещении, фигуры и предметы обращены к фотоаппарату теневой стороной, а на таком притемненном фоне светлая дымка читается особенно отчетливо.
& Невыгоден для воспроизведения воздушной дымки передний свет, освещающий объект съемки со стороны фотоаппарата.
\end{easylist}
\end{document}