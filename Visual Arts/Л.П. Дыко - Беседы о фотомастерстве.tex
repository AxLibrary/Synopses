\documentclass{article}
% Кодировка, поддержка русского языка
\usepackage[T2A]{fontenc}
\usepackage[utf8]{inputenc}
\usepackage[english,russian]{babel}
% Отступы от края страницы
\usepackage{geometry}
\geometry{left=2cm}
\geometry{right=2cm}
\geometry{top=2cm}
\geometry{bottom=2cm}
\geometry{bindingoffset=0cm}
% Вложенные списки
\usepackage[ampersand]{easylist}
\ListProperties(Hide=100,Hang=true,Progressive=3.5ex,Style*=$\triangleright$ )

\newcommand{\note}[1]{\textit{#1}}
\newcommand{\important}[1]{\textbf{#1}}
\newcommand{\enquote}[1]{,,#1''}
\renewcommand{\section}[2]{
	\vspace{6em}
	\begin{flushright}
		\Large
		\baselineskip=0.5\baselineskip
		\textbf{#1}
		\\
		\rule[0.5\baselineskip]{\textwidth}{0.15pt}
		\\
		\textbf{#2}
	\end{flushright}
}
\renewcommand{\subsection}[1]{
	\vspace{1em}
	\begin{flushright}
		\large
		\textbf{#1}
	\end{flushright}
}
\renewcommand{\title}[2]{
	\begin{center}
		\LARGE
		\baselineskip=0.5\baselineskip
		\textbf{#1}
		\\
		\rule[0.5\baselineskip]{0.7\textwidth}{0.15pt}
		\\
		\textbf{#2}
		\\\baselineskip=2\baselineskip(конспект)		
	\end{center}
}
\newcommand{\define}[2]{
	\textbf{#1} --- #2
}
\begin{document}
\title{Л.П. Дыко}{Беседы о фотомастерстве}
\section{Беседа первая}{Путь в фотоискусство}
\begin{easylist}
& Технический прием --- не является целью творчества.
&& Средство раскрытия творческого замысла, выражения содержания.
& Мастерство --- найти яркие проявления жизни и воплотить наблюдения в образах.
& Зритель оценивает не световые пятна и линейные сходы, а картины жизни и характеры.
\end{easylist}
\section{Беседа вторая}{Искусство фотографии и его изобразительные средства}
\begin{easylist}
& Изобразительные средства фотографии: линейная композиция, световое решение, тональный рисунок.
& Возможно творческое использование технических средств: экспозиция, выдержка, глубина резкости
\end{easylist}
\section{Беседа третья}{Изобразительное решение темы}
\begin{easylist}
& Начало формирования кадра --- отбор материала, введение деталей, связанных с сюжетом.
& Без замысла и сюжета, не может быть найдено решение снимка.
& Решение темы начинается с определения смыслового центра сюжета, осмысления материала.
& Необходимо выделить эти элементы и изобразить отчетливо, выпукло, объемно.
& Смысловой центр снимка должен стать зрительным центром.
\end{easylist}
\section{Беседа четвертая}{Смысловой и изобразительный центр кадра}
\begin{easylist}
& Отчетливому изображению главного объекта способствуют:
&& Сюжетный центр в крупном масштабе.
&& Главный объект на переднем плане.
&& Тональное различие объекта и фона.
&& Световой акцент с высокой яркостью на главном элементе.
&& Наводка на резкость по главному объекту изображения, потеря резкости на фоне.
&& Направление основных линий в кадре, ведущих к центру композиции.
&& Главный объект изображения в центре или близко к нему.
&& Контрастный тональный рисунок на главном и мягкая градация тонов на фоне.
\end{easylist}
\section{Беседа пятая}{О принципах заполнения картинной плоскости}
\begin{easylist}
& Получение композиционно слаженного снимка основано на интуиции и таланте фотографа.
& Можно получить различные результаты с помощью одной и той же техники.
\end{easylist}
\section{Беседа шестая}{Уравновешенные композиции}
\begin{easylist}
& Самая строгая уравновешенность --- композиция из симметрично расположенных элементов.
& Фронтальная композиция --- статичность изображения, спокойствие и строгость в трактовке материала.
&& Видна только одна из плоскостей, ограничивающих объемные предметы в центре кадра.
&& Главные оси элементов картины совпадают с осью всей картины.
&& Приводит к потере глубины, ослабленной передаче объемов.
& Размещение главного объекта в центре - уравновешенность композиции.
&& Ограниченное применение при съемке сюжетов, связанных с движением.
&& Зафиксированный в центре объект теряет динамизм.
& Равновесие может быть достигнуто введением в общую картину композиции элементов освещения.
&& Светотеневой рисунок, световое пятно, блик.
& Движение может быть уравновешено свободным пространством по направлению.
& Важные элементы, привлекающие внимание, легко уравновешивают менее значимые компоненты.
& Композиция может быть уравновешена по вертикали.
&& Нижняя часть картины выступает в роли опоры.
& Равновесие может быть нарушено для достижения художественного эффекта.
\end{easylist}
\section{Беседа седьмая}{Ритмический рисунок кадра}
\begin{easylist}
& Линия на фотографии --- производная от тона. 
&& Представляет раздел двух тонов.
&& Может быть ясной и четкой, мягкой, расплывчатой.
& Ритм --- закономерное чередование композиционных элементов, их повторяемость, порядок их сочетания.
&& Ритм не главное.
&& Дает возможность получить иногда строгий и стройный, иногда нарядный рисунок.
&& Выявляет характерные особенности; раскрывает смысл.
&& Демонстрирует живописно-художественную трактовку темы.
\end{easylist}
\section{Беседа восьмая}{О других проблемах изобразительного решения снимка}
\begin{easylist}
& Цельность требует прочных внутрикадровых связей элементов композиции.
& Композиция может распаться на части.
&& Два основных элемента в противоположных частях и разделены свободным пространством.
&& Четкие горизонтальные или вертикальные линии параллельно границам кадра.
&&& Особенно, если совпадают с центральными осями прямоугольника кадра.
& Световая разработка фона помогает правдиво передать обстановку, среду.
&& Связанна с освещением главного объекта изображения.
& Соотношение фона и главного объекта изображения по степени резкости.
&& Акцентирует внимание на главном объекте.
&& Может способствовать передаче пространства.
& Лаконизм предлагает простоту конструктивного построения снимка.
&& Использование минимума элементов для раскрытия темы.
& Границы хорошо скомпонованного снимка обусловлены размещением материала в кадре.
&& Опираются на фигуры и предметы, находящиеся у краев кадра.
\end{easylist}
\section{Беседа девятая}{Свет в природе и фотографии}
\subsection{Свет в природе}
\begin{easylist}
& Главный источник света - солнце.
&& Даже когда свет рассивается слоем облаков.
& Со стороны источника света на предмете образуются высокие яркости --- света.
& С противоположной от источника света стороны возникают собственные тени.
& Освещаемый предмет отбрасывает на окружающие поверхности падающую тень.
& Рефлекс --- встречный поток отраженного рассеянного света от окружающих поверхностей.
&& В тени отраженный свет виден достаточно хорошо и высветляет теневые участки.
& Полутени --- мягкие переходы от ярких светов к глубоким теням.
&& Поверхности предмета обращены к свету под различными углами
&& Одни выглядят темнее, другие --- светлее.
& Блики --- лучи света, отразившиеся от освещаемой поверхности под углом зеркального отражения.
\end{easylist}
\subsection{Источники света на натурных съемках}
\begin{easylist}
& Дневное освещение может дать и мягкий рисунок светотени и контрастный.
&& Количество света меняется в зависимости от состояния неба.
&& Безоблачное небо дает наименьшее количество света.
&& Облака повышают количество рассеянного света на освещаемом объекте.
&&& Смягчает контрасты светотени.
&& При затянутом легким слоем облаков небе, рисунок теней еле намечается.
&&& Объект съемки предельно насыщается заполняющим светом.
\end{easylist}
\subsection{Направление солнечного света}
\begin{easylist}
& Световой рисунок кадра зависит от направления освещения.
& Фронтальное освещение.
&& Выразительный световой рисунок отсутствует.
&& Нет необходимой градации светов и теней.
&& Объемные формы и пространства плохо передаются.
&& Рисунок становится плоским.
&& Тени от предметов падают назад и скрываются за самими предметами.
&& Пространство очерчивается вяло.
&& Позволяет передать собственные тона объекта, не изменяя их световым рисунком.
&&& Тонкое, нежное, пастельное изображение.
& Боковое освещение.
&& Дает представление о пространственной протяженности объекта.
&&& Чередование светов и теней, освещенных и затененных участков.
&& Снимок теряет плоскостной рисунок.
&& Очерчиваются объемы и рельефы.
&& Передаются фактуры поверхностей и предметов.
& Контровое освещение.
&& Снимок приобретает темную тональность.
&&& Предметы обращены теневой стороной, в кадре преобладают тени.
&& Bзображение становится еще более собранным в тоне.
&&& Блики занимают небольшие площади.
&& Хорошо передается пространство.
&&& Нарастание светлоты тонов в глубине
&&& Проработка воздушной среды.
& Рассеянный свет.
&& Одинаковая освещенность по всему кадру.
&& Изображение возникает за счет различных яркостей отдельных участков.
&& Организующее начало --- гармоничное сочетание тонов и цветов объекта.
&&& Множество случайно расположенных тонов даст плохой результат.
& Сочетание направленного и рассеянного света.
&& Направление солнечных лучей влияет на форму и раскладку светотени.
&& Количество рассеянного света влияет на контрасты светотени.
\end{easylist}
\subsection{Изменение характера освещения с течением времени дня}
\begin{easylist}
& Все съемочное время суток можно разделить на несколько периодов.
&& Отличаются световыми особенностями и своеобразным колоритом снимка.
& Сумеречное освещение.
&& Освещенность невелика.
&& Небо выступает ярким фоном для затемненных наземных предметов.
&&& Дает силуэтные рисунки со строгим колоритом.
&& Утреннее небо передается светлой тональностью.
&& Вечерний закат богат красками
&&& Облака часто подсвечены солнцем, скрывшимся за горизонтом.
& Эффектное освещение.
&& Угол стояния солнца не выше 15$^{\circ}$
&& Контрасты светотени относительно невысоки.
&&& Количество рассеянного света велико, тени получают подсветку.
&& Над землей часто поднимается легкая туманная дымка.
&&& Смягчает контрасты.
&&& Выразительный компонент фото-картины.
& Нормальное освещение.
&& Угол стояния солнца от 15$^{\circ}$ до 60$^{\circ}$.
&& Выразительная светотень, достаточно смягченная рассеянным светом неба.
&& Тени становятся более короткими.
&& Светотеневой рисунок выразителен.
&&& Очерчивает объемы.
&&& Выявляет фактуры и пространственные положения предметов.
& Зенит.
&& Тени коротки, а контрасты высоки.
&& Вертикальные поверхности освещены значительно меньше горизонтальных.
&& Повышенная яркость неба.
&& Тяжело ожидать хороших снимков.
\end{easylist}
\section{Беседа десятая}{Тональное решение снимка}
\begin{easylist}
& На черно-белых снимках присутствует гамма черных, серых и белых тонов.
&& Отличаются только по светлоте.
& Тона различной светлоты не могут распределяться произвольно.
&& Требуют согласования.
& Минорные тона и полутона должны быть согласованны c основным доминирующим тоном.
&& Обозначаются на снимке более легкими \enquote{мазками}.
&& Не разрушаютединства тонального решения.
& Тональное решение снимка --- средство выполнения замысла фотографа.
\end{easylist}
\section{Беседа одиннадцатая}{Пространственность кадра}
\begin{easylist}
& Возможность прямой передачи на снимке глубины пространства отсутствует.
&& Может привести к образованию совершенно плоских изображений.
& Восприятие пространства связано с проявлениями линейной перспективы:
&& Фигуры и предметы кажутся тем меньшими, чем дальше они находятся.
&& Параллельные линии, уходящие в даль, обнаруживают стремление сойтись.
&& Грани предметов, направленные по лучу зрения глаза наблюдателя, кажутся короче.
& На перспективу оказывают влияние:
&& Расстояние.
&& Высота установки фотоаппарата.
&& Направление оптической оси по отношению к объекту. 
& Фокусное расстояние объектива на перспективу не влияет.
\end{easylist}
\subsection{Съемка с различных точек}
\begin{easylist}
& Центральные точки съемки не являются распространенными.
&& Перспектива таких кадров ослаблена.
&& Смещение точки вверх или вниз может добавить перспективы.
& Съемка с боковых точек.
&& Раскрывается пространственная протяженность объекта.
&&& Композиционные линии направлены к боковым точкам схода.
&&& Эффект усиливается при смещении точки схода от центрального положения.
&& Благодаря сходу линий, рисунок приобретает направленность и динамичность.
& Нормальные по высоте точки.
&& Находятся на уровне глаз.
&& Предметы и пространства выглядят естественно.
& Съемка с нижней точки.
&& Изменяется сопоставление переднего и дальнего планов по высоте.
&& Невысокие предметы переднего плана кажутся выше.
& Съемка с верхней точки.
&& Предметы переднего плана теряют высоту.
& \important{Ракурс} --- съемка с верхних или нижних точек с близкого расстояния.
&& Создает необычную перспективу, отличающуюся от обычной.
&&& Линии, уходящие в глубину, резко сокращаются по длине.
&&& С удалением от точки съемки уменьшаются масштабы отдельных деталей объекта.
&& Помогает выявиться и подчеркнуть характерные особенности объекта.
& Плоскостные решения имеют право на жизнь.
&& Часто имеют своеобразный линейный ритм.
&& В ряде случаев, становятся основой интересных линейных построений.
\end{easylist}
\section{Беседа двенадцатая}{Пленэр у фотографов}
\subsection{Воздушная перспектива}
\begin{easylist}
& \define{Тональная (воздушная) перспектива}{изменение цветов и тонов, обусловленное толщиной воздушного слоя}
& Особенно заметно в случаях, когда воздух насыщен влагой, пылью, дымами и пр.
& Восприятие пространства связано с закономерностями воздушной перспективы:
&& Четкость и ясность очертаний теряются по мере удаления
&& Уменьшается насыщенность  цветов, теряется яркость в отдалении
&& Контрасты светотени в глубине смягчаются
&& Угасают блики и рефлексы
&& Глубина, дали кажутся более светлыми, чем передний план.
\end{easylist}
\subsection{Воздушная дымка}
\begin{easylist}
& Воздушная дымка образуется вследствие рассеяния света в земной атмосфере.
&& Прозрачность воздуха уменьшается:
&&& С увеличением толщины воздушного слоя.
&&& Когда воздух ярко освещен.
&&& Чем больше в воздухе взвешено мельчайших частиц (пыль, дымы, капельки влаги).
& Контровое освещение выгодно выделяет воздушную дымку.
&& Лучи, встречаясь c частицами в воздухе, отражаются близко к углам зеркального отражения.
&& Взвешенные частицы начинают бликовать.
&& Вся воздушная среда приобретает повышенную яркость.
&& Фигуры обращены теневой стороной, светлая дымка отчетливо читается.
& Невыгодно для воспроизведения воздушной дымки фронтальное освещение.
\end{easylist}
\subsection{Тональная перспектива}
\begin{easylist}
& Тональную перспективу можно получить при отсутствии воздушной дымки.
&& Отчетливый передний план и отсутствие резкости в глубине.
&& Затемненный передний план и более светлая глубина.
& Нарушение закономерностей тональной перспективы приводит к потере пространственности.
& Для выявления перспективы можно использовать передний план.
&& Содержит второстепенные элементы.
&& Должен иметь темную тональность, быть отчетливым и резким.
&& Помогает ощутить пространство до объекта съемки за счет разницы размеров.
\end{easylist}
\section{Беседа тринадцатая}{Передача объемов на фотоснимке}
\end{document}