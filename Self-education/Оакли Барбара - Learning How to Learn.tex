\documentclass{article}

\usepackage[T2A]{fontenc}
\usepackage[utf8]{inputenc}
\usepackage[russian,english]{babel}
\usepackage[left=2cm,right=2cm,top=2cm,bottom=2cm,bindingoffset=0cm]{geometry}

\usepackage{sectsty}
\allsectionsfont{\centering}

\parindent=0cm
\begin{document}
\title{Оакли Барбара --- Learning How to Learn\\(основные тезисы курса)}
\author{Tass}
\date{2016/06/30}
\maketitle

\section{Два режима мышления}
В целом, человек обладает двумя принципиально разными режимами мышления, которые можно назвать сфокусированным и рассеянным.
\medskip

\textbf{Сфокусированное мышление}, в основном, используется для сосредоточения на пунктах, которые уже тесно связаны в вашем сознании. В данном режиме легче обдумывать конкретную мысль. Если провести аналогию с пинбольным автоматом, где каждый бампер соответствует нейрону мозга, то данный режим можно охарактеризовать наличием маленького расстояния между резиновыми бамперами, из-за чего мысль, представленная запущенным шариком, может перемещаться лишь на небольшом участке поля.
\medskip

\textbf{Рассеянное мышление} позволяет шире взглянуть на вещи, с разных точек зрения на общую картину; образовать новые нейронные связи посредством путешествия по новым маршрутам. В пинбольном автомате, рассеянному мышлению соответствует поле с большим расстоянием между бамперами, благодаря чему мысль может проследовать через всё поле.
\medskip

Разницу между рассеянным и сфокусированным мышлением можно также проиллюстрировать аналогией с ручным фонариком, у которого есть два режима: сфокусированный луч четко высвечивает небольшое пространство, а рассеянный свет освещает большую зону без отчетливого выделения конкретных предметов.
\medskip

\textbf{Переключение между режимами} помогает в работе над проблемой. Стоит отвлечься от задачи и в действие вступает рассеянный режим, позволяющий мысли пройти по широкому пространству, что может привести к нахождению верного решения. Так же, переключение способствует лучшему усвоению материала, позволяя избежать эффекта установки.
\medskip

\textbf{Эффект установки} --- неудача с освоением новых понятий или решением задач, обусловленная фиксацией на неверном подходе. Избавиться от такого эффекта можно путем переключения мышления со сфокусированного на рассеянное.

\section{Порции информации}
\section{Память}
\textbf{Две основные системы памяти}: рабочая память способна вместить ~четыре объекта; долговременная память способна хранить большое количество материала.
\medskip

\section{Прокрастинация}
\section{Способы преуспеть}
\textbf{Стараться вспоминать}. После каждой прочитанной страницы отведите от нее взгляд и вспомните основные идеи. Не выделяйте (например, подчеркиванием) большое количество текста на странице и никогда не отмечайте то, чего предварительно не закрепили в памяти. Пытайтесь вспоминать учебный материал в условиях, где вы не занимались им изначально. Способность вспоминать, т.е. генерировать идеи изнутри сознания, --- один из ключевых показателей эффективной учебы.
\smallskip

\textbf{Проверять себя}. Во всем. Постоянно. Карточки с информацией --- ваш постоянный спутник.
\smallskip

\textbf{Создавать порции информации}. Формировать порции информации --- значит понимать суть задачи и заниматься ее решением таким образом, что бы весь ход решения разом приходил в голову. После того, как вы решили задачу, повторите процесс и убедитесь, что вы знаете без подсказки каждый этап решения. Сделайте вид, будто это песня, и приучитесь прокручивать ее в голове снова и снова, что бы информация оформилась в одну удобную порцию, которую вы можете вытащить из памяти в любой момент.
\smallskip

\textbf{Следовать правильному режиму повторения материала}. Каждый день повторяйте немного больше, чем накануне, --- так же, как тренируются спортсмены. Мозг --- аналог мышц: за один раз он может выполнить ограниченное количество упражнений по одному учебному предмету.
\smallskip

\textbf{Применять разные подходы, практикуясь в решении задач}. Никогда не используйте один и тот же способ решения задачи слишком долго в течение одного занятия, иначе через некоторое время вы начнете механически применять его к другим задачам, которым этот метод не подходит. Чтобы усвоить, каким образом и в каких случаях использовать данный метод решения, беритесь за самые разные типы задач. После каждого задания и теста делайте работу над ошибками: убедитесь, что понимаете свою ошибку и затем решите задачу правильно.
\smallskip

\textbf{Делать перерывы}. Невозможность решить задачу или усвоить понятие с первого раза --- обычное дело, поэтому недолгие ежедневные занятия гораздо лучше, нежели долгое однократное занятие. Когда вас начинаете раздражать задача, сделайте перерыв, чтобы ей занимался другой участок мозга в фоновом режиме. Также, перерывы способствуют перемещению порций информации из рабочей памяти в долговременную.
\smallskip

\end{document}