\documentclass{article}

\usepackage[T2A]{fontenc}
\usepackage[utf8]{inputenc}
\usepackage[english,russian]{babel}
\usepackage[left=2cm,right=2cm,top=2cm,bottom=2cm,bindingoffset=0cm]{geometry}

\usepackage{sectsty}
\allsectionsfont{\centering}

\parindent=0cm
\begin{document}
\title{Оакли Барбара --- Learning How to Learn\\(основные тезисы курса)}
\author{Tass}
\date{2016/06/30}
\maketitle

\section{Два режима мышления}
В целом, человек обладает двумя принципиально разными режимами мышления, которые можно назвать сфокусированным и рассеянным.

\begin{itemize}
\item \textbf{Сфокусированное мышление}, в основном, используется для сосредоточения на пунктах, которые уже тесно связаны в вашем сознании. В данном режиме легче обдумывать конкретную мысль. Если провести аналогию с пинбольным автоматом, где каждый бампер соответствует нейрону мозга, то данный режим можно охарактеризовать наличием маленького расстояния между резиновыми бамперами, из-за чего мысль, представленная запущенным шариком, может перемещаться лишь на небольшом участке поля.
\item \textbf{Рассеянное мышление} позволяет шире взглянуть на вещи, с разных точек зрения на общую картину; образовать новые нейронные связи посредством путешествия по новым маршрутам. В пинбольном автомате, рассеянному мышлению соответствует поле с большим расстоянием между бамперами, благодаря чему мысль может проследовать через всё поле.
\item Разницу между рассеянным и сфокусированным мышлением можно также проиллюстрировать аналогией с ручным фонариком, у которого есть два режима: сфокусированный луч четко высвечивает небольшое пространство, а рассеянный свет освещает большую зону без отчетливого выделения конкретных предметов.
\item \textbf{Переключение между режимами} помогает в работе над проблемой. Стоит отвлечься от задачи и в действие вступает рассеянный режим, позволяющий мысли пройти по широкому пространству, что может привести к нахождению верного решения. Так же, переключение способствует лучшему усвоению материала, позволяя избежать эффекта установки.
\item \textbf{Эффект установки} --- неудача с освоением новых понятий или решением задач, обусловленная фиксацией на неверном подходе. Избавиться от такого эффекта можно путем переключения мышления со сфокусированного на рассеянное.
\end{itemize}

\section{Порции информации}
\begin{itemize}
\item \textbf{Порция информации} --- ментальная связка, объединяющая отдельные фрагменты информации через общий смысл.
\item \textbf{Формирование} порции информации происходит в 3 шага. Сперва необходимо сосредоточиться на информации, которую необходимо сформировать в порцию. Далее необходимо понять основную идею, которая будет превращена в порцию. В-третьих --- необходимо накапливать контекст, что бы знать не только то, как применять данную порцию информации, но и то, когда ее применять.
\item \textbf{Формирование порции пошагово}: решить задачу; решить задачу заново, обращая внимания на главные этапы; сделать перерыв; поспать; решить задачу еще раз; перейти к другой задаче; делать активные повторения.
\item \textbf{Абстрактное обобщение} позволяет переносить идеи из одной сферы в другую. Хорошие порции информации создают нейронные паттерны, которые могут действовать в рамках не только изучаемого предмета, но и других дисциплин и сфер жизни.
\end{itemize}

\section{Память и обучение}
\begin{itemize}
\item \textbf{Две основные системы памяти}: рабочая память способна вместить ~четыре объекта; долговременная память способна хранить большое количество материала.
\item \textbf{Попытка вспомнить материал}, который вы изучаете, --- т.е. практика извлечения данных из памяти, --- гораздо эффективней простого перечитывания. В этом случае, вы будете более сосредоточены и потратите время эффективней.
\item \textbf{Практика}, наряду с воспроизведением материала по памяти, дают больше знаний и на более глубоком уровне, нежели любые другие подходы.
\item \textbf{Закон озарений}: удача благоволит к тем, кто прилагает усилия и стремится к цели. Изучение второго понятия будет проще, нежели изучение первого.
\item \textbf{Интерливинг}, или чередование, --- это вид деятельности, при котором вы имеете дело с разными типами задач, решение которых требует разных стратегий. Способствует развитию умения выбирать и применять нужный способ решения задачи.
\item \textbf{Любой тест} --- мощный обучающий опыт. Он реорганизует знания и добавляет к ним новые, а также ощутимо улучшает способность удерживать материал в памяти. 
\item \textbf{Мнемонические техники} --- мощный инструмент, позволяющий упростить запоминание. Использование зрительных образов может помочь усвоить понятия, с виду запутанные и труднозапоминаемые. Метод "дворца памяти" основан на использовании воспоминаний о знаком месте в качестве визуального блокнота: на него можно наложить образы-понятия, которые необходимо запомнить. 
\item \textbf{Метафоры} --- помогают гораздо быстрее запоминать сложные идеи; позволяют сведениям из разных областей взаимодействовать друг с другом.	 Пример --- использованная в начале аналогия с пинбольным автоматом и фонариком.
\item \textbf{Осмысленное объединение} фактов в группы и создание аббревиатур упрощает понимание и помогает формированию порций информации. Рассказы и истории, даже придуманные как мнемонические приемы, позволяют легче удержать в памяти изучаемый материал.
\item \textbf{Перенос} --- способность применить то, что вы изучаете, в новом для себя контексте. Выучив один иностранный язык, второй освоить проще.
\item \textbf{Проверка} --- важная часть обучения. Перепроверка даст более широкий взгляд на сделанное, поскольку задействует другие нейронные процессы, позволяющие замечать недочеты.
\end{itemize}

\section{Прокрастинация}
\begin{itemize}
\item \textbf{Привычка} содержит четыре этапа: сигнал; последовательность действий; вознаграждение; вера. Измените привычку, реагируя не так, как того требует сигнал. Вознаграждение и вера сделают такую перемену стабильной.
\item \textbf{Сосредоточиться на процессе}, а не на продукте. Именно мысль о продукте повергает в дискомфорт, заставляя прокрастинировать. Кроме того, отвечающая за привычки часть мозга, любит процессы, поскольку, пока они идут, можно двигаться вперед. Фокусируясь на процессе, вы выходите из режима постоянного самоконтроля и беспокойства. 
\item \textbf{Метод помидора} --- использование 25-минутных отрезков, во время которых фокусировка идет на изучаемом материале, а всё, что может отвлекать --- игнорируется. Вознаграждайте себя после каждого успешного отрезка.
\item \textbf{Планирование} --- один из самых действенных способов борьбы с прокрастинацией. Список дел на день, составленный вечером, поможет настроиться на рабочий лад. Разделение работы на отдельные небольшие порции --- поможет мозгу настроиться на процесс. После достижения результат, обязательно стоит вознаградить себя.
\end{itemize}

\section{Best practice}
\begin{itemize}
\item \textbf{Стараться вспоминать}. После каждой прочитанной страницы отведите от нее взгляд и вспомните основные идеи. Не выделяйте (например, подчеркиванием) большое количество текста на странице и никогда не отмечайте то, чего предварительно не закрепили в памяти. Пытайтесь вспоминать учебный материал в условиях, где вы не занимались им изначально. Способность вспоминать, т.е. генерировать идеи изнутри сознания, --- один из ключевых показателей эффективной учебы.
\item \textbf{Проверять себя}. Во всем. Постоянно. Карточки с информацией --- ваш постоянный спутник.
\item \textbf{Создавать порции информации}. Формировать порции информации --- значит понимать суть задачи и заниматься ее решением таким образом, что бы весь ход решения разом приходил в голову. После того, как вы решили задачу, повторите процесс и убедитесь, что вы знаете без подсказки каждый этап решения. Сделайте вид, будто это песня, и приучитесь прокручивать ее в голове снова и снова, что бы информация оформилась в одну удобную порцию, которую вы можете вытащить из памяти в любой момент.
\item \textbf{Следовать правильному режиму повторения материала}. Каждый день повторяйте немного больше, чем накануне, --- так же, как тренируются спортсмены. Мозг --- аналог мышц: за один раз он может выполнить ограниченное количество упражнений по одному учебному предмету.
\item \textbf{Применять разные подходы, практикуясь в решении задач}. Никогда не используйте один и тот же способ решения задачи слишком долго в течение одного занятия, иначе через некоторое время вы начнете механически применять его к другим задачам, которым этот метод не подходит. Чтобы усвоить, каким образом и в каких случаях использовать данный метод решения, беритесь за самые разные типы задач. После каждого задания и теста делайте работу над ошибками: убедитесь, что понимаете свою ошибку и затем решите задачу правильно.
\item \textbf{Делать перерывы}. Невозможность решить задачу или усвоить понятие с первого раза --- обычное дело, поэтому недолгие ежедневные занятия гораздо лучше, нежели долгое однократное занятие. Когда вас начинаете раздражать задача, сделайте перерыв, чтобы ей занимался другой участок мозга в фоновом режиме. Также, перерывы способствуют перемещению порций информации из рабочей памяти в долговременную.
\item \textbf{Объяснять материал воображаемому собеседнику и пользоваться простыми аналогиями}. Когда вам не дается понятие, спросите себя: "Как бы я объяснил его десятилетнему ребенку?". Аналогии в этом случае очень полезны. Не просто прокручивайте объяснение в мыслях: проговорите его вслух или запишите. Дополнительный эффект от говорения или написания позволяет глубже закодировать изучаемую информацию.
\item \textbf{Сосредотачиваться}. Выключите все отвлекающие звонки и сигналы в телефоне и компьютере, затем включите таймер на 25 минут. На это время прицельно сконцентрируйтесь на изучаемом понятии, явлении или задаче и попытайтесь работать как можно более прилежно. По истечении этого срока, наградите себя чем-нибудь приятным или забавным. Несколько таких сеансов в день помогут ощутимо продвинуться. Выбирайте время и место так, что бы вы могли заниматься без помех и в условиях, располагающих к занятиям.
\item \textbf{Сначала съедать лягушек}. Самое сложное делайте в начале дня, на свежую голову.
\item \textbf{Помнить о своей мечте}. Окиньте мысленным взором свою нынешнюю жизнь и сравните ее с той, которой вы мечтаете жить в результате получения выбранной профессии. Повесьте над своим рабочим местом плакат с соответствующим изображением или текстом, описывающим ваше возможное будущее, --- он станет напоминать о вашей мечте. Смотрите на плакат, когда заметите, что мотивация снизилась. Этот способ полезен и для вас, и для ваших близких.
\item \textbf{Заниматься физическими упражнениями}. В результате недавних экспериментов на животных и людях, обнаружилось, что регулярные физические упражнения способствуют значительному улучшению памяти и повышению способностей к обучению. Они, судя по результатам, помогают формировать новые нейроны в участках, связанных с памятью, а также создают новые сигнальные пути. И аэробные упражнения, и упражнения с нагрузкой имеют схожее по силе, мощное воздействие на обучение и память.
\end{itemize}

\section{Worst practice}
\begin{itemize}
\item \textbf{Пассивно перечитывать, т.е. просто сидеть и скользить взглядом по тексту}. Если вы можете вспомнить главные идеи текста, не подглядывая в него, и тем самым доказать, что материал прочно осел в памяти, такой способ полезен. Если нет --- он лишь бесполезная трата времени.
\item \textbf{Делать слишком много выделений в тексте}. Выделяя часть текста, вы вводите мозг в заблуждение --- считаете, будто в голове что-то отложилось, хотя на самом деле вы всего лишь водите карандашом по странице. Подчеркивать или выделять текстом иногда и полезно для того, чтобы выдвинуть на первый план особо важные фрагменты информации. Однако, если вы выделяете текст для того, чтобы лучше его запомнить, то проверяйте, действительно ли он отложился в памяти.
\item \textbf{Заглядывать в раздел ответов и, узнав способ решения задачи, считать, будто теперь вы знаете, как ее решить}. Это одна из самых серьезных ошибок при обучении. Вы должны уметь решить задачу шаг за шагом, не заглядывая в учебник.
\item \textbf{Начинать готовиться к тестированию в последний момент}. Мозг как мышца: он способен выдерживать лишь ограниченную нагрузку в течение одного занятия по одному предмету.
\item \textbf{Раз за разом решать однотипные задачи, способ решения которых вы уже знаете}. Сидеть и упражняться в решении одного типа задач --- не значит готовиться к экзамену. Это примерно то же самое, что готовиться к баскетбольному матчу и тренировать только удары об пол. Тем не менее, это позволяет отработать действия до автоматизма.
\item \textbf{Превращать совместные занятия с друзьями в посиделки}. Совместно решать задачи и проверять друг друга --- хороший способ сделать обучение более приятным, выявить огрехи в подходах к материалу и углубить знания. Однако, если совместные занятия переходят в болтовню раньше, чем выполнены все задания, вы просто теряете время. В этом случае, лучше найти другую группу для совместных занятий.
\item \textbf{Игнорировать необходимость прочитать нужный раздел учебника, прежде чем приступить к решению задачи}. Учебник --- пособие, он показывает пути к решению задач. Если взяться за задачи, не читая учебника, вы просто потеряете время. Однако, начать работу с беглого просмотра всей главы или раздела полезно: это позволит получить общее представление о материале.
\item \textbf{Пренебрегать возможностью консультироваться с преподавателями и сокурсниками в сложных случаях}. Преподавателям не в диковинку отвечать на вопросы студентов, не понимающих материал, --- такова наша работа. Напротив, мы тревожимся из-за студентов, которые не приходят за советом. Не будьте одним из них.
\item \textbf{Считать, будто можно надежно выучить материал, если постоянно отвлекаться}. Каждая попытка отвлечься означате, что у мозга останется меньше сил на усвоение материала. Каждый раз, когда вы отвлекаетесь, нейронные связи оказываются выкорчеваны раньше, чем они успеют прорасти.
\item \textbf{Мало спать}. Во сне мозг обрабатывает методы решения задач и повторяет все то, что вы заложили в память перед сном. От постоянной усталости, в мозгу накапливаются токсины, разрушающие нейронные связи, позволяющие соображать быстро и продуктивно. Если вы не выспались, никакая подготовка вас не спасет.
\end{itemize}
\end{document}