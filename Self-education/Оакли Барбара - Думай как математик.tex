\documentclass{article}

\usepackage[T2A]{fontenc}
\usepackage[utf8]{inputenc}
\usepackage[russian,english]{babel}

\begin{document}

\title{Оакли Барбара - Думай как математик (конспект книги)}
\author{Tass}
\date{\today}
\maketitle

\newpage
\section{Откройте дверь}
Из данной книги можно узнать о мыслительных процессах; о том, как мозг усваивает новые знания; почему порой никакого обучения не происходит, хотя кажется что вы чему-то учитесь. Так же книга содержит управжнения, для развития навыков обучения.

\section{Легкость --- лучший подход}
\begin{itemize}
\item[-] Когда вы впервые начинаете просматривать главу учебника по математике или естественным наукам, полезно пробежать глазами весь раздел, составляя себе общую картину: взглянуть на диаграммы, схемы, фотогографии, заголовки разделов, выводы и даже вопросы в конце текста.
\end{itemize}
\textbf{Обобщение:}
\begin{itemize}
\item При мыслительной деятельности наш мозг находится в двух состояних --- сфокусированном и рассеяном. Он переключается из одного режима в другой, не используя оба одновременно.
\item Когда мы сталкиваемся с новыми понятиями и идеями, замешательство и непонимание --- обычная реакция.
\item Для усвоения новых понятий и решения задач важна не только начальная концентрация внимания, но и последующее расфокусирование взгляда, когда мы позволяем мозку отвлечься от предмета.
\item Эффект установки --- это когда неуспех с усвоением новых понятий или решением задач обусловлен нашей фиксацией на неверном подходе. Избавиться от такого эффекта можно путем переключения мышления со сфокусированного на рассеянное.\\Не забывайте, что гибкость мышления --- ваш помощник: режим мышления при усвоении нового материала или решении задач необходимо менять, так как первоначальный подход может оказаться неверн.
\end{itemize}

\section{Учиться --- значит творить}
\begin{itemize}
\item[-] Рассеянное мышление помогает постигать материал на глубоком и творческом уровне.
\item[-] Установка на первую же идею, которая приходит в голову при работе, может помешать найти более подходящее решение.
\end{itemize}
\textbf{Обобщение:}
\begin{itemize}
\item Сфокусированный режим мышления используется для первого знакомства с понятиями и задачами в математике и естественных науках.
\item После первого усиленного сеанса работы в сфокусированном режиме, дайте волю рассеянному состоянию.
\item Растущее недовольство и раздражение --- знак того, что пора переключить внимание: мозгу лучше перейти в рассеянное состояние и поработать в фоновом режиме.
\item Лучше заниматься в малых количествах: ежедневно и понемного. Тогда и у сфокусированного, и у рассеянного мышления будет время сделать так, чтобы вы понимали изучаемый материал. Именно таким образом создаются прочные нейронные структуры.
\item Если вам мешает прокрастинация, попробуйте поставить таймер на 25 минут и усердно сфокусируйтесь на задаче, не отвлекаясь.
\item Существует две основные системы памяти:
\begin{itemize}
\item Рабочая память --- "жонглер", у которого только четыре предмета могут находиться в воздухе.
\item Долговременная память --- "склад", способный вмещать большое количество материала, однако для того, чтобы объекты памяти оставались доступными, его нужно периодически инспектировать.
\end{itemize}
\item Перерывы между изучением материала помогают переместить нужную информацию из рабочей памяти в долговременную.
\item Сон -- жизненно важная часть процесса обучения. Он помогает:
\begin{itemize}
\item Создать нейронные связи, необходимые для нормального мыслительного процесса.
\item Находить решения сложных задач и видеть смысл того, что вы изучаете.
\item Закреплять и повторять важные пункты материала и избавляться от балласта.
\end{itemize}
\end{itemize}

\section{Порции информации и иллюзия компетентности}
\begin{itemize}
\item[-] Намерение учиться помогает только в тех случаях, когда оно приводит к использованию правильных стратегий обучения
\item[-] Интерливинг, или чередование - вид деятельности, при котором вы имеете дело с разными типами задач, решение которых требует разных стратегий.
\end{itemize}
\textbf{Обобщение:}
\begin{itemize}
\item Практические упражнения помогают создавать стойкие нейронные паттерны --- т.е. понятийные порции информации.
\item Практические управжнения делают сознание более живым и гибким.
\item Порции информации лучшего всего создаются при помощи:
\begin{itemize}
\item Сфокусированного внимания.
\item Понимания общей идеи.
\item Практики, помогающей увидеть широкий контекст.
\end{itemize}
\item Простое воспоминание --- когда вы вызываете в памяти ключевые пункты материала, не глядя в текст, --- один из лучших методов, способствующих формированию порций информации.
\end{itemize}

\section{Как не поддаться прокрастинации}
\textbf{Обобщение:}
\begin{itemize}
\item Мы прокрастинируем из-за того, что занятие, которое мы откладываем, нам неприятно. Но то, что приносит удовольствие на короткое время, не всегда полезно в долгосрочной перспективе.
\item Прокрастинация похожа на яд, принимаемый небольшими дозами. В каждом мелком случае она может показаться невинной. Однако ее отдаленные последствия могут быть разрушительны.
\end{itemize}

\section{Зомби вокруг нас}
\begin{itemize}
\item[-] Привычка --- средство экономии наших сил, она позволяет освободить сознание для других видов деятельности.
\item[-] Фокусируясь на процессе, а не на продукте, вы выходите из режима постоянного самоконтроля и беспокойства и расслабляетесь, погружаясь в рабочее состояние.
\item[-] Учитесь игнорировать желание отвлечься. Это более действенный способ, чем пытаться силой воли заставить себя не питать этого желания.
\end{itemize}
\textbf{Обобщение:}
\begin{itemize}
\item Недолгий период работы над чем-то неприятным, может в результате привести к благоприятным результам.
\item Прокрастинация и аналогичные привычки имеют четыре составляющих:
\begin{itemize}
\item Сигнал.
\item Последовательность действий.
\item Вознаграждение.
\item Веру.
\end{itemize}
\item Измените привычку, реагируя не так, как того требует сигнал, или вовсе игнорируйте сигнал. Вознаграждение и вера сделают такую перемену стабильной.
\item Сфокусируйтесь на процессе (способе времяпрепровождения), а не на продукте (желаемой цели).
\item Используйте 25-минутные отрезки "помидора" для плодотворной работы. Вознаграждайте себя после каждого успешного периода сфокусированного внимания.
\item Не забывайте об отдыхе, как тренировке режима рассеянного мышления.
\item Мысленное сравнение --- действенный мотивирующий прием: вспомните худшие черты своего прошлого или нынешнего опыта и сопоставьте их с желанным будущим.
\item Если вы делаете много дел сразу, то не способны цельно и глубого мыслить, поскольку часть мозга, отвечающая за необходимые связи, постоянно отвлекается от процесса раньше, чем нейронные связи успевают закрепиться.
\end{itemize}

\section{Порции информации или ступор}
\begin{itemize}
\item[-] Генерация, (т.е. вспоминание) материала помогает выучить его более эффективно, чем с помощью простого перечитывания.
\item[-] Любой тест сам по себе --- мощный обучающий опыт. Он реорганизует знания и добавляет к ним новые, а также ощутимо улучшает способность удерживать материал в памяти.
\end{itemize}
\textbf{Обобщение:}
\begin{itemize}
\item Сформировать порцию информации --- значит интегрировать некое понятие в четко оформленный нейронный паттерн.
\item Формирование порций помогает увеличить объем доступной рабочей памяти.
\item Создание библиотеки сформированных порций информации, содержащих необходимые концепции и способы решения, помогает развитию интуиции при решении задач.
\item При создании библиотеки порций важно сознательно фокусироваться на самых сложных понятих и аспектах решения задач.
\item Временами, как бы вы ни старались овладеть материалом, вас постигает неудача. В таких случаях не забывайте "закон озарений": если вы эффективно поработаете над материалом и создадите хорошую библиотек решений, то увидите, что удача все чаще будет на вашей стороне. Иными словами, если вы не стараетесь освоить материал, вам гарантирована неудача, однако те, кто целенаправленно прилагает усилия, достигнут гораздо больших успехов.
\end{itemize}
\end{document}