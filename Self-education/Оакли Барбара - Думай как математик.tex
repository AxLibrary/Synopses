\documentclass{article}

\usepackage[T2A]{fontenc}
\usepackage[utf8]{inputenc}
\usepackage[russian,english]{babel}

\begin{document}

\title{Оакли Барбара - Думай как математик (конспект книги)}
\author{Tass}
\date{\today}
\maketitle

\newpage
\section{Откройте дверь}
Из данной книги можно узнать о мыслительных процессах; о том, как мозг усваивает новые знания; почему порой никакого обучения не происходит, хотя кажется что вы чему-то учитесь. Так же книга содержит управжнения, для развития навыков обучения.

\section{Легкость --- лучший подход}

\begin{itemize}
\item При мыслительной деятельности наш мозг находится в двух состояних --- сфокусированном и рассеяном. Он переключается из одного режима в другой, не используя оба одновременно.
\item Когда мы сталкиваемся с новыми понятиями и идеями, замешательство и непонимание --- обычная реакция.
\item Для усвоения новых понятий и решения задач важна не только начальная концентрация внимания, но и последующее расфокусирование взгляда, когда мы позволяем мозку отвлечься от предмета.
\item Эффект установки --- это когда неуспех с усвоением новых понятий или решением задач обусловлен нашей фиксацией на неверном подходе. Избавиться от такого эффекта можно путем переключения мышления со сфокусированного на рассеянное.\\Не забывайте, что гибкость мышления --- ваш помощник: режим мышления при усвоении нового материала или решении задач необходимо менять, так как первоначальный подход может оказаться неверн.
\end{itemize}

\end{document}