\documentclass{article}

\usepackage[T2A]{fontenc}
\usepackage[utf8]{inputenc}
\usepackage[russian,english]{babel}
\usepackage[left=2cm,right=2cm,top=2cm,bottom=2cm,bindingoffset=0cm]{geometry}

\usepackage{sectsty}
\allsectionsfont{\centering}

\begin{document}
\title{Оакли Барбара --- Думай как математик\\(конспект книги)}
\author{Tass}
\date{2016/06/30}
\maketitle

\tableofcontents
\newpage

\section{Откройте дверь}
Из данной книги можно узнать о мыслительных процессах; о том, как мозг усваивает новые знания; почему порой никакого обучения не происходит, хотя кажется что вы чему-то учитесь. Так же книга содержит управжнения, для развития навыков обучения.

\section{Легкость --- лучший подход}
\begin{itemize}
\item[-] Когда вы впервые начинаете просматривать главу учебника по математике или естественным наукам, полезно пробежать глазами весь раздел, составляя себе общую картину: взглянуть на диаграммы, схемы, фотогографии, заголовки разделов, выводы и даже вопросы в конце текста.
\end{itemize}
\textbf{Обобщение:}
\begin{itemize}
\item При мыслительной деятельности наш мозг находится в двух состояних --- сфокусированном и рассеяном. Он переключается из одного режима в другой, не используя оба одновременно.
\item Когда мы сталкиваемся с новыми понятиями и идеями, замешательство и непонимание --- обычная реакция.
\item Для усвоения новых понятий и решения задач важна не только начальная концентрация внимания, но и последующее расфокусирование взгляда, когда мы позволяем мозгу отвлечься от предмета.
\item Эффект установки --- это когда неуспех с усвоением новых понятий или решением задач обусловлен нашей фиксацией на неверном подходе. Избавиться от такого эффекта можно путем переключения мышления со сфокусированного на рассеянное.\\Не забывайте, что гибкость мышления --- ваш помощник: режим мышления при усвоении нового материала или решении задач необходимо менять, так как первоначальный подход может оказаться неверн.
\end{itemize}

\section{Учиться --- значит творить}
\begin{itemize}
\item[-] Рассеянное мышление помогает постигать материал на глубоком и творческом уровне.
\item[-] Установка на первую же идею, которая приходит в голову при работе, может помешать найти более подходящее решение.
\end{itemize}
\textbf{Обобщение:}
\begin{itemize}
\item Сфокусированный режим мышления используется для первого знакомства с понятиями и задачами в математике и естественных науках.
\item После первого усиленного сеанса работы в сфокусированном режиме, дайте волю рассеянному состоянию.
\item Растущее недовольство и раздражение --- знак того, что пора переключить внимание: мозгу лучше перейти в рассеянное состояние и поработать в фоновом режиме.
\item Лучше заниматься в малых количествах: ежедневно и понемного. Тогда и у сфокусированного, и у рассеянного мышления будет время сделать так, чтобы вы понимали изучаемый материал. Именно таким образом создаются прочные нейронные структуры.
\item Если вам мешает прокрастинация, попробуйте поставить таймер на 25 минут и усердно сфокусируйтесь на задаче, не отвлекаясь.
\item Существует две основные системы памяти:
\begin{itemize}
\item Рабочая память --- "жонглер", у которого только четыре предмета могут находиться в воздухе.
\item Долговременная память --- "склад", способный вмещать большое количество материала, однако для того, чтобы объекты памяти оставались доступными, его нужно периодически инспектировать.
\end{itemize}
\item Перерывы между изучением материала помогают переместить нужную информацию из рабочей памяти в долговременную.
\item Сон -- жизненно важная часть процесса обучения. Он помогает:
\begin{itemize}
\item Создать нейронные связи, необходимые для нормального мыслительного процесса.
\item Находить решения сложных задач и видеть смысл того, что вы изучаете.
\item Закреплять и повторять важные пункты материала и избавляться от балласта.
\end{itemize}
\end{itemize}

\section{Порции информации и иллюзия компетентности}
\begin{itemize}
\item[-] Намерение учиться помогает только в тех случаях, когда оно приводит к использованию правильных стратегий обучения
\item[-] Интерливинг, или чередование - вид деятельности, при котором вы имеете дело с разными типами задач, решение которых требует разных стратегий.
\end{itemize}
\textbf{Обобщение:}
\begin{itemize}
\item Практические упражнения помогают создавать стойкие нейронные паттерны --- т.е. понятийные порции информации.
\item Практические управжнения делают сознание более живым и гибким.
\item Порции информации лучшего всего создаются при помощи:
\begin{itemize}
\item Сфокусированного внимания.
\item Понимания общей идеи.
\item Практики, помогающей увидеть широкий контекст.
\end{itemize}
\item Простое воспоминание --- когда вы вызываете в памяти ключевые пункты материала, не глядя в текст, --- один из лучших методов, способствующих формированию порций информации.
\end{itemize}

\section{Как не поддаться прокрастинации}
\textbf{Обобщение:}
\begin{itemize}
\item Мы прокрастинируем из-за того, что занятие, которое мы откладываем, нам неприятно. Но то, что приносит удовольствие на короткое время, не всегда полезно в долгосрочной перспективе.
\item Прокрастинация похожа на яд, принимаемый небольшими дозами. В каждом мелком случае она может показаться невинной. Однако ее отдаленные последствия могут быть разрушительны.
\end{itemize}

\section{Зомби вокруг нас}
\begin{itemize}
\item[-] Привычка --- средство экономии наших сил, она позволяет освободить сознание для других видов деятельности.
\item[-] Фокусируясь на процессе, а не на продукте, вы выходите из режима постоянного самоконтроля и беспокойства и расслабляетесь, погружаясь в рабочее состояние.
\item[-] Учитесь игнорировать желание отвлечься. Это более действенный способ, чем пытаться силой воли заставить себя не питать этого желания.
\end{itemize}
\textbf{Обобщение:}
\begin{itemize}
\item Недолгий период работы над чем-то неприятным, может в результате привести к благоприятным результам.
\item Прокрастинация и аналогичные привычки имеют четыре составляющих:
\begin{itemize}
\item Сигнал.
\item Последовательность действий.
\item Вознаграждение.
\item Веру.
\end{itemize}
\item Измените привычку, реагируя не так, как того требует сигнал, или вовсе игнорируйте сигнал. Вознаграждение и вера сделают такую перемену стабильной.
\item Сфокусируйтесь на процессе (способе времяпрепровождения), а не на продукте (желаемой цели).
\item Используйте 25-минутные отрезки "помидора" для плодотворной работы. Вознаграждайте себя после каждого успешного периода сфокусированного внимания.
\item Не забывайте об отдыхе, как тренировке режима рассеянного мышления.
\item Мысленное сравнение --- действенный мотивирующий прием: вспомните худшие черты своего прошлого или нынешнего опыта и сопоставьте их с желанным будущим.
\item Если вы делаете много дел сразу, то не способны цельно и глубого мыслить, поскольку часть мозга, отвечающая за необходимые связи, постоянно отвлекается от процесса раньше, чем нейронные связи успевают закрепиться.
\end{itemize}

\section{Порции информации или ступор}
\begin{itemize}
\item[-] Генерация, (т.е. вспоминание) материала помогает выучить его более эффективно, чем с помощью простого перечитывания.
\item[-] Любой тест сам по себе --- мощный обучающий опыт. Он реорганизует знания и добавляет к ним новые, а также ощутимо улучшает способность удерживать материал в памяти.
\end{itemize}
\textbf{Обобщение:}
\begin{itemize}
\item Сформировать порцию информации --- значит интегрировать некое понятие в четко оформленный нейронный паттерн.
\item Формирование порций помогает увеличить объем доступной рабочей памяти.
\item Создание библиотеки сформированных порций информации, содержащих необходимые концепции и способы решения, помогает развитию интуиции при решении задач.
\item При создании библиотеки порций важно сознательно фокусироваться на самых сложных понятих и аспектах решения задач.
\item Временами, как бы вы ни старались овладеть материалом, вас постигает неудача. В таких случаях не забывайте "закон озарений": если вы эффективно поработаете над материалом и создадите хорошую библиотек решений, то увидите, что удача все чаще будет на вашей стороне. Иными словами, если вы не стараетесь освоить материал, вам гарантирована неудача, однако те, кто целенаправленно прилагает усилия, достигнут гораздо больших успехов.
\end{itemize}

\section{Способы, советы и хитрости}
\begin{itemize}
\item[-] Нежелание начинать работу --- совершенно естественное чувство. Вопрос в том, как вы с ним справляетесь.
\item[-] Зараннее определять для себя время окончания работы не менее важно, чем планировать рабочий период.
\item[-] Разбивать работу на дневные порции --- важная часть процесса.
\item[-] Список приложений, которые могут помочь: страница 127.
\end{itemize}
\textbf{Обобщение:}
\begin{itemize}
\item Ментальные трюки могут быть действенным средством. Вот некоторые из наиболее эффективных:
\begin{itemize}
\item Что бы избежать прокрастинации, устройтесь в месте, где вас не будут беспокоить.
\item Практикуйтесь в игнорировании отвлекающих мыслей: просто не обращайте на них внимания, пусть они вас не затрагивают.
\item Если у вас пропал настрой на работу, измените фокус внимания так, что бы переключиться с негативных мыслей на позитивные.
\item Не забывайте: нежелание начинать работу --- совершенно нормальное чувство.
\end{itemize}
\item Планирование развлечений --- один из самых дественных способов борьбы с прокрастинацией и один из важнейших стимулов избегать прокрастинации.
\item Основа борьбы с прокрастинацией --- составление осмысленного ежедневного списка дел с еженедельной проверкой того, правильно ли вы продвигаетесь вперед.
\item Составляйте список дел на день накануне вечером.
\item Лягушек съедайте сразу.
\end{itemize}

\section{Зомби-прокрастинации, окончание}
\textbf{Обобщение:}
\begin{itemize}
\item Ведите ежедневник-планировщик, чтобы было легче отслеживать выполнение поставленных целей и определять, какие методы работают, а какие нет.
\item Определите, какому роду занятий вы посвящаете каждый день недели, и следуйте этой схеме.
\item Составляйте список дел, намеченных на тот или иной день, накануне вечером --- так у мозга будет больше возможности их осмыслить и обеспечить успех.
\item Разделите работу на отдельные небольшие порции; не забывайте о щедрых наградах для себя (и своих зомби!); уделяйте несколько минут тому, чтобы насладиться радостью от успеха.
\item Вознагарждайте себя только после достижения результата.
\item Отслеживайте сигналы, обычно ведущие к прокрастинации, и не поддавайтесь им.
\item Смените обстановку, чтобы сигналов было меньше, --- например, занимайтесь в тихом месте библиотеки.
\item Препятствия всегда будут, но не привыкайте винить во всем внешние факторы. Если вы во всем вините других --- пора взглянуть в зеркало.
\item Научитесь доверять новой системе занятий. В сфокусированном режиме работайте интенсивно, но не забывайте и о рассеянном состоянии: научитесь расслабляться и не испытывать при этом чувство вины.
\item Имейте запасной план на случай прокрастинации. Никто из нас не безргрешен.
\item Лягушек съедайте сразу.
\end{itemize}

\section{Совершенствование памяти}
\textbf{Обобщение:}
\begin{itemize}
\item Метод "дворца памяти" (условное размещение запоминаемых объектов в знакомом месте) позволяет вам подключать к делу мощный ресурс зрительной памяти.
\item Когда вы учитесь использовать свою память более упорядоченным, но все же более творческим образом, это помогает вам научиться фокусировать внимание --- в том числе и тогда, когда вы создаете обширные ассоциативные связи, способствующие прочному запоминанию.
\item Если вы хорошо понимаете запоминаемый материал, вы можете несравнимо глубже его усвоить. Кроме того, вы пополняете и укрепляете свою мыслительную "библиотеку", необходимую для полного овладения материалом.
\end{itemize}

\section{Еще несколько советов по запоминанию}
\textbf{Обобщение:}
\begin{itemize}
\item Метафоры помогают быстрее запоминать сложные идеи.
\item Повторение --- важный этап запоминания, не позволяющий усвоенной информации выветриться из памяти.
\item Осмысленное объединение фактов в группы и создание аббревиатур упрощает понимание и помогает формированию порций информации, из-за чего материал в целом легче укладывается в памяти.
\item Рассказы и истории --- даже если они придуманы только как забавные мнемонические приемы - позволяют легче удержать в памяти изучаемый материал.
\item Писать от руки и произносить вслух то, что вы изучаете, облегчает процесс сохранения материала в памяти.
\item Физические упражнения --- мощное средство, позволяющее нейронам расти и создавать новые связи.
\end{itemize}

\section{Учимся ценить свой талант}
\begin{itemize}
\item[-] Мы учимся с помощью попыток сопоставить информацию, которую получаем. Сложные понятия врят ли можно усвоить, просто слушая чей-то рассказ.
\item[-] Именно практика --- особенно целенаправленная работа с самыми сложными аспектами материала --- может возвысить средние способности мозга до уровня интеллектуальных возможностей людей, обладающих природным даром.
\end{itemize}
\textbf{Обобщение:}
\begin{itemize}
\item После того как порции информации сформированы, наступает момент, когда вы перестаете осознанно отслеживать каждую мелкую деталь и начинаете действовать автоматически.
\item Порой кажется унизительным заниматься вместе со студентами, которые схватывают материал быстрее, чем вы. Однако "средние" студенты подчас обладают преимуществами в плане инициативы, трудолюбия и креативности.
\item Ключ к творческому подходу частично зависит от способности переключаться с полной концентрации на расслабленный и мечтательный рассеянный режим.
\item Слишком сильная рассредоточенность может помешать поискам верного решения --- как попытка забить шуруп молотком таким образом, словно вы имеете дело с гвоздем. Когда вы зашли в тупик, порой лучше отвлечься от задачи ненадолго заняться чем-то другим или просто лечь спать.
\end{itemize}

\section{Формируем мозг}
\begin{itemize}
\item[-] Мы можем существенно менять мозг, меняя способ мышления.
\item[-] Хорошие порции информации создают нейронные паттерны, которые могут действовать в рамках не только изучаемого предмета, но и других дисциплин и сфер жизни. Абстрактное обобщение помогает переносить идеи из одной сферы в другую.
\end{itemize}
\textbf{Обобщение:}
\begin{itemize}
\item Мозг у разных людей становится зрелым не в одном и том же возрасте. Многие достигают зрелости только к 25 годам.
\item Многие из особо выдающихся ученых начинали как безнаденжные сорвиголовы.
\item Успешные профессионалы в естественных науках, математике и технике постепенно учатся формировать порции информации, что бы абстрагировать ключевые идеи.
\item Метафоры и аналогии --- средство формирования порций информации, которые позволяют сведениям из разных областей взаимодействовать друг с другом.
\item Независимо от вашей профессии --- нынешней или той, которой вы только овладеваете, --- не замыкайтесь в одной области и обязательно изучайте математику и естественные науки. Это даст вам богатый запас порций информации, благодаря которому вы станете лучше справляться с любыми жизненными и профессональными трудностями.
\end{itemize}

\section{Развитие внутреннего зрения через уравнения-стихи}
\textbf{Обобщения:}
\begin{itemize}
\item Уравнения --- это всего-лишь способ абстрагировать и упростить понятия. Это значит, что уравнения обладают глубинным уровнем смысла, схожим с глубинным смыслом, присутствующим в поэзии.
\item Ваш "внутренний взор" важен, поскольку он помогает разыгрывать сценки и одушевлять то, что вы изучаете.
\item Перенос --- это способность брать изучаемое и применять его в других контекстах.
\item Понимание сути математических принципов --- важная часть обучения, поскольку оно облегчает перенос и применение конкретного принципа в различные сферы.
\item Заниматься посторонними вещами во время занятий --- значит усваивать знания недостаточно глубоко, а это отрицательно влияет на способность к переносу изучаемого материала.
\end{itemize}

\section{Мы в ответе за собственные знания}
\begin{itemize}
\item[-] Гордитесь тем, кто вы есть, особенно теми качествами, которые отличают вас от толпы, и считайте их своим талисманом на пути к успеху.
\end{itemize}
\textbf{Обобщение:}
\begin{itemize}
\item Самообразование --- один из самых полезных и высокоэффективных способов обучения, поскольку оно усиливает способность мыслить независимо и помогает отвечать на нестандартные вопросы, которые преподаватели порой задают на экзаменах.
\begin{itemize}
\item Упорство при изучении материала часто более важно, чем ум.
\item Научитесь заговаривать с людьми, которыми вы восхищаетесь. Тем самым вы можете обрести мудрых наставников, которые одной простой фразой способны изменить ход вашей жизни. Однако не отнимайте у них время.
\end{itemize}
\item Если вам не даются основы изучаемого предмета --- не отчаивайтесь. Медленно соображающие студенты, к общему удивлению, часто усваивают фундаментальные знания лучше, чем их более понятливые сокурсники. Когда вы наконец покорите текущий уровень знаний, вам откроется путь к более глубинным уровням.
\item Люди склонны не только сотрудничать но и конкурировать. Всегда найдутся те, кто будет вас критиковать или принижать ваши успехи. Научитесь относиться к таким ситуациям бесстрастно.
\end{itemize}

\section{Как избежать излишней уверенности}
\textbf{Обобщение:}
\begin{itemize}
\item Сфокусированный режим не гарантирует отсутствие ошибок, хоть вы и будете уверенны, что сделали все верно. Перепроверка даст более широкий взгляд на сделанное, поскольку она задействует другие нейронные процессы, позволяющие замечать недочеты.
\item Работа с людьми, не боящимися выражать несогласие, может:
\begin{itemize}
\item Помочь увидеть ошибки в своих рассуждениях.
\item Научить оперативно мыслить и реагировать в стрессовых ситуациях.
\item Улучшить процесс получения знаний, поскольку объясняя материал другим, вы сможете сами лучше понять его и закрепить собственные знания.
\item Познакомить вас с людьми, важными для будущей карьеры, и направить на путь к лучшим возможностям.
\end{itemize}
\item Критика во время учебы --- ваша или в ваш адрес --- не должна восприниматься как личные нападки. Критика --- способ разобраться и понять.
\end{itemize}

\section{Экзамены и тесты}
\begin{itemize}
\item[-] Тестирование --- необычайно мощный опыт обучения.
\item[-] Приступайте к решению задач, начиная с самой сложной, однако твердо пообещайте себе переключиться с нее на что-то другое, если на первой-второй минуте дело застопорится или вы заподозрите, что идете по неверному пути.
\end{itemize}
\textbf{Обобщение:}
\begin{itemize}
\item Бессонная ночь перед экзаменом может свести на нет любые усилия по подготовке.
\item Экзамен --- серьезное испытание. Не забывайте, что, так же как врачам и боевым летчикам нужно следовать перечню необходимых действий, вам полезно будет сверяться со списком контрольных вопросов для подготовки к экзаменам и тестам. Это хороший способ повысить свои шансы на успех.
\item В состоянии стресса организм выделяет определенные химические вещества, и многое зависит от того, как вы воспринимаете реакцию организма. Если с мысли "Я боюсь экзамена" перестроиться на мысль "Экзамен --- это так интересно, я хочу показать себя с лучшей стороны!", то вероятность успеха возрастет.
\item Если на экзаменах вы впадаете в панику, временно переключите внимание на дыхание. Расслабьте живот, положите на него ладонь и медленно сделайте глубокий вдох. Рука должна податься вперед, грудная клетка расшириться.
\item Сознание порой может внушать вам иллюзию, будто все задания выполнены верно, хотя на самом деле это не так. При любом удобном случае моргните, переключите внимание, а затем перепроверьте ответы с другой точки зрения, спрашивая себя "Нет ли здесь бессмыслицы?".
\end{itemize}

\section{Раскройте свой потенциал}
\begin{itemize}
\item[-] Желание немедленно что-то выяснить мешает самому процессу выяснения.
\item[-] Перестроить мозг в вашей власти! Главное при этом --- терпеливо и настойчиво, со знанием дела работать над сильными и слабыми сторонами мозга. 
\end{itemize}
\textbf{10 способов преуспеть}
\begin{itemize}
\item Стараться вспоминать.
\item Проверять себя.
\item При решении задач создавать порции информации.
\item Следовать правильному режиму повторения материала.
\item Применять разные подходы, практикуясь в решении задач.
\item Делать перерывы.
\item Объяснять материал воображаемому собеседнику и пользоваться простыми аналогиями.
\item Сосредотачиваться.
\item Сначала съедать лягушек.
\item Помнить о своей мечте.
\end{itemize}
\textbf{10 шагов к провалу}
\begin{itemize}
\item Пассивно перечитывать, т.е. просто сидеть и скользить взглядом по тексту.
\item Делать слишком много выделений в тексте.
\item Заглядывать в раздел ответов и, узнав способ решения задачи, считать, будто теперь вы знаете, как ее решить.
\item Начинать готовиться к тестированию в последний момент.
\item Раз за разом решать однотипные задачи, способ решениях которых вы уже знаете.
\item Превращать совместные занятия с друзьями в посиделки.
\item Игнорировать необходимость прочитать нужный раздел учебника, прежде чем приступить к решению задачи.
\item Пренебрегать возможностью консультироваться с преподавателями и сокурсниками в сложных случаях.
\item Считать, будто можно надежно выучить материал, если постоянно отвлекаться.
\item Мало спать.
\end{itemize}
\end{document}