\documentclass{article}

\usepackage[T2A]{fontenc}
\usepackage[utf8]{inputenc}
\usepackage[russian,english]{babel}
\usepackage[left=2cm,right=2cm,top=2cm,bottom=2cm,bindingoffset=0cm]{geometry}

\usepackage{sectsty}
\allsectionsfont{\centering}

\renewcommand{\thesection}{\Roman{section}} 

\parindent=0cm
\begin{document}
\title{Георгий Челпанов - Учебник логики}
\author{Tass}
\date{2016/09/01}
\maketitle

\newpage
\tableofcontents
\newpage

\section{Определение и задачи логики}
\textbf{Логика} --- наука о законах правильного мышления.
\\
\textbf{Психология и логика}: Психология изучает, как происходит процесс мышления. Логика рассматривает условия, при которых мысль может быть правильной.
\\
\textbf{Непосредственно очевидное положение} --- положение, которое является результатом чувственного восприятия.
\\
\textbf{Посредственно очевидное положение} --- посредственное знание доказывается при помощи непосредственных знаний.
\\
\textbf{Доказательство} --- сведение неочевидных положений к положениям непосредственно очевидным.
\\
\textbf{Задача логики} --- показать, каким правилам должно следовать умозаключение, чтобы быть верным.
\\ 
\textbf{Может ли "здравый смысл" заменить логику?} Найти ошибку, охарактеризовать, сказать почему умозаключение нужно считать ошибочным зачастую можно только с помощью знания правил логики.
\\
\textbf{Основные направления в логике}: Формальная логика изучает те отделы логики, в которых может быть применяем формальный критерий истинности. Индуктивная логика разрабатывает те отделы, в которых применяется материальный критерий истинности.

\section{О различных классах понятий}
\textbf{Термины и понятия} --- Можно рассматривать или понятия в том виде, как они нами мыслятся, или их выражения при помощи слов. На самом деле, между двумя этими рассмотрениями нет существенной разницы. Каждое понятие в мышлении фиксируется, приобретает устойчивость, определённость благодаря тому или иному термину. Слово является заместителем понятий.
\\
Тем не менее, иногда с одним термином могут быть связаны несколько понятий.
\\
\textbf{Общие понятия} --- относятся к группе или классу предметов или явлений, имеющих известное сходство между собой.
\\
\textbf{Индивидуальные понятия} --- понятия, которые относятся к предметам единичным, индивидуальным. Так же --- имена собственные.
\\
Употребление понятия в \textbf{собирательном смысле} подразумевает одно целое, группу, состоящую из однородных единиц. В собирательном смысле могут употребляться как индивидуальные, так и общие понятия. Например, \textit{"парламент издал закон о всеобщей воинской повинности"}.
\\
Употребление понятия в \textbf{разделительном смысле} подразумевает, что утверждения будут справедливы относительно каждой отдельной единицы, входящей в ту или иную группу предметов. Например, \textit{"все рабочие утомились"}.
\\
\textbf{Собирательные и общие термины}. Собирательные понятия служат для обозначения целого, состоящего из однородных единиц. Тот же самый термин становится общим, когда мы мыслим их отдельными представителями известного класса.
\\
\textbf{Абстрактный термин} обозначает качества, состояния, свойства, действия, которые рассматриваются сами по себе, без вещей. Например \textit{"тяжесть", "объем", "твёрдость", "гуманность"}. Абстрактными, в отличном от этого смысле, так же могут называться понятия таких вещей, которые не могут восприниматься нами как известная, определённая вещь, например, \textit{"вселенная", "человечество"}.
\\
\textbf{Конкретный термин} --- понятие вещей, предметов, лиц, факторов, событий, состояний сознания, если мы рассматриваем их имеющими определённое существование, например, \textit{"квадрат", "пламя", "дом"}. Прилагательные всегда являются терминами конкретными; употребляя прилагательное "белый", мы всегда мыслим вещь.
\\
\textbf{Положительные термины} характеризуются тем, что они служат для обозначения наличности того или иного качества.
\\
\textbf{Отрицательный термин} указывает на отсутствие того или иного качества.
\\
\textbf{Абсолютный термин} --- термин, который в своём значении не содержит никакого отношения к чему-либо другому, не принуждает мыслить о каких-либо вещах, кроме тех, которые он обозначает.
\\
\textbf{Относительный термин} --- термин, который кроме того предмета, который он обозначает, предполагает существование также и другого предмета. Например, термин \textit{"родители"} необходимо предполагает существование детей.
\end{document}