\documentclass{article}

\usepackage[T2A]{fontenc}
\usepackage[utf8]{inputenc}
\usepackage[english,russian]{babel}
\usepackage[left=2cm,right=2cm,top=2cm,bottom=2cm,bindingoffset=0cm]{geometry}

\usepackage{sectsty}
\allsectionsfont{\centering}

\renewcommand{\thesection}{\Roman{section}} 

\parindent=0cm
\begin{document}
\title{Георгий Челпанов - Учебник логики}
\author{Tass}
\date{2016/09/01}
\maketitle

\newpage
\tableofcontents
\newpage

\section{Определение и задачи логики}
\textbf{Логика} --- наука о законах правильного мышления.
\\
\textbf{Психология и логика}: Психология изучает, как происходит процесс мышления. Логика рассматривает условия, при которых мысль может быть правильной.
\\
\textbf{Непосредственно очевидное положение} --- положение, которое является результатом чувственного восприятия.
\\
\textbf{Посредственно очевидное положение} --- посредственное знание доказывается при помощи непосредственных знаний.
\\
\textbf{Доказательство} --- сведение неочевидных положений к положениям непосредственно очевидным.
\\
\textbf{Задача логики} --- показать, каким правилам должно следовать умозаключение, чтобы быть верным.
\\ 
\textbf{Может ли "здравый смысл" заменить логику?} Найти ошибку, охарактеризовать, сказать почему умозаключение нужно считать ошибочным зачастую можно только с помощью знания правил логики.
\\
\textbf{Основные направления в логике}: Формальная логика изучает те отделы логики, в которых может быть применяем формальный критерий истинности. Индуктивная логика разрабатывает те отделы, в которых применяется материальный критерий истинности.

\section{О различных классах понятий}
\textbf{Термины и понятия} --- Можно рассматривать или понятия в том виде, как они нами мыслятся, или их выражения при помощи слов. На самом деле, между двумя этими рассмотрениями нет существенной разницы. Каждое понятие в мышлении фиксируется, приобретает устойчивость, определённость благодаря тому или иному термину. Слово является заместителем понятий.
\\
Тем не менее, иногда с одним термином могут быть связаны несколько понятий.
\\
\textbf{Общие понятия} --- относятся к группе или классу предметов или явлений, имеющих известное сходство между собой.
\\
\textbf{Индивидуальные понятия} --- понятия, которые относятся к предметам единичным, индивидуальным. Так же --- имена собственные.
\\
Употребление понятия в \textbf{собирательном смысле} подразумевает одно целое, группу, состоящую из однородных единиц. В собирательном смысле могут употребляться как индивидуальные, так и общие понятия. Например, \textit{"парламент издал закон о всеобщей воинской повинности"}.
\\
Употребление понятия в \textbf{разделительном смысле} подразумевает, что утверждения будут справедливы относительно каждой отдельной единицы, входящей в ту или иную группу предметов. Например, \textit{"все рабочие утомились"}.
\\
\textbf{Собирательные и общие термины}. Собирательные понятия служат для обозначения целого, состоящего из однородных единиц. Тот же самый термин становится общим, когда мы мыслим их отдельными представителями известного класса.
\\
\textbf{Абстрактный термин} обозначает качества, состояния, свойства, действия, которые рассматриваются сами по себе, без вещей. Например \textit{"тяжесть", "объем", "твёрдость", "гуманность"}. Абстрактными, в отличном от этого смысле, так же могут называться понятия таких вещей, которые не могут восприниматься нами как известная, определённая вещь, например, \textit{"вселенная", "человечество"}.
\\
\textbf{Конкретный термин} --- понятие вещей, предметов, лиц, факторов, событий, состояний сознания, если мы рассматриваем их имеющими определённое существование, например, \textit{"квадрат", "пламя", "дом"}. Прилагательные всегда являются терминами конкретными; употребляя прилагательное "белый", мы всегда мыслим вещь.
\\
\textbf{Положительные термины} характеризуются тем, что они служат для обозначения наличности того или иного качества.
\\
\textbf{Отрицательный термин} указывает на отсутствие того или иного качества.
\\
\textbf{Абсолютный термин} --- термин, который в своём значении не содержит никакого отношения к чему-либо другому, не принуждает мыслить о каких-либо вещах, кроме тех, которые он обозначает.
\\
\textbf{Относительный термин} --- термин, который кроме того предмета, который он обозначает, предполагает существование также и другого предмета. Например, термин \textit{"родители"} необходимо предполагает существование детей.

\section{Содержание и объём понятий}
\textbf{Признаки понятий} --- составные элементы представления или понятия. Признаки есть то, чем одно представление или понятие отличается от другого. Например, признаками золота мы считаем "металл", "драгоценный", "имеющий определенный удельный вес", etc.
\\
\textbf{Существенные и второстепенные признаки}. Существенные или основные признаки --- признаки, без которых мы не можем мыслить известного понятия и которые излагают природу предмета. Остальные признаки --- второстепенные. Например, для ромба существенным является тот признак, что он есть четырёхугольник с параллельными и равными сторонами. Несущественным для ромба является тот признак, что он имеет ту или иную величину сторон/углов.
\\
\textbf{Какие признаки понятий можно отличить?} Со времён Аристотеля, признаки понятий принято делить на 5 классов: родовой признак, видовое различие, вид, собственный признак, несобственный признак.
\\
Род (genus) или \textbf{родовой признак} есть понятие класса, в который мы вводим другое рассматриваемое понятие. Например, \textit{наука} есть родовой признак для понятия \textit{химия}.
\\
\textbf{Видовое различие} (differentia specifica) --- признак, предназначенный для выделения понятия из ряда ему подобных понятий. Например, \textit{"моряк русский"}, \textit{"моряк английский"}. В этом случае, \textit{"русский"}, \textit{"английский"} --- видовое различие.
\\
\textbf{Вид} --- объединение видового различия и родового признака. Например, \textit{"Здание для склада оружия" --- арсенал}. В этом случае \textit{"здание"} есть род, \textit{"для хранения оружия"} есть видовое различие.
\\
\textbf{Собственный признак} (proprium) --- признак, который присущ всем вещам данного класса, который не содержится в числе существенных признаков, но который может быть выведен из них. Например, существенный признак треугольника --- это прямолинейная плоская фигура с тремя сторонами. Что же касается того признака треугольника, что сумма углов его равняется двум прямым, то это есть его собственный признак, что вытекает или выводится из основных признаков.
\\
\textbf{Несобственный признак} (accidents) --- признак, который не может быть выведен из существенного признака, хотя и может быть присущ всем вещам данного класса. Например, чёрный цвет ворона есть accidents. 
\\
\textbf{Отделимые и неотделимые несобственные признаки}. Неотделимые несобственные признаки (accidents inseparabile) присущи всем вещам данного класса, например, чёрный цвет ворона. Отделимые несобственные признаки (accidents separabile) присущи только некоторым вещам того или иного класса, например, чёрный цвет волос человека.
\\
\textbf{Содержание понятия} --- то, что мыслится в понятии, сумма признаков понятия. Поэтому, каждое понятие можно разложить на ряд присущих ему признаков. Например, в понятии "сахар" мыслятся признаки "сладкий", "белый", "шероховатый".
\\
\textbf{Объём понятия} --- то, что мыслится посредством понятия, сумма тех классов, родов, видов, etc., к которым данное понятие может быть приложено. Пример объема понятия "животное": птица, рыба, насекомое, человек. Понятие с большим объёмом называется \textit{родом} по отношению к тому понятию с меньшим объёмом, которое входит в его объём. Понятие с меньшим объёмом в таком случае называется \textit{видом}.
\\
\textbf{Summum genus} --- высший род, который уже не может быть видом для другого рода.
\\
\textbf{Infima species} --- низший вид, в объём которого не входят понятия с меньшим объёмом, а только отдельные индивидумы.
\\
\textbf{Proximum genus} --- ближайший высший род того или иного вида.
\\
\textbf{Обобщение} (generalisatio) --- процесс преобразования более общего понятия из менее общего, при котором некоторое количество признаков от данного понятия отнимается. Образует род из видов.
\\
\textbf{Ограничение} (determinatio) --- процесс образования менее общих понятий из более общих путем прибавления признаков к более общему понятию, благодаря чему понятие уясняется. Например, что бы из понятия "дерево" получить понятие "пальма", надо к признакам дерева прибавить специальные признаки пальмы.
\\
\textbf{Отношение между объёмом и содержанием понятия}. Объём понятия "человек" обширнее, чем, например, объём понятия "негр". Но содержание понятия "негр" будет обширнее содержания понятия "человек", т.к. к признакам понятия "человек" добавляютcя дополнительные признаки. Т.е. по мере увеличения содержания понятия, уменьшается его объём. И наоборот.

\section{Логические категории и отношения между понятиями}
\textbf{Категория} --- понятие, которое служит для обозначения самых общих сходств между предметами.
\\
\textbf{Категории по Аристотелю}:
\begin{itemize}
\item Субстанция (substantia)
\item Количество (quantitas)
\item Качество (qualitas)
\item Отношение (relatio)
\item Место (ubi)
\item Время (quando)
\item Положение (situs)
\item Обладание (habitus)
\item Действие (actio)
\item Страдание (passio)
\end{itemize}
\textbf{Какие категории следует признавать?} В новейшей философии в качестве наиболее общих классов мыслимого, различают вещь, свойство и отношение.
\\
\textbf{Вещь} --- то, что обладает большим или меньшим постоянством формы.
\\
\textbf{Свойство} --- качество, действие, состояние вещи.
\\
\textbf{Отношение} --- отношение одной вещи к другой вещи. Например, одна вещь может быть больше, чем другая.
\\
\textbf{Подчинение понятий} (subordinatio notionum): Одно понятие относится к другому, как вид к своему роду; одно понятие входит в объём другого как часть его объёма. Например, \textit{"дерево"} и \textit{"берёза"}.
\\
\textbf{Соподчинение понятий} (coordinatio notionum): Два или более низших понятия, которые входят в объём одного и того же более широкого понятия. Например, \textit{"мужество"} и \textit{"умеренность"} входят в объём понятия \textit{"добродетель"}.
\\
\textbf{Равнозначащие понятия} (notiones aequipollenetes): Два понятия с различным содержанием, но одинаковым объёмом. Например, \textit{"христианин"} и \textit{"крещённый"}.
\\
\textbf{Противные или противоположные понятия} (notiones contrariae): Понятия, входящие в один и тот же объём, но очень отличающиеся друг от друга. Например, \textit{"чёрный"} и \textit{"белый"}.
\\
\textbf{Противоречащие понятия} (notiones contradictoriae): Два понятия, относительно которых известно, что понятие B не есть A. Напрмер, \textit{"белый"} и \textit{"небелый"}
\\
\textbf{Скрещивающиеся понятия} (notiones inter se convenientes): Два понятия, содержание которых различно, но объёмы некоторыми своими частями совпадают. Например, \textit{"писатели"} и \textit{"учёные"}.
\\
\textbf{Несравнимые понятия} (notiones disparatae): Понятия, не имеющие общего ближайшего родового понятия, в объём которого они бы могли войти как координированные. Например, \textit{"душа"} и \textit{"треугольник"}.
\\
\textbf{Что необходимо для того, что бы понятия можно было сравнивать?} Ближайшее общее понятие (tertium comparationis), в объём которого бы входили сравниваемые понятия.

\section{Об определении}
\textbf{Сложные и простые понятия}. Понятия, имеющие много признаков, которые могут быть определены, имеют сложное содержание. Понятия, содержание которых не может быть раскрыто, называются простыми. Например \textit{пунцовый цвет}, \textit{тяжесть}, \textit{равенство}.
\\
\textbf{Понятия, которые невозможно определить}: Определяемыми не могут быть простые понятия. Так же нельзя определить индивидуальные понятия, поскольку они имеют бесконечное множество признаков.
\\
\textbf{Определение} --- указание ближайшего рода понятия с присоединением его видового различия.
\\
\textbf{Условия правильности определения}: 
\begin{itemize}
\item Определение должно быть \textit{соразмерным}, т.е. таким, в которым объемы определяемого и определяющего тождественны. Если правило это нарушено, то определение \textit{неадекватно} или \textit{несоразмерно}.
\item Определение не должно быть рекурсивным.
\item Определение не должно быть отрицательным.
\item Определение должно быть ясным, т.е. нельзя пользоваться выражениями двусмысленными и малопонятными.
\end{itemize}
\textbf{Какие определения называются слишком широкими?}. В широком определении объем определяющего превосходит объем определяемого, например \textit{лошадь --- домашнее животное}. 
\\
\textbf{Какие определения называются слишком узкими?}. В узком определении объем определяющего меньше объема определяемого, например \textit{треугольник --- плоская прямолинейная фигура с тремя равными сторонами}.
\\
\textbf{Рекурсивное определение} --- определяемое понятие определяется понятием, которое делается понятным только посредством определяемого. Например \textit{"вращение есть движение вокруг своей оси"} является рекурсивным, т.к. ось --- прямая, вокруг которой происходит вращение. Так же, рекурсивными являются определния, где определяющее и определяемое не являются разными и самостоятельными понятиями, например \textit{свет есть то, чему присущ свет}. Такая ошибка называется \textit{тавтологией}.
\\
\textbf{Почему признаки в определении не должны иметь отрицательного характера?} Отрицательные признаки чужды понятию, кроме того, их можно указать множество; они не раскрывают содержание понятия, оставляют его неопределенным. Но, отрицательные определения могут быть употребляемы тогда, когда когда определяемое понятие имеет отрицательный характер. Например \textit{чужестранец --- человек не принадлежащий к данной стране}.
\\
\textbf{Приемы, заменяющие определения.}
\begin{itemize}
\item Указание --- приведение кого-либо в соприкосновение с предметами непосредственного восприятия.
\item Описание --- приведение возможно точно и полно признаков предмета. Употребляется при ознакомлении с индвидуальными предметами или при ознакомлением со свойствами какой-либо вещи.
\item Характеристика --- приведение выдающихся признаков какого-либо предмета или явления.
\item Сравнения --- знакомство с предметом, путём его сравнения с другими похожими понятиями; уяснение понятия при помощи иного, более ясного понятия.
\item Различие --- указание на различие между данным поняием и другим.
\end{itemize}

\section{О делении}
\textbf{Задача деления} заключается в том, чтобы указать все виды, совокупность которых составляет объем данного понятия.
\\
\textbf{Что называется делимым понятием?} То понятие, объем которого мы раскрываем (totum dividentum).
\\
\textbf{Что называется членами деления?} Виды, получаемые от деления (membra divisionis).
\\
\textbf{Основание деления} --- признак, дающий нам возможность разделить род на виды (fundamentum divisionis). При делении понятия "треугольник", основанием может выступать, к примеру, величина углов.
\\
\textbf{Подразделение} --- процесс деления на подвиды видов, полученных в результате деления понятия.
\\
\textbf{Дихотомия} --- деление понятия A на противоречащие понятия B и не-B.
\\
\textbf{Недостатки дихотомии}. Недостатком дихотомии является то, что часть объёма делимого понятия, которая обозначается частицей "не", остается крайне неопределенной. 
\\
\textbf{Преимущества дихотомии}. Дихотомия облегчает процесс деления, потому что предает ему исчерпывающий характер. 
\\
\textbf{Правила деления}
\begin{itemize}
\item Деление должно быть адекватно или соразмерно. Необходимо перечислить все виды, не уменьшая и не увеличивая их количества.
\item Члены деления должны исключать друг друга. 
\item Деление должно иметь одно основание. При подразделении, основание должно измениться.
\item Деление должно быть непрерывным, т.е. при делении необходимо переходить к ближайшему низшему роду, дабы не получить скачок в делении.
\end{itemize}
\end{document}