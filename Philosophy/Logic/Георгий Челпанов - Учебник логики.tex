\documentclass{article}

\usepackage[T2A]{fontenc}
\usepackage[utf8]{inputenc}
\usepackage[russian,english]{babel}
\usepackage[left=2cm,right=2cm,top=2cm,bottom=2cm,bindingoffset=0cm]{geometry}

\usepackage{sectsty}
\allsectionsfont{\centering}

\renewcommand{\thesection}{\Roman{section}} 

\parindent=0cm
\begin{document}
\title{Георгий Челпанов - Учебник логики}
\author{Tass}
\date{2016/09/01}
\maketitle

\newpage
\tableofcontents
\newpage

\section{Определение и задачи логики}
\textbf{Логика} --- наука о законах правильного мышления.
\\
\textbf{Психология и логика}: Психология изучает, как происходит процесс мышления. Логика рассматривает условия, при которых мысль может быть правильной.
\\
\textbf{Непосредственно очевидное положение} --- положение, которое является результатом чувственного восприятия.
\\
\textbf{Посредственно очевидное положение} --- посредственное знание доказывается при помощи непосредственных знаний.
\\
\textbf{Доказательство} --- сведение неочевидных положений к положениям непосредственно очевидным.
\\
\textbf{Задача логики} --- показать, каким правилам должно следовать умозаключение, чтобы быть верным.
\\ 
\textbf{Может ли "здравый смысл" заменить логику?} Найти ошибку, охарактеризовать, сказать почему умозаключение нужно считать ошибочным зачастую можно только с помощью знания правил логики.
\\
\textbf{Основные направления в логике}: Формальная логика изучает те отделы логики, в которых может быть применяем формальный критерий истинности. Индуктивная логика разрабатывает те отделы, в которых применяется материальный критерий истинности.
\end{document}