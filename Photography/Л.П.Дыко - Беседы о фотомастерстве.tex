\documentclass{article}

\usepackage[T2A]{fontenc}
\usepackage[utf8]{inputenc}
\usepackage[russian,english]{babel}
\usepackage[left=2cm,right=2cm,top=2cm,bottom=2cm,bindingoffset=0cm]{geometry}

\usepackage{sectsty}
\allsectionsfont{\centering}

\parindent=0cm
\begin{document}
\title{Л.П.Дыко - Беседы о фотомастерстве}
\author{Tass}
\date{2016/07/25}
\maketitle

\tableofcontents
\newpage
\section{Путь в фотоискусство}
\textbf{Художественное мастерство фотографа}. Смысл и значение художественного мастерства состоят в том, чтобы найти в жизни типичные и яркие ее проявления, воплотить жизненные наблюдения в образах, использовать все возможности и средства фотографии, для полного, глубокого раскрытия содержания произведения. Зритель увидит и оценит не световые пятна и линейные сходы, как таковые, а правдивые картины жизни, типические характеры, ярко, выпукло, живописно воспроизведенные фотографом.

\section{Искусство фотографии и его изобразительные средства}
\textbf{Изобразительные средства фотогорафии} --- линейная композиция, световое решение, тональный рисунок. Так же, возможно творческое использование технических средств --- экспозиции, выдержки, глубины резкости.

\section{Изобразительное решение темы}
\textbf{Начало формирования кадра} --- сознательный отбор жизненного материала, введение в кадр только тех деталей, которые органически связаны с выбранным сюжетом. Без ясного авторского замысла, без четкого сюжета не может быть найдено законченное изобразительное решение снимка.
\\
\textbf{Изобразительное решение темы} начинается с определения смыслового центра снимаемого сюжета. Это не просто нахождение внешнего рисунка кадра --- линейного ритма, градации тонов, характера светотени, etc., это --- осмысление материала, умение понять его и выразить в конкретном моменте развивающегося действия. Отобрав из всего материала то, что яснее всего раскрывает смысл, содержание картины, фотограф должен выделить эти элементы и изобразительно нарисовать их отчетливо, выпукло, объемно, то есть смысловой центр снимка сделать зрительным центром.

\section{Смысловой и изобразительный центр кадра}
\textbf{Отчетливому изображению главного объекта} будут способствовать:
\begin{itemize}
\item Укрупнение, изображение сюжетного центра в крупном масштабе: съемка ведется с близкого расстояния, а рамка кадра очерчивает относительно небольшое пространство.
\item Размещение главного объекта на переднем плане. В кадр могут попасть и достаточно широкие пространства, снимок может представлять собой общий план, и только главный объект находится на близком расстоянии от точки съемки.
\item Тональное различие объекта и фона. Контраст тонов помогает выявлению главного объекта изображения.
\item Световой акцент, при котором самые высокие яркости образуются на главном элементе композиции.
\item Наводка на резкость по главному объекту изображения и потеря резкости на фоне и второстепенных элементах композиции.
\item Направление основных линий в кадре, ведущих глаз зрителя к центру композиции.
\item Размещение главного объекта изображения в центре картинной плоскости или близко к нему.
\item Создание контрастного тонального рисунка на главном и мягкая градация тонов на второстепенных композиционных элементах и на фоне.
\end{itemize}
\end{document}