\documentclass{article}
% Кодировка, поддержка русского языка
\usepackage[T2A]{fontenc}
\usepackage[utf8]{inputenc}
\usepackage[english,russian]{babel}
% Отступы от края страницы
\usepackage{geometry}
\geometry{left=2cm}
\geometry{right=2cm}
\geometry{top=2cm}
\geometry{bottom=2cm}
\geometry{bindingoffset=0cm}
% Вложенные списки
\usepackage[ampersand]{easylist}
\ListProperties(Hide=100,Hang=true,Progressive=3.5ex,Style*=$\triangleright$ )

\newcommand{\note}[1]{\textit{#1}}
\newcommand{\important}[1]{\textbf{#1}}
\newcommand{\enquote}[1]{,,#1''}
\renewcommand{\section}[2]{
	\vspace{6em}
	\begin{flushright}
	\Large
	\baselineskip=0.5\baselineskip
	\textbf{#1}
	\\
	\rule[0.5\baselineskip]{\textwidth}{0.15pt}
	\\
	\textbf{#2}
	\end{flushright}
	}
\renewcommand{\subsection}[1]{
	\vspace{2em}
	\begin{flushright}
		\large
		\textbf{#1}
	\end{flushright}
	}
\renewcommand{\title}[2]{
	\begin{center}
		\LARGE
		\baselineskip=0.5\baselineskip
		\textbf{#1}
		\\
		\rule[0.5\baselineskip]{0.7\textwidth}{0.15pt}
		\\
		\textbf{#2}
		\\\baselineskip=2\baselineskip(конспект)		
	\end{center}
	}
\newcommand{\define}[2]{
	\textbf{#1} --- #2
	}
\begin{document}
\title{Д. Уиттакер}{Как тестируют в Google}
\section{Глава 1}{Первое знакомство с организацией тестирования в Google}
\begin{easylist}
& Тестирование неотделимо от разработки.
&& Одним тестированием качества не добиться.
& Роли в разработке.
&& Разработчик.
&&& Пишет много тестового кода.
&&& TDD и юнит-тестирование.
&&& Отвечает за фичи приложения и их качество.
&& Разработчик в тестировании.
&&& Тестируемость кода.
&&& Создание инфраструктуры тестирования.
&&& Анализ архитектуры, качества кода и рисков проекта.
&&& Заинтересован в улучшении качества и тестового покрытия.
&&& Отвечает за фичи тестирования.
&& Инженер по тестированию.
&&& Пользователи на первом месте.
&&& Организуют работу по тестированию.
&&& Управляют выполнением тестов.
&&& Интерпретируют результаты тестов.
& Организационная структура.
&& Тестирование находится в отдельном горизонтальном направлении.
&& Существует параллельно с продуктовыми направлениями.
&& Временно поступают к командам разработки.
&& Назначают руководители направления продуктивности разработки (тестирования).
& Релизы.
&& Канареечный канал --- ежедневные сборки.
&& Канал разработки --- используется в повседневной работе.
&& Тестовый канал --- для внутренних пользователей.
&& Бета-канал --- стабильные сборки, первые доступны пользователям.
& Виды тестов.
\end{easylist}
\end{document}