\documentclass{article}

\usepackage[T2A]{fontenc}
\usepackage[utf8]{inputenc}
\usepackage[english,russian]{babel}
\usepackage[left=2cm,right=2cm,top=2cm,bottom=2cm,bindingoffset=0cm]{geometry}
\usepackage{sectsty}
\allsectionsfont{\centering}

\newcommand{\important}[1]{\textbf{#1}}

\begin{document}
\title{Ю.П.Сурмин --- Теория систем и системный анализ\\(конспект книги)}
\author{Tass}
\date{2017/08/25}
\maketitle

\tableofcontents
\newpage

\section{История возникновения и становления системного подхода}
\subsection{Сущность и основные характеристики системности}
Наиболее общим понятием, которое обозначает все возможные проявления систем, является "системность", заключающая в себе два смысла. Первый составляет отождествление системности с объективным, независимым от человека свойством действительности. Другой подразумевает накопленные людьми представления о самом свойстве, т.е. гносеологическое явление, некоторые знания о системах различной природы.
\\
Гносеологическая системность проявляется в трёх аспектах:
\begin{itemize}
\item \important{Системный подход} --- некоторый методологический подход человека к действительности, представляющий собой некоторую общность принципов. Системный подход состоит в том, что любой более или менее сложный объект рассматривается в качестве относительно самостоятельной системы со своими особенностями функционирования и развития.
\item \important{Теория систем} объясняет происхождение, устройство, функционирование и развитие систем различной природы.
\item \important{Системный метод} выступает как некоторая интегральная совокупность относительно простых методов и приемов познания, а также преобразования действительности.
\end{itemize}
\subsection{Возникновение и развитие системных идей}
Системные идеи прошли несколько этапов:
\begin{itemize}
\item Первый этап завершился к началу XX ст. --- этап возникновения и развития системных идей, которые складывались в практической и познавательной деятельности.
\item Второй этап развёртывается с начала прошлого века до его середины, когда происходит теоретизация системных идей, формирование первых системных теорий, широкое распространение системности во все отрасли знания, освоение их системными идеями.
\item Третий этап характеризуется тем, что происходит превращение системности в метод научных исследований, аналитической деятельности.
\end{itemize}
\subsection{Мир в свете системных представлений}
По отношению к системному подходу, можно выделить две мировоззренческие парадигмы:
\\
\important{Первая} признает системность как объективное свойство всего сущего, как важнейшую характеристику материи.
\\
Согласно \important{второй} парадигме, системность представляет собой не свойство материи, а свойство познающего субъекта. Эта парадигма говорит о том, что мир есть такой, какой он есть, а системность представляет собой лишь способ его видения и познания.
\section{Понятие "Система"}
\subsection{Категориальный аппарат системного подхода}
\subsubsection*{Основные смысловые вариации понятия "система"}
Понятие "система" используется по отношению к самым различным предметам, явлениям и процессам.
\\
Система --- это теория; классификация; завершенный метод практической деятельности; некоторый способ мыслительной деятельности; совокупность объектов природы; некоторое явление общества; совокупность установившихся норм жизни, правил поведения.
\subsubsection*{Характеристика основных определений системы}
\important{Ограниченность} --- система отделена от окружающей среды границами.
\\
\important{Целостность} --- ее свойство целого принципиально не сводится к сумме свойств составляющих элементов.
\\
\important{Структурность} --- поведение системы обусловлено не только особенностями отдельных элементов, сколько свойствами ее структуры.
\\
\important{Взаимозависимость со средой} --- система формирует и проявляет свойства в процессе взаимодействия со средой.
\\
\important{Иерархичность} --- соподчиненность элементов в системе.
\\
\important{Множественность описаний} --- по причине сложности познание системы требует множественности ее описаний.
\subsubsection*{Дескриптивный и конструктивный подходы к определению системы}
Дескриптивный подход основывается на признании того, системность свойственна действительности и лежит в основе системного анализа. Подход заключается в том, что характер функционирования системы объясняют ее структурой, элементами.
\\
В конструктивном подходе по заданной функции конструируется соответствующая ей структура. Согласно данному подходу, система есть конечное множество функциональных элементом и отношений между ними, выделяемое из среды в соответствии с заданной целью в рамках определённого временного интервала.
\subsubsection*{Основные категории системного подхода}
Классификацию системного подхода можно представить по таким основаниям, как: базовые категории; категории системы; категории составляющих системы; категории, характеризующие свойства; категории состояний системы; категории окружения системы, категории процессов; категории отражения системы; категории, характеризующие эффекты системности; категории системного анализа.
\subsection{Системообразующие факторы}
\subsubsection*{Понятие системообразующего фактора}
Системообразующий фактор, с одной стороны, представляется объективным явлением, ибо характеризует способность материи обретать и проявлять системность. С другой стороны, он выступает средством для вычленения исследователем системы из среды, т.е. он --- инструмент проверки того, есть ли то, что определяется им, системой.
\\
Встречается мнение, что системообразующим фактором является \important{цель}, благодаря которой элементы системы объединяются и функционируют ради ее достижения. Это приемлемо для живой природы и социальной жизни. В качестве системообразующего фактора может рассматриваться и \important{время} --- будущее, настоящее или прошлое.
\\
В качестве оснований классификации системообразующий факторов, выделяют \important{активность} (активный или пассивный фактор), \important{способ проявления} (открытый или латентный), \important{положение по отношению к системе} (внешний или внутренний), \important{аспекты системы} (целевой, временной, структурный, организационный, функциональный), \important{соответствие реальности} (искусственный или естественный), \important{характер действия} (стабилизирующий или дестабилизирующий).
\\
Системообразующие факторы выполняют вполне определённые функции по отношению к системам: выступают источником возникновения систем, играют важную роль в поддержании равновесия систем, обеспечивают процесс наследования в системах.
\subsubsection*{Внешние и внутренние системообразующие факторы}
В результате действия внешних факторов, элементы образуемой системы индифферентны по отношению друг к другу. Факторы нередко бывают крайне противоположными образуемой системе, являются чуждыми для ее элементов, но могут быть внутренними и необходимыми в над-системе.
\\
Внутренние факторы образуют систему, выступающую как единство подобных элементов.
\\
Некоторая совокупность объектов всегда является системой. Необходимо только понять, в каком отношении данную совокупность можно считать системой.
\section{Типология систем}
\subsection{Проблема построения классификации систем}
\subsection*{Сущность и необходимость классификации систем}
Классификация систем представляет исключительно сложную проблему, которая ещё не разрешена в науке. Анализ существующих классификаций показывает, что многие из них отличаются эклектичностью,несущественностью и неполнотой.
\subsection*{Сущностная классификация систем}
Система характеризуется четырьмя основными параметрами: \important{субстанцией}, \important{строением}, \important{функционированием} и \important{развитием}.
\\
Каждая из четырёх составляющих может быть представлена совокупностями основополагающих параметров, соответствующих их природе. Субстанция может быть представлена природой систем, их сложностью, масштабами, детерминацией, происхождением и способом бытия. Для строения свойственны элементы, связи, организация, структура и сложность. Функционирование выражается равновесием, цель, результатом и эффективностью. Развитие --- адаптивностью, скоростью, воспроизводством, вектором и траекторией.
\subsection{Характеристика сложных систем}
\subsubsection*{Сложные системы и их специфика}
Если попытаться интерпретировать сложность системы в аспекте системности, то ее можно представить как совокупность сложностей состава и организации. Сложность состава можно описать как сумму субстратной, параметрической, динамической и генетической сложностей. Сложность организации - как совокупность многообразия связей и отношений и многообразия законов.
\end{document}