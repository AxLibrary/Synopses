\documentclass{article}
% Кодировка, поддержка русского языка
\usepackage[T2A]{fontenc}
\usepackage[utf8]{inputenc}
\usepackage[english,russian]{babel}
% Отступы от края страницы
\usepackage{geometry}
\geometry{left=4cm}
\geometry{right=4cm}
\geometry{top=2cm}
\geometry{bottom=2cm}
\geometry{bindingoffset=0cm}
% Центрирование заголовков
\usepackage{sectsty}
\allsectionsfont{\centering}
% Расстояние между элементами списка, отступ слева
\usepackage{enumitem}
\setlist{nosep,leftmargin=*}

\newcommand{\note}[1]{\textit{#1}}
\newcommand{\important}[1]{\textbf{#1}}	
\renewcommand{\section}[2]{
	\vspace{5em}
	\begin{flushright}
	\Large
	\baselineskip=0.5\baselineskip
	\textbf{#1}
	\\
	\rule[0.5\baselineskip]{\textwidth}{0.15pt}
	\\
	\textbf{#2}
	\end{flushright}
	}
\renewcommand{\subsection}[1]{
	\vspace{2em}
	\begin{flushright}
		\large
		\textbf{#1}
	\end{flushright}
	}
\renewcommand{\title}[2]{
	\begin{center}
		\LARGE
		\baselineskip=0.5\baselineskip
		\textbf{#1}
		\\
		\rule[0.5\baselineskip]{0.7\textwidth}{0.15pt}
		\\
		\textbf{#2}
		\\\baselineskip=2\baselineskip(конспект)		
	\end{center}
	}
\begin{document}
\title{Ф.П. Тарасенко}{Прикладной системный анализ}
\section{Введение.}{Как возник системный анализ}
Почему накопление и обобщение опыта решения проблем началось (и продолжается) в рамках каждой отдельной профессии?
\\
Для решения проблем могут потребоваться различные профессиональные знания. Это создаёт впечатление, что проблемы различных специальностей уникальны, что и приводит к накоплению соответствующего опыта в рамках отдельных профессий.
\\
Почему, несмотря на громадное разнообразие проблем, технология (совокупность приёмов) их решения практически одинакова в случае успеха и различается в случае неудач?
\\
Идея универсального алгоритма действий по решению проблем опирается на понятие всеобщей системности, олицетворяющей единство и общность законов мироздания.
\\
Сформулируйте основные отличия прикладного системного анализа от традиционных наук.
\\
Прикладной системный анализ нацелен на решение конкретной проблемы, а не на поиск общих закономерностей; для решения проблемы могут понадобиться знания любой из традиционных наук.
\\
Почему прикладной системный анализ можно назвать над-дисциплинарной и меж-дисциплинарной область деятельности как в теоретической, так и в практической его сфере?
\\
В теоретической сфере, прикладной системный анализ может требовать использования знаний из различных областей традиционных наук. В практической деятельности, аналитик направляет коллектив участников ситуации, являющихся специалистами в требуемой области знания.
\section{Глава 1.}{Проблема и способы ее решения}
Поясните различия между понятиями "проблемная ситуация" и "проблема"
\\
Термин \note{"проблемная ситуация"} выделяет объективный компонент, реальную ситуацию; \note{"проблема"} --- субъективное отношение к реальной ситуации субъекта.
\\
Что значит "решить проблему"?
\\
Решение проблемы объединяет в себе какие-либо действия, призванные уменьшить или совсем снять недовольство субъекта.
\subsection{Варианты решения проблем}
Все способы решения проблем можно разделить на две группы:
\begin{itemize}
	\item воздействовать на субъект с целью уменьшить его недовольство, не изменяя реальности
	\item изменить реальность так, чтобы недовольство субъекта ослабло
\end{itemize}
\subsection{Способы влияния на субъект}
Какие три способа воздействия на субъект без изменения реальности могут (при определённых условиях) привести к решению его проблемы? Каковы эти условия?
\begin{itemize}
	\item Сообщение субъекту дополнительной информации о ситуации, среди которой может оказаться информация позитивного характера. Может осуществляться в виде \note{обучения} субъекта.\\Дополнительная информация не обязательно должна правдивой, но обязательно должна быть положительной.\\Возможно сокрытие правды, либо отфильтрованная полуправда.
	\item Изменение восприятия данной реальности субъектом. Формы воздействия могут отличаться: психические, физические, химические.
	\item Прерывание взаимодействия субъекта с ситуацией. Субъект может быть повышен, направлен в отпуск, переведён в другой отдел, уволен.
\end{itemize}
\subsection{Вмешательство в реальность}
Каково основное отличие субъекта от объекта?
\\
Субъект существует в реальной физической среде и подвержен её воздействиям. В отличие от объекта, субъект не только подчинён природным закономерностям, но и наделен способностью \note{оценивать} свои взаимодействия со средой: что-то ему может нравиться, а может и не нравиться. Таким образом, все оценки имеют сугубо субъективный, индивидуальный характер.
\\
Как определить смысл оценки, выраженной неким субъектом?
\\
\note{Всякий раз, когда в вашем присутствии прозвучит любое оценочное слово} (хорошо --- плохо, полезно --- вредно и т.п.), \note{насторожитесь и задайте вопрос <<В каком смысле?>>}. Оценки не бывают объективны и если вы хотите понять истинный смысл сказанного, надо выяснить, какие критерии применяет оценивающий.
\\
Почему при вмешательстве в реальность с целью решения проблемы приходится опираться на какую-то идеологию?
\\
\note{Правильным} считается поведение, максимально согласующееся с принятой субъектом идеологией. Именно идеология и определяет, что плохо, а что хорошо. 
\\
Воспроизведите классификацию идеологий на три типа. Каково основное отличие между ними?
\\
Основное отличие между идеологиями --- это определение того, \note{какое отношение к другим субъектам считать правильным}.
\begin{itemize}
	\item Первый тип идеологии --- <<принцип приоритета меньшинства>>. Это принцип приводит к тому, чтобы осуществить вмешательство, угодное клиенту, а интересы других участников не принимаются во внимание. Жизненные примеры реализации --- диктатура, иерархическая организация, эгоизм, и т.д.
	\item Второй тип идеологии --- <<принцип приоритета группы>>. Согласно ей среди участников ситуации, кроме клиента, есть другие субъекты, не менее важные и ценные. Вмешательство должно проводиться с учётом интересов всех <<наших>>. Примеры --- расизм, национализм, фашизм, коммунизм, и т.д.
	\item Третий тип --- <<принцип приоритета каждого>>. В основе лежит два постулата: нет ни одного одинакового субъекта; все субъекты равноценны и равноправны. Правильным, моральным признается только улучшающее вмешательство.
\end{itemize}
Целью прикладного системного анализа является создание улучшающего вмешательства. Перечислите не менее трёх причин, по которым в действительности это не может получиться.
\\
???
\subsection{Четыре типа вмешательств}
Назовите четыре типа улучшающих вмешательств.
\begin{itemize}
	\item \important{ABSOLUTION} --- \note{невмешательство} в расчёте на то, что естественный ход событий приведёт к разрешению проблемы. Невмешательств обладает одним из признаков улучшающего вмешательства: при этом никому не становится хуже. Так же, необходимо, что бы события действительно вели к разрешению проблемы, а любые предлагаемые вмешательства приводили к худшим результатам. Примеры: поведения врача при невозможности исцелению пациента, действия сапёра при встрече с незнакомым взрывным устройством.
	\item \important{RESOLUTION} --- \note{частичное вмешательство}, снижающее неудовлетворённость, ослабляющее остроту проблемы, но не устраняющее ее полностью. Обычно применяется при дефиците ресурсов. Примеры: распределение по жребию или очереди.
	\item \important{SOLUTION} --- \note{оптимальное решение}, наилучшее в данных условиях. 
	\item \important{DISSOLUTION} --- вмешательство, заканчивающееся полным исчезновением проблемы и непоявлением новых проблем. Условия и ограничения рассматриваются не как незыблемо фиксированные, а как подлежащие изменениям с целью поиска новых, недопустимых ранее вариантов, среди которых могут оказаться гораздо более эффективные, чем ранее оптимальные.
\end{itemize}
Оптимальность обеспечивается только при совокупном соблюдении двух требований. Каковы эти требования?
\\
\note{Оптимальный} значит \note{<<наилучший в данных условиях>>}. Первым требованием является определение, по какому критерию или критериям будут упорядочиваться сравниваемые варианты, т.е. в каком смысле мы будем употреблять термин <<наилучший>>. Вторым требованием является определение, в рамкам каких ограничений будут выбираться сравниваемые варианты.
\\
Каков важный результат прикладного системного анализа конкретной проблемы, кроме решения самой проблемы?
\\
Выполнять процесс системного анализа будут сами работники фирмы клиента, т.к. для построения модели проблемной ситуации необходима информация, которой обладают только ее участники; воплощать разработанное вмешательство так же будут именно они. В выполнение какой-либо работы собственными усилиями является самой эффективной формой обучения этой деятельности. Таким образом, в прикладном системном анализе оказывается естественно встроенным, неотъемлемым элементом обучение самому системному анализу.
\\
В данной главые введены специальные понятия (и соответствующие им термины), входящие в профессиональный язык прикладного системного анализа. Часть этих терминов используется и в разговорном языке, но в другом, более расплывчатом, смысле. Другие несут специальную смысловую нагрузку. Проверьте, можете ли вы воспроизвести профессиональные определения для всех перечисленных ниже понятий (если какое-то нет --- обязательно найдите такое определение и постарайтесь его запомнить):
\begin{itemize}
	\item \important{Проблемная ситуация} --- некоторое реальное стечение обстоятельств, положение вещей, которым кто-то недоволен, неудовлетворён и хотел бы изменить.
	\item \important{Оценка} (чего-то субъектом) - мнение, суждение субъекта о чем-либо.
	\item \important{Проблема} --- субъективное отрицательное отношение субъекта к реальности.
	\item \important{Решение проблемы} --- уменьшение или исчезновение недовольства субъекта.
	\item \important{Вмешательство} --- изменение проблемной ситуации, снижающее недовольство клиента.
	\item \important{Улучшающее вмешательство} --- изменение проблемной ситуации, которое положительно оценивается хотя бы одним из ее участников и неотрицательно --- всеми остальными.
	\item \important{Прикладной системный анализ} --- теория и практика проектирования и реализации улучшающих вмешательств. Методика решения проблем реальной жизни без создания новых проблем.
	\item \important{Оптимальность} --- сочетание наилучших параметров по заданным критериям в рамках существующих ограничений.
	\item \important{"Твёрдая проблема"} --- хорошо структурированная, допускающая построение количественных математических моделей. Например, многие инженерные и научные проблемы.
	\item \important{"Мягкая проблема"} --- неструктурированные, описанные на естественном языке ситуации, не относящиеся к точным наукам.
\end{itemize}
\end{document}