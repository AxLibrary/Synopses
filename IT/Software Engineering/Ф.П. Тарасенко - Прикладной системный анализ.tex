\documentclass{article}
% Кодировка, поддержка русского языка
\usepackage[T2A]{fontenc}
\usepackage[utf8]{inputenc}
\usepackage[english,russian]{babel}
% Отступы от края страницы
\usepackage{geometry}
\geometry{left=4cm}
\geometry{right=4cm}
\geometry{top=2cm}
\geometry{bottom=2cm}
\geometry{bindingoffset=0cm}
% Расстояние между элементами списка, отступ слева
\usepackage{enumitem}
\setlist{nosep,leftmargin=*}

\newcommand{\note}[1]{\textit{#1}}
\newcommand{\important}[1]{\textbf{#1}}	
\renewcommand{\section}[2]{
	\vspace{6em}
	\begin{flushright}
	\Large
	\baselineskip=0.5\baselineskip
	\textbf{#1}
	\\
	\rule[0.5\baselineskip]{\textwidth}{0.15pt}
	\\
	\textbf{#2}
	\end{flushright}
	}
\renewcommand{\subsection}[1]{
	\vspace{2em}
	\begin{flushright}
		\large
		\textbf{#1}
	\end{flushright}
	}
\renewcommand{\title}[2]{
	\begin{center}
		\LARGE
		\baselineskip=0.5\baselineskip
		\textbf{#1}
		\\
		\rule[0.5\baselineskip]{0.7\textwidth}{0.15pt}
		\\
		\textbf{#2}
		\\\baselineskip=2\baselineskip(конспект)		
	\end{center}
	}
\newcommand{\define}[2]{
	\textbf{#1} --- #2
	}
\newcommand{\marked}[2]{
	\begin{flushright}\textbf{!}\hspace{2ex}\vline\hspace{2ex}
		\begin{minipage}{0.9\textwidth}
			\define{#1}{#2}
		\end{minipage}
	\end{flushright}
	}
\newcommand{\question}[2]{
	\begin{flushright}
		Q:\hspace{2ex}\vline\hspace{2ex}
		\begin{minipage}{0.9\textwidth}
			\large
			\textit{#1}
		\end{minipage}
	\end{flushright}
	\begin{center}
		\begin{minipage}{0.95\textwidth}
			#2
		\end{minipage}
	\end{center}
	}
\begin{document}
\title{Ф.П. Тарасенко}{Прикладной системный анализ}
\section{Введение}{Как возник системный анализ}
Любая деятельность человека состоит в решении постоянно возникающих \note{проблем}. Для их решения могут потребоваться самые различные специфические знания. Это создало впечатление, что проблемы различных специальностей уникальны, что привело к накоплению и обобщению опыта решения проблем в рамках каждой профессии отдельно.\\
Было замечено, что вероятность успеха повышается, если следовать одним и тем же советам, независимо от природы проблемы. Эта идея опирается на понятие всеобщей системности, олицетворяющей единство и общность законов мироздания.\\
Идея формирования общеупотребительной методики решения проблем была доведена до создания специальной технологии --- \important{прикладного системного анализа}. Данная дисциплина нацелена на решение конкретной проблемы, а не на поиск общих закономерностей. В теоретической сфере, прикладной системный анализ может требовать знаний из различных областей традиционных наук. В практической деятельности аналитик направляет коллектив участников ситуации, являющихся специалистами в требуемой области.
\section{Глава 1}{Проблема и способы ее решения}
\marked{Проблемная ситуация}{некоторое реальное стечение обстоятельств, положений вещей, которым кто-то недоволен и хотел бы изменить.}
\marked{Проблема}{Субъективное отрицательное отношение субъекта к реальности}
В понятиях проблемы и проблемной ситуации неразрывно связаны два аспекта --- объективный (наличие реальной ситуации) и субъективной (негативная оценка реальности субъектом). Отличие понятий в том, на чем делается акцент.
\marked{Решение проблемы}{какие-либо действия, призванные уменьшить или совсем снять недовольство субъекта.}
Все способы решения проблем можно разделить на две группы:
\begin{itemize}
	\item Воздействовать на субъект с целью уменьшить его недовольство, не изменяя реальности.
	\item Изменить реальность так, чтобы недовольство субъекта ослабло.
\end{itemize}
\subsection{Способы влияния на субъект}
Существует три возможности изменить к лучшему отношение субъекта к реальности, не изменяя самой реальности:
\begin{itemize}
	\item Сообщение субъекту дополнительной информации о ситуации, среди которой может оказаться информация позитивного характера. Может осуществляться в виде \note{обучения} субъекта.\\Дополнительная информация не обязательно должна быть правдивой, но обязательно должна быть положительной.\\Возможно сокрытие правды, либо отфильтрованная полуправда.
	\item Изменение восприятия данной реальности субъектом. Формы воздействия могут отличаться: психические (гипноз, пропаганда, реклама), физические (воздействие акустических, электрических, магнитных полей), химические (психотропные медикаменты, наркотики, алкоголь).
	\item Прекращение взаимодействия субъекта с ситуацией. Субъект может быть повышен, направлен в отпуск, переведен в другой отдел, уволен.
\end{itemize}
\subsection{Вмешательство в реальность}
В реальной ситуации участвуют не только недовольный субъект, но и многие другие субъекты, которые оценивают эту же ситуацию со своих позиций. Всякое изменение ситуации в результате вмешательства, будет замечено и оценено всеми ее участниками.\\
Субъект существует в реальной физической среде и подвержен её воздействиям. В отличии от объекта, субъект не только подчинён природным закономерностям, но и наделен способностью \note{оценивать} свои взаимодействия со средой: что-то ему может нравиться, а может и нет. Таким образом, все оценки имеют сугубо субъективный, индивидуальный характер.
\note{Всякий раз, когда в вашем присутствии прозвучит любое оценочное слово} (хорошо --- плохо, полезно --- вредно и т.п.), \note{насторожитесь и задайте вопрос <<В каком смысле?>>}. Оценки не бывают объективны и если вы хотите понять истинный смысл сказанного, надо выяснить, какие критерии применяет оценивающий.
\subsection{Три типа идеологий вмешательства}
\note{Правильным} считается поведение, максимально согласующееся с принятой субъектом идеологией. Именно идеология и определяет, что плохо, а что хорошо. Идеологии могут быть различными, отличаясь определением того, \note{какое отношение к другим субъектам считать правильным}. Явные различия можно провести между тремя типами идеологий:
\begin{itemize}
	\item Первый тип идеологии --- <<принцип приоритета меньшинства>>. Этот принцип приводит к тому, чтобы осуществить вмешательство, угодное клиенту, а интересы других участников не принимаются во внимание. Жизненные примеры реализации --- диктатура, иерархическая организация, эгоизм, и т.д.
	\item Второй тип идеологии --- <<принцип приоритета группы>>. Согласно ей среди участников ситуации, кроме клиента, есть другие субъекты, не менее важные и ценные. Вмешательство должно проводиться с учётом интересов всех <<наших>>. Примеры --- расизм, национализм, фашизм, коммунизм, и т.д.
	\item Третий тип --- <<принцип приоритета каждого>>. В основе лежит два постулата: нет ни одного одинакового субъекта; все субъекты равноценны и равноправны. Правильным, моральным признается только улучшающее вмешательство.
\end{itemize}
\marked{Улучшающее вмешательство}{изменение проблемной ситуации, которое положительно оценивается хотя бы одним из ее участников и неотрицательно --- всеми остальными.}
\marked{Прикладной системный анализ}{теория и практика проектирования и реализации улучшающих вмешательств. Методика решения проблем реальной жизни без создания новых проблем.}
\subsection{Четыре типа вмешательств}
Выделяют четыре типа способов решения проблем:
\begin{itemize}
	\item \important{ABSOLUTION} --- \note{невмешательство} в расчете на то, что естественный ход событий приведет к разрешению проблемы. Невмешательство обладает одним из признаков улучшающего вмешательства: при этом никому не становится хуже. Чтобы стать <<улучшающим вмешательством>>, необходимо, чтобы события действительно вели к разрешению проблемы, а выбор подобного поведения может быть обусловлен тем, что любые предлагаемые вмешательства приводят к худшим результатам. Примеры: поведения врача при невозможности исцелению пациента, действия сапера при встрече с незнакомым взрывным устройством.
	\item \important{RESOLUTION} --- \note{частичное вмешательство}, снижающее неудовлетворенность, ослабляющее остроту проблемы, но не устраняющее ее полностью. Обычно применяется при дефиците ресурсов. Примеры: распределение по жребию или очереди.
	\item \important{SOLUTION} --- \note{оптимальное решение}, наилучшее в данных условиях. 
	\item \important{DISSOLUTION} --- вмешательство, заканчивающееся полным исчезновением проблемы и не появлением новых проблем. Условия и ограничения рассматриваются не как незыблемо фиксированные, а как подлежащие изменениям с целью поиска новых, недопустимых ранее вариантов, среди которых могут оказаться гораздо более эффективные, чем ранее оптимальные.	
\end{itemize}
\marked{Оптимальность}{сочетание наилучших параметров по заданным \note{критериям} в рамках существующих \note{условий} и ограничений.}
\subsection{Еще о прикладном системном анализе}
Процесс решения проблемы не может быть выполнен лишь самим системным аналитиком. Построение модели проблемной ситуации требует информации, которой обладают сами ее участники. Их вовлечение в процесс работы над проблемой является обязательным элементом технологии. Более того, воплощать разработанное вмешательство так же будут они.\\
Выполнение какой-либо работы собственными усилиями является самой эффективной формой обучения. Таким образом, в прикладном системно анализе оказывается естественно встроенным обучение самому системному анализу.
\section{Глава 2}{Понятие системы}
\subsection{Статические свойства системы}
\note{Статическими свойствами} называются особенности конкретного состояния системы.
\marked{Целостность}{первое свойство}
\define{Целостность}{существование системы как чего-то единого, целого, обособленного, отличающегося от всего остального; факт внешней различимости в среде.}
\marked{Открытость}{второе свойство}
\define{Открытость}{связность системы и окружающей среды, обмен между ними любыми видами ресурсов.}\\
Связи системы со средой имеют направленный характер: по одним среда влияет на систему (\note{входы}), по другим система оказывает влияние на среду (\note{выходы}). Перечень входов и выходов системы называют \note{моделью черного ящика}. В этой модели отсутствует информация о внутренних особенностях системы.\\
В модели должны быть отражены все связи, существенные для достижения цели. Оценку важности той или иной связи может дать только субъект. Возможны четыре типа ошибок при построении модели.
\begin{itemize}
	\item \note{Ошибка первого рода} происходит, когда субъект расценивает связь как существенную и принимает решение о включении ее в модель, хотя на самом деле она несущественна и могла бы быть не учитываемой.
	\item \note{Ошибка второго рода} происходит, когда субъект принимает решение, что данная связь несущественна и не должна быть включена в модель.
	\item \note{Ошибкой третьего рода} принято считать последствия незнания о существовании какой-либо связи. Если связь существенна, испытываемые трудности будут соответствовать ошибке второго рода. Ошибку третьего рода труднее исправить: необходимо добывать новые знания.
	\item \note{Ошибка четвертого рода} может возникнуть при неверном отнесении известной и признанной существенной связи к числу входов или выходов. 
\end{itemize}
Следствием открытости систем является очевидность \note{всеобщей взаимосвязи и взаимозависимости в природе}. Между любыми двумя системами обязательно существует, и ее можно отыскать, длинная или короткая цепочка систем, связывающая их: выход каждой системы является входом другой.
\marked{Внутренняя неоднородность}{третье свойство}
Описание внутренней неоднородности системы сводится к обособлению относительно однородных участков, проведения границ между ними, выделения \note{частей системы}. Выделенные крупные части тоже не однородны, что позволяет выделить еще более мелкие части и получить иерархический список частей системы --- \note{модель состава системы}.\\
Трудности построения модели состава можно представить тремя положениями:
\begin{itemize}
	\item Целое можно делить на части по-разному, в зависимости от того, что требуется для достижения цели. При этом, можно \note{различать} нужные для цели части, но не следует \note{разделять их}.
	\item Количество частей в модели состава зависит от того, на каком уровне остановить дробление системы. Части на конечных ветвях получающегося иерархического дерева называют \note{элементами}. Прекращение декомпозиции производится на разных уровнях, в зависимости от обстоятельств и того, что считать \note{элементарным}.
	\item Любая система является частью какой-либо большей системы. Мета-систему тоже можно делить на подсистемы по-разному. Таким образом, \note{внешняя граница системы имеет относительный, условный характер}. Определение границ системы производится с учетом целей субъекта.
\end{itemize}
\marked{Структурированность}{четвертое свойство}
Части системы не независимы и не изолированы друг с другом, они связаны и взаимодействуют между собой. При этом свойства системы в целом существенно зависят от того, как именно взаимодействуют ее части. Перечень \note{существенных связей между элементами системы} называется \note{моделью структуры системы}.
\begin{itemize}
	\item Модель структуры определяется после выбора модели состава и зависит от нее. При этом, модель структуры вариабельна даже при зафиксированном составе из-за возможности по-разному определить существенность связей.
	\item Каждый элемент системы представляет собой <<черный ящик>>, соответственно могут привнести ошибки, полученные при определении своих входов и выходов.
\end{itemize}
\subsection{Динамические свойства системы}
Динамическими свойствами именуются особенности изменений со временем внутри системы и вне ее. О любых изменениях мы имеем возможность говорить в терминах перемен в статических моделях системы.
\marked{Функциональность}{пятое свойство}
Процессы на выходах системы, рассматриваются как ее \note{функции}.\define{Функции системы}{это ее поведение во внешней среде; изменения, производимые системой в окружающей среде; результаты ее деятельности.} Из множественности выходов следует множественность функций. Субъект, использующий систему, будет \note{оценивать} ее функции и \note{упорядочивать} их по отношению к своим потребностям.
\marked{Стимулируемость}{шестое свойство}
Процессы на входах системы называются \note{стимулами}.\define{Стимулируемость}{подверженность системы воздействиям извне и изменение ее поведения под этими воздействиями.}
\marked{Изменчивость системы со временем}{изменение значений внутренних параметров, состава и структуры системы и любых их комбинаций в .}
\note{Функционированием} называется тип динамики системы, при котором изменения не затрагивают структуру системы: одни элементы заменяются другими, эквивалентными; параметры могут меняться без изменения структуры.\\
Изменения могут носить преимущественно \note{количественный} характер. Наращивание состава системы называют \note{ростом}. Обратные изменения называют \note{спадом}.\\
При \note{качественных} изменения системы происходит изменение ее существенных свойств. Если такие изменения идут в позитивном направлении, они называются \note{развитием}. Утрату или ослабление полезных свойств именуют \note{деградацией}. Развитая система добивается более высоких результатов с теми же ресурсами.\\
Рост есть увеличение в размерах и численности; развитие --- увеличение компетентности, т.е. знаний. Объемность --- результат роста, компетентность --- развития. Недостаток материальных ресурсов может ограничить рост, но не развитие; развитие извне не ограничено.\\
Существует внутреннее ограничение на развитие. Развитие есть результат \note{обучения}, но обучение нельзя осуществить для и вместо обучаемого. Таким образом, \note{развитие возможно только как саморазвитие}.\\
Монотонные изменения не могут длиться вечно, в истории системы можно усмотреть периоды спада и подъема, последовательность которых образует индивидуальный \note{жизненный цикл} системы. При построении описания жизненного цикла, особое внимание необходимо обратить на непрерывность его траектории, так как история проектируемой системы закончится на первом же пробеле в описании ее жизненного цикла.
\marked{Существование в изменяющейся среде}{восьмое свойство}
Изменяется не только сама система, но и окружающая ее среда. Необходимость существовать в подобном окружении имеет множество последствий для самой системы, начиная с необходимости ее приспособления к внешним переменам, до различных других реакций системы.\\
Чем сильнее внешние изменения, тем активнее должны проводиться внутренние. И хотя важными средствами остаются прогнозирование и обучение, более эффективными считаются выработка иммунитета к неподконтрольным изменениям и усиление контроля над остальными.
\subsection{Синтетические свойства системы}
\note{Синтетическими} называют обобщающие, собирательные, интегральные свойства, делающие упор на взаимодействия системы со средой, на целостность.
\marked{Эмерджентность}{девятое свойство}
Объединение частей в систему порождает у системы качественно новые свойства, не сводящиеся к свойствам частей, не выводящиеся из свойств частей, присущие только самой системе и существующие только пока система составляет одно целое. Например, ни одна из частей самолета летать не может, а самолет летает. Эмерджентные свойства определяются \note{структурой} системы.
\begin{itemize}
	\item У системы есть и не-эмерджентные свойства, одинаковые со свойствами ее частей. При \note{фрактальном} построении системы, части обладают свойствами системы в целом.
	\item Система выступает как единое целое потому, что она является носителем эмерджентного свойства.
	\item Для появления нового качества достаточно объединить хотя бы два элемента.
	\item В естественных системах эмерджентность определяет, какие части должны быть соединены и как они должны взаимодействовать. Живой организм определяет смысл скелета, сердца и печени.
	\item Действие системы больше зависит от того, как ее части взаимодействуют, чем от того, как они действуют сами по себе.
\end{itemize}
\marked{Неразделимость на части}{десятое свойство}
При изъятии из системы некоторой части изменяется состав системы, а значит и ее структура. Это будет уже \note{другая} система, с отличающимися свойствами.\\
Часть в системе и вне ее - это не одно и то же. Изменяются ее свойства, т.к. свойства объекта проявляются во взаимодействиях с окружающими его объектами.
\marked{Ингерентность}{одиннадцатое свойство}
Система тем более ингерентна, чем лучше она согласована, приспособлена к окружающей среде. Ингерентность привязана к некоторой конкретной функцией.\\
В естественных системах ингерентность повышается путем естественного отбора. В искусственных системах она должна быть особой заботой конструктора.
\marked{Целесообразность}{двенадцатое свойство системы}
\define{Целесообразность}{подчиненность всего поставленной цели.}Цель определяет, какое эмерджентное свойство будет обеспечивать реализацию цели. \note{Система есть средство достижения цели}.\\
Под целью стоит понимать \note{желаемые будущие состояние системы в различные моменты времени}. Конечная цель --- \note{будущее реальное состояние}. <<Цель как образ желаемого будущего>> и <<цель как реальное будущее>> --- не одно и то же; первое --- \note{субъективная цель}, а второе --- \note{объективная цель}. Субъективные цели есть порождение воображения, тогда как объективные --- результат проявления законов природы.  Ограничения на мысленные конструкции гораздо слабее ограничений на возможные реальные события, поэтому достижима не всякая субъективная цель, но лишь так, которая принадлежит к числу объективных целей.\\
Важно установить реализуемость субъективной цели до начала попыток реализовать ее. Заведомо недостижимые цели, не считаемые недостойными стремления к ним, называются \note{идеалами}.
\subsection{Системная картина мира}
Системное видение мира состоит в том, чтобы, понимая его всеобщую системность, приступить к рассмотрению конкретной системы, уделяя основное внимание ее индивидуальным особенностям: \note{думай глобально, действуй локально}
\section{Глава 3}{Модели и моделирование}
\subsection{Моделирование --- необъемлемая часть любой деятельности}
Все возможные виды деятельности можно разбить на два типа: познание мира и преобразование его. Любая деятельность субъекта становится возможной только благодаря \note{моделям} --- системам, предназначенным для обеспечения взаимодействия между субъектом и реальностью.\\
Что бы человек не делал, до начала самой работы он должен определить цель, т.е. модель того, чего пока нет. Для достижения конечного результата необходимо выполнить определенную последовательность промежуточных действий, для чего их нужно \note{описать}. Таким образом, преобразовательная деятельность невозможна без моделирования.\\
Конечный результат познания должен быть зафиксирован, описан, представлен в виде определенной модели. Конечной целью познания является построение моделей интересующей нас части мира. \note{Модель есть форма существования знания}.
\subsection{Анализ и синтез как методы построения моделей}
Процедура анализа состоит в последовательном выполнении следующих операций:
\begin{itemize}
	\item Сложно расчленить на более мелкие, предположительно более простые, части.
	\item Дать объяснение полученным фрагментам.
	\item Объединить объяснение частей в объяснение целого.
\end{itemize}
Если какая-то часть системы остается непонятной, операция декомпозиции повторяется. Самая серьезная ловушка анализа - опасность разорвать связи частей, нарушив эмерджентные свойства системы. Таким образом, правильный анализ должен осуществлять различение частей, а не разбиение на части.\\
Продуктами анализа являются модель состава и структуры системы, модель черного ящика для каждого элемента системы.
Процедура синтеза состоит в последовательном выполнении следующих операций:
\begin{itemize}
	\item Выделение метасистемы.
	\item Рассмотрение состава и структуры метасистемы.
	\item Объяснение роли искомой системы в метасистеме через ее связи с другими подсистемами.
\end{itemize}
Продуктами синтеза являются модели состава и структуры метасистемы, модель черного ящика системы.\\
Анализ и синтез не противоположны, а дополняют друг друга. В анализе есть синтетический компонент, а в синтезе --- аналитический.
\subsection{Что такое модель?}
\begin{itemize}
	\item Модель есть средство осуществления любой деятельности субъекта.
	\item Модель есть форма существования знаний.
	\item Модель есть системное отображение оригинала.
\end{itemize}
\subsection{Аналитический подход к понятию модели}
Модели подразделяются на \note{абстрактные} --- средства мышления и \note{реальные} --- материальные средства. Абстрактные модели могут быть воплощены средствами языка для передачи другим субъектам. Язык является \note{универсальным} средством моделирования. Одно из свойств, делающих это возможным, является \note{расплывчатость смысла слов}.\\
Всякая групповая деятельность требует выработки специального, более точного \note{профессионального} языка. Профессиональные языки более точны, нежели разговорный, за счет большей определенности терминов. Снятие неопределенности возможно только за счет новой, \note{дополнительной информации}. Максимальный предел точности --- язык математики. Элементарная языковая модель --- \note{слово}.\\
Существует спектр языков разной степени определенности, которым соответствует спектр моделей разной степени точности. Одна из главных особенностей прикладного системного анализа --- попытка развить описание проблемной ситуации в сторону более точного описания. Важен факт движения в сторону уточнения, пока точности определения условий не хватит для решения проблемы.
\subsection{Классификация --- простейшая абстрактная модель разнообразия реальности}
Описать бесконечно разнообразный мир конечными фразами можно только упрощенно и приблизительно. Выделение мало различающихся объектов в группы, с последующим рассмотрением членов группы как одинаковых между собой, дает \note{класс}. Оставшиеся вне класса объекты так же могут быть объединены между собой в новые классы, что позволит описать мир конечным множеством отличающихся друг от друга классов. Слова языка представляют собой названия некоторых классов. \note{Классификация есть простейшая абстрактная модель разнообразия действительности.}\\
\note{Любая} классификация есть только \note{модель} разнообразия реальности. Всегда найдется объект, который нельзя однозначна отнести к тому или иному классу.
\subsection{Искусственная и естественная классификации}
При искусственной классификации разделение на классы производится исходя из поставленной цели --- на столько классов и с такими границами, как это диктуется целью. Искусственную классификацию так же называют \note{произвольной}.\\
Естественная классификация подразумевает классификацию на основе существующих природных группировок, которые как бы напрашиваются быть определенными как классы.\\
Классификация лежит в основе более сложных абстрактных моделей, что достигается как увеличением числа классов, так и введением новых соотношений между классами.
\subsection{Реальные модели}
Реальной моделью может выступать какой-либо материальный предмет, используемый в качестве модели. Классификация по происхождению подобия между оригиналом и моделью, приводит к трем типам реальных моделей:
\begin{itemize}
	\item Модели \note{прямого подобия} созданы с помощью либо непосредственного взаимодействия (следы, печать), либо цепочки таких взаимодействий (фотография, макет здания).
	\item Модели \note{косвенного подобия} или аналогии. Похожесть, аналогичность двух явлений объясняется совпадением закономерностей, которым они подчиняются. Моделями косвенного подобия являются: аналоговые ЭВМ, исторические параллели, подопытные животные в медицине.\\
	Следует осторожно пользоваться аналогиями, поскольку, кроме совпадающих закономерностей, у разных явлений есть и несовпадающие.
	\item Модели \note{условного подобия}. Соответствие такой модели и оригинала устанавливается в результате соглашения между ее пользователями и носит условный характер. Они успешно работают до тех пор, пока известны и соблюдаются договоры об их значении. Примеры: деньги --- модели стоимости, буквы --- модели звуков.
\end{itemize}
\subsection{Синтетический подход к понятию модели}
Модель никогда не тождественна оригиналу. Зачастую в этом нет необходимости, т.к. для определенной цели требуется лишь некоторая, а не вся информация об оригинале. Разнообразие целей ведет к множественности моделей для одного и того же оригинала. Модели можно различать по типу целей, например, полезным бывает разделение моделей на \note{познавательные} и \note{прагматические}.\\
Оригинал и модель могут различаться между собой. Во-первых, не вся информация об оригинале необходима в модели. \note{Истинная} информация является общей и для модели, и для оригинала. Именно благодаря ей модель может служить заменителем оригинала. Собственные свойства модели, не имеющие отношения к оригиналу, называются \note{ложными}.\\
Познавательные модели обслуживают процессы получения информации о внешнем мире и не претендуют на окончательность: всегда остается что-то непознанное. Они подвержены изменениям при присоединении к ним новых знаний. В познавательной практике принято терпимо относиться к отличающимся и даже противоречивым мнениям.\\
Прагматические модели обслуживают процессы преобразования реальности в соответствии с целями субъекта, отображая несуществующее, но желаемое, и имеют директивный характер. Это придает им статус <<единственно верных>>, что ярко выражено в религиях, морали, стандартах, etc. В прагматической деятельности реальность <<подгоняется>> под модель.
\subsection{Понятие адекватности}
Разные модели обеспечивают разную степень успешности в достижении цели, это свойство можно назвать \note{степенью их адекватности}. Модели, позволяющие субъекту достигнуть цели, называются \note{адекватными}, а не обеспечивающие успеха --- \note{неадекватными}.\\
Для познавательных моделей, целевая принадлежность которых --- накопление истинных знаний, адекватность и истинность являются, по существу, синонимами. Для прагматических моделей достижение цели может быть проще с помощью лжи. Таким образом, ложные модели могут быть адекватными.
\subsection{Согласованность модели с культурой}
Чтобы модель реализовала свою модельную функцию, недостаточно только наличия самой модели. Необходимо, что бы модель была совместима, согласована с окружающей средой. \note{Ингерентность модели культуре} является необходимым требованием для осуществление моделирования.
\subsection{Иерархия моделей}
Модели могут описывать реальное и желаемое состояния рассматриваемой системы с разной степенью подробности. Акофф предложит такую классификацию:
\begin{itemize}
	\item \note{Данные} --- (что?) --- описание результатов измерений; исходные, <<сырые>> данные.
	\item \note{Информация} --- (состав?) --- результат первичной обработки данных, их упорядочение, классификация, структуризация.
	\item \note{Знание} --- (структура?) --- результат вторичной обработки данных; выявления связей и закономерностей между группами, классами данных.
	\item \note{Понимание} --- (почему?) --- объяснение выявленных закономерностей, построение теорий, дающих такое объяснение.
	\item \note{Мудрость} --- (зачем?) --- сведения о том, зачем это все надо, хорошо ли это, надо ли это продолжать или прекращать --- подход с точки зрения эстетики и этики.
\end{itemize}
\subsection{Заключение}
Любая деятельность субъекта возможна только благодаря моделированию.\\
Модель есть отображение оригинала: целевое; абстрактное или реальное; упрощенное, приближенное; имеющее как истинное, так и ложное содержание; адекватное цели; ингерентное культуре пользователя.
\section{Глава 4}{Управление}
Исходное определение управления --- \note{целенаправленное воздействие на систему}.
\subsection{Аналитический подход к управлению: пять компонентов управления}
Первым компонентом управления является сам \note{объект управления, управляемая система}.\\
Вторым обязательным компонентом системы управления является \note{цель управления}.\\
\note{Управляющее воздействие} есть третий компонент управления. Тот факт, что входы и выходы системы связаны между собой, позволяется надеяться, что существует такое управляющее воздействие, при котором на выходе реализуется цель.\\
Существует два типа управления:
\begin{itemize}
	\item Первый состоит в том, чтобы подать на управляемый вход какое-либо воздействие и посмотреть, что получиться. Если цель не достигнута --- подать другое воздействие и т.д. Иногда такой способ может быть единственно возможным, но чаще всего, такое способ управления является неразумным по ряду причин: цена, сроки, etc.
	\item Второй подход основан на использовании всей имеющейся информации об управляемом объекте. Это означет, что поиск необходимого воздействия следует осуществлять не на самой системе, а на ее модели.
\end{itemize}
Таким образом, \note{модель системы} становится четветртой часть процесса управления. Поиск управления на модели тоже требует потерь, но эти потери несравнимо меньше тех, которые были бы понесены во время поиска нужного управления на самой системе.\\
Функция выполнения действий, необходимых для управления, возлагает на пятую составляющую процесса управления --- \note{блок управления, система (подсистема) управления, управляющее устройство}. Блок управления может быть как подсистемой управляемой системы, так и внешней системой.\\
Два первых обязательных шага процесса управления:
\begin{itemize}
	\item Найти \note{на модели} системы нужное управляющее воздействие.
	\item Выполнить это воздействие \note{на системе}.
\end{itemize}
\subsection{Этап нахождения нужного управления}
Управление тем <<лучше>>, чем ближе выход системы к цели. Поскольку поиск управления проходит на модели, наилучшим придется считать то управление, которое максимально приблизит выход модели к цели.\\
Если выходы измеримы численно, то вводится некоторый числовой критерий, который равнялся бы нулю при свопадении сравниваемых функций и возрастал при любом их различии. Для целей, задаваемых нечисловым способом, все-равно вводятся измеримые характеристики близости результата к цели.
\subsection{Синтетический подход к управлению: семь типов управления}
\marked{Первый тип управления}{\note{управление простой системой или программное управление}}
Если подача на вход системы воздействия, обеспечивающего цель на выходе модели, приводит к такому же результату на выходе системы, это означает, что модель оказалась \note{адекватной}. В этом сулчае, систему будем называть \note{простой}. Простота есть следсвтие адекватности модели. Управляющее возедйствие в таком случае называется \note{программой}, а данный тип управления --- \note{программным управлением}.\\
Примеры: исправные бытовые приборы, различные автоматы, компьютеры, etc.
\marked{Второй тип управления}{\note{управление сложной системой}}
Если на найденное на модели управляющее воздействие откликается вовсе не так, как модель, это означает, что модель оказалась \note{неадекватна}. Система не подчиняется управлению и является \note{сложной}. Причиной сложности оказывается неадекватность ее модели, вызванная недостаточностью информации об управляемом объекте.

\question{Какие пять составляющих обеспечивают выполнение процесса управления?}{?}
\question{При каких условиях поиск управляющего воздействия на самой системе является неразумным, неприемлемым?}{?}
\question{Что называется простой системой? В чем причина простоты?}{?}
\question{Какую систему называют сложной? Какова причина сложности?}{?}
\question{Опишите алгоритм метода проб и ошибок. Какими особенностями он обладает?}{?}
\question{Чем отличается метод проб и ошибок от <<метода тыка>>?}{?}
\question{Перечислите, какие функции выполняет регулятор.}{?}
\question{В чем состоит управление по целям? При каких условиях применим этот тип управления?}{?}
\question{Что такое <<большая система>>? Каковы варианты управления ею?}{?}
\question{Придумайте примеры систем, которые были бы одновременно: малой и простой, малой и сложной, большой и простой, большой и сложной.}{?}

\newpage
\section{Приложение}{Вопросы для Anki}
\subsection{Введение}
\question{Почему накопление и обобщение опыта решения проблем началось (и продолжается) в рамках каждой отдельной профессии?}{Для решения проблем могут потребоваться различные профессиональные знания. Это создает впечатление, что проблемы различных специальностей уникальны, что и приводит к накоплению соответствующего опыта в рамках отдельных профессий.}
\question{Почему, несмотря на громадное разнообразие проблем, технология (совокупность приемов) их решения практически одинакова в случае успеха и различается в случае неудач?}{Идея универсального алгоритма действий по решению проблем опирается на понятие всеобщей системности, олицетворяющей единство и общность законов мироздания.}
\question{Сформулируйте основные отличия прикладного системного анализа от традиционных наук.}{Прикладной системный анализ нацелен на решение конкретной проблемы, а не на поиск общих закономерностей; для решения проблемы могут понадобиться знания любой из традиционных наук.}
\question{Почему прикладной системный анализ можно назвать над-дисциплинарной и меж-дисциплинарной область деятельности как в теоретической, так и в практической его сфере?}{В теоретической сфере, прикладной системный анализ может требовать использования знаний из различных областей традиционных наук. В практической деятельности, аналитик направляет коллектив участников ситуации, являющихся специалистами в требуемой области знания.}
\subsection{Проблема и способы ее решения}
\question{Поясните различия между понятиями "проблемная ситуация" и "проблема"}{Термин \note{"проблемная ситуация"} выделяет объективный компонент, реальную ситуацию; \note{"проблема"} --- субъективное отношение к реальной ситуации субъекта.}
\question{Что значит "решить проблему"?}{Решение проблемы объединяет в себе какие-либо действия, призванные уменьшить или совсем снять недовольство субъекта.}
\question{Какие три способа воздействия на субъект без изменения реальности могут (при определенных условиях) привести к решению его проблемы? Каковы эти условия?}{
	\begin{itemize}
		\item Сообщение субъекту дополнительной информации о ситуации, среди которой может оказаться информация позитивного характера. Может осуществляться в виде \note{обучения} субъекта.\\Дополнительная информация не обязательно должна быть правдивой, но обязательно должна быть положительной.\\Возможно сокрытие правды, либо отфильтрованная полуправда.
		\item Изменение восприятия данной реальности субъектом. Формы воздействия могут отличаться: психические (гипноз, пропаганда, реклама), физические (воздействие акустических, электрических, магнитных полей), химические (психотропные медикаменты, наркотики, алкоголь).
		\item Прерывание взаимодействия субъекта с ситуацией. Субъект может быть повышен, направлен в отпуск, переведен в другой отдел, уволен.
	\end{itemize}}
\question{Каково основное отличие субъекта от объекта?}{Субъект существует в реальной физической среде и подвержен её воздействиям. В отличие от объекта, субъект не только подчинён природным закономерностям, но и наделен способностью \note{оценивать} свои взаимодействия со средой: что-то ему может нравиться, а может и не нравиться. Таким образом, все оценки имеют сугубо субъективный, индивидуальный характер.}
\question{Как определить смысл оценки, выраженной неким субъектом?}{\note{Всякий раз, когда в вашем присутствии прозвучит любое оценочное слово} (хорошо --- плохо, полезно --- вредно и т.п.), \note{насторожитесь и задайте вопрос <<В каком смысле?>>}. Оценки не бывают объективны и если вы хотите понять истинный смысл сказанного, надо выяснить, какие критерии применяет оценивающий.}
\question{Почему при вмешательстве в реальность с целью решения проблемы приходится опираться на какую-то идеологию?}{\note{Правильным} считается поведение, максимально согласующееся с принятой субъектом идеологией. Именно идеология и определяет, что плохо, а что хорошо.}
\question{Воспроизведите классификацию идеологий на три типа. Каково основное отличие между ними?}{Основное отличие между идеологиями --- это определение того, \note{какое отношение к другим субъектам считать правильным}.
	\begin{itemize}
		\item Первый тип идеологии --- <<принцип приоритета меньшинства>>. Этот принцип приводит к тому, чтобы осуществить вмешательство, угодное клиенту, а интересы других участников не принимаются во внимание. Жизненные примеры реализации --- диктатура, иерархическая организация, эгоизм, и т.д.
		\item Второй тип идеологии --- <<принцип приоритета группы>>. Согласно ей среди участников ситуации, кроме клиента, есть другие субъекты, не менее важные и ценные. Вмешательство должно проводиться с учётом интересов всех <<наших>>. Примеры --- расизм, национализм, фашизм, коммунизм, и т.д.
		\item Третий тип --- <<принцип приоритета каждого>>. В основе лежит два постулата: нет ни одного одинакового субъекта; все субъекты равноценны и равноправны. Правильным, моральным признается только улучшающее вмешательство.
	\end{itemize}}
\question{Назовите четыре типа улучшающих вмешательств.}{
	\begin{itemize}
		\item \important{ABSOLUTION} --- \note{невмешательство} в расчете на то, что естественный ход событий приведет к разрешению проблемы. Невмешательство обладает одним из признаков улучшающего вмешательства: при этом никому не становится хуже. Чтобы стать <<улучшающим вмешательством>>, необходимо, чтобы события действительно вели к разрешению проблемы, а выбор подобного поведения может быть обусловлен тем, что любые предлагаемые вмешательства приводят к худшим результатам. Примеры: поведения врача при невозможности исцелению пациента, действия сапера при встрече с незнакомым взрывным устройством.
		\item \important{RESOLUTION} --- \note{частичное вмешательство}, снижающее неудовлетворенность, ослабляющее остроту проблемы, но не устраняющее ее полностью. Обычно применяется при дефиците ресурсов. Примеры: распределение по жребию или очереди.
		\item \important{SOLUTION} --- \note{оптимальное решение}, наилучшее в данных условиях. 
		\item \important{DISSOLUTION} --- вмешательство, заканчивающееся полным исчезновением проблемы и непоявлением новых проблем. Условия и ограничения рассматриваются не как незыблемо фиксированные, а как подлежащие изменениям с целью поиска новых, недопустимых ранее вариантов, среди которых могут оказаться гораздо более эффективные, чем ранее оптимальные.
	\end{itemize}}
\question{Оптимальность обеспечивается только при совокупном соблюдении двух требований. Каковы эти требования?}{\note{Оптимальный} значит \note{<<наилучший в данных условиях>>}. Первым требованием является определение, по какому критерию или критериям будут упорядочиваться сравниваемые варианты, т.е. в каком смысле мы будем употреблять термин <<наилучший>>. Вторым требованием является определение, в рамкам каких ограничений будут выбираться сравниваемые варианты.}
\question{Каков важный результат прикладного системного анализа конкретной проблемы, кроме решения самой проблемы?}{Выполнять процесс системного анализа будут сами работники фирмы клиента, т.к. для построения модели проблемной ситуации необходима информация, которой обладают только ее участники; воплощать разработанное вмешательство так же будут именно они. В выполнение какой-либо работы собственными усилиями является самой эффективной формой обучения этой деятельности. Таким образом, в прикладном системном анализе оказывается естественно встроенным, неотъемлемым элементом обучение самому системному анализу.}
\question{Дайте определение следующим понятиям:}{
	\begin{itemize}
		\item \define{Проблемная ситуация}{некоторое реальное стечение обстоятельств, положение вещей, которым кто-то недоволен, неудовлетворён и хотел бы изменить.}
		\item \define{Оценка}{мнение, суждение субъекта о чем-либо.}
		\item \define{Проблема}{субъективное отрицательное отношение субъекта к реальности.}
		\item \define{Решение проблемы}{какие-либо действия, призванные уменьшить или совсем снять недовольство субъекта.}
		\item \define{Вмешательство}{изменение проблемной ситуации, снижающее недовольство клиента.}
		\item \define{Улучшающее вмешательство}{изменение проблемной ситуации, которое положительно оценивается хотя бы одним из ее участников и неотрицательно --- всеми остальными.}
		\item \define{Прикладной системный анализ}{теория и практика проектирования и реализации улучшающих вмешательств. Методика решения проблем реальной жизни без создания новых проблем.}
		\item \define{Оптимальность}{сочетание наилучших параметров по заданным критериям в рамках существующих ограничений.}
		\item \define{"Твёрдая проблема"}{хорошо структурированная, допускающая построение количественных математических моделей. Например, многие инженерные и научные проблемы.}
		\item \define{"Мягкая проблема"}{неструктурированные, описанные на естественном языке ситуации, не относящиеся к точным наукам.}
	\end{itemize}}
\subsection{Понятие системы}
\question{Что такое статические свойства систем? Перечислите четыре статических свойства.}{\note{Статическими свойствами} называются особенности конкретного состояния системы.
	\begin{itemize}
		\item \define{Целостность}{существование системы как чего-то единого, целого, обособленного, отличающегося от всего остального; факт внешней различимости в среде.}
		\item \define{Открытость}{связность системы и окружающей среды, обмен между ними любыми видами ресурсов.}
		\item \define{Внутренняя неоднородность}{различимость частей системы.}
		\item \define{Структурированность}{Наделенность любой системы определенной структурой.}
	\end{itemize}}
\question{Как из открытости систем вытекает факт всеобщей взаимосвязанности в природе?}{Следствием открытости систем является очевидность \note{всеобщей взаимосвязи и взаимозависимости в природе}. Между любыми двумя системами обязательно существует, и ее можно отыскать, длинная или короткая цепочка систем, связывающая их: выход каждой системы является входом другой.}
\question{Что называется моделью черного ящика? Назовите четыре рода ошибок, которые можно совершить при построении модели черного ящика.}{Перечень входов и выходов системы называют \note{моделью черного ящика}. В этой модели отсутствует информация о внутренних особенностях системы.
	\begin{itemize}
		\item \note{Ошибка первого рода} происходит, когда субъект расценивает связь как существенную и принимает решение о включении ее в модель, хотя на самом деле она несущественна и могла бы быть неучитываемой.
		\item \note{Ошибка второго рода} происходит, когда субъект принимает решение, что данная связь несущественна и не должна быть включена в модель.
		\item \note{Ошибкой третьего рода} принято считать последствия незнания о существовании какой-либо связи. Если связь существенна, испытываемые трудности будут соответствовать ошибке второго рода. Ошибку третьего рода труднее исправить: необходимо добывать новые знания.
		\item \note{Ошибка четвертого рода} может возникнуть при неверном отнесении известной и признанной существенной связи к числу входов или выходов. 
	\end{itemize}}
\question{Что называется моделью состава системы? Каковы три трудности ее построения?}{\define{Модель состава системы}{иерархический список частей системы.}\\Трудности построения модели состава можно представить тремя положениями:
	\begin{itemize}
		\item Целое можно делить на части по-разному, в зависимости от того, что требуется для достижения цели. При этом, можно \note{различать} нужные для цели части, но не следует \note{разделять их}.
		\item Количество частей в модели состава зависит от того, на каком уровне остановить дробление системы. Части на конечных ветвях получающегося иерархического дерева называют \note{элементами}. Прекращение декомпозиции производится на разных уровнях, в зависимости от обстоятельств и того, что считать \note{элементарным}.
		\item Любая система является частью какой-либо большей системы. Мета-систему тоже можно делить на подсистемы по-разному. Таким образом, \note{внешняя граница системы имеет относительный, условный характер}.
	\end{itemize}}
\question{При каких предположения можно говорить о наличии частей у системы?}{?}
\question{Как определяется граница системы?}{\note{Внешняя граница системы имеет относительный, условный характер}. Определение границ системы производится с учетом целей субъекта.}
\question{Что называется моделью структуры системы? В чем трудности ее построения?}{Перечень \note{существенных связей между элементами системы} называется \note{моделью структуры системы}.
	\begin{itemize}
		\item Модель структуры определяется после выбора модели состава и зависит от нее. При этом, модель структуры вариабельна даже при зафиксированном составе из-за возможности по-разному определить существенность связей.
		\item Каждый элемент системы представляет собой <<черный ящик>>, соответственно могут привнести ошибки, полученные при определении своих входов и выходов.
	\end{itemize}}	
\question{Что такое динамические свойства систем? Перечислите их (все четыре).}{\define{Динамические свойства}{особенности изменений со временем внутри системы и вне ее.}
	\begin{itemize}
		\item Функциональность
		\item Стимулируемость
		\item Изменчивость системы со временем
		\item Существование в изменяющейся среде
	\end{itemize}}
\question{Поясните различие между ростом и развитием системы.}{Рост подразумевает наращивание состава системы, увеличение в размерах и численности. Развитие --- существенно изменение свойств системы в позитивном направлении. Рост системы может быть ограничен недостатком ресурсов, тогда как развитие является результатом обучения и требует внутреннего желания}
\question{Что мы называем синтетическими свойствами систем? Перечислите четыре таких свойства.}{Синтетическими называют обобщающие, собирательные, интегральные свойства, делающие упор на взаимодействия системы со средой, на целостность.
	\begin{itemize}
		\item \define{Эмерджентность}{объединение частей в систему, порождающее качественно новые свойства, не сводящиеся к свойствам частей, не выводящиеся из свойств частей, присущие только самой системе и существующие только пока система составляет одно целое}
		\item \important{Неразделимость на части}
		\item \define{Ингерентность}{согласованность, приспособленность к окружающей среде в рамках некой функции}
		\item \define{Целесообразность}{подчиненность всего поставленной цели}
	\end{itemize}}
\question{Какое из статических свойств системы обеспечивает существование эмерджентных свойства системы?}{?}
\question{Что называется субъективной целью системы?}{Субъективной целью называют цель, представляемую конструктором системы, ограниченная только желаниями и фантазией субъекта.}
\question{Почему не любая субъективная цель достижима?}{Достижение субъективной цели может быть невозможно в том случае, если данная цель находится за пределами законов природы, т.е. не принадлежит к объективным целям.}
\question{Дайте определение следующим понятиям:}{
	\begin{itemize}
		\item \define{Целостность системы}{существование системы как чего-то единого, целого, обособленного, отличающегося от всего остального; факт внешней различимости в среде.}
		\item \define{Открытость системы}{связность системы и окружающей среды, обмен между ними любыми видами ресурсов.}
		\item \define{Черный ящик}{совокупность входов и выходов системы}
		\item \define{Ошибка первого рода}{оценка связи как существенной и включение ее в модель, хотя на самом деле она несущественна и могла бы быть неучитываемой.}
		\item \define{Ошибка второго рода}{оценка связь как несущественной и не включение ее в модель.}
		\item \define{Ошибкой третьего рода}{незнание о существовании какой-либо связи. Если связь существенна, испытываемые трудности будут соответствовать ошибке второго рода. Ошибку третьего рода труднее исправить: необходимо добывать новые знания.}
		\item \define{Ошибка четвертого рода}{неверное отнесение известной и признанной существенной связи к числу входов или выходов.}
		\item \define{Модель состава системы}{иерархический список частей системы.}
		\item \define{Подсистема}{система, являющаяся частью более общей системы.}
		\item \define{Элемент системы}{часть системы, считаемая простейшей и неделимой, в рамках заданной декомпозиции.}
		\item \define{Модель структуры системы}{перечень существенных связей между элементами системы.}
		\item \define{Функция системы}{процесс на выходе системы.}
		\item \define{Стимулируемость}{подверженность системы воздействиям извне и изменение ее поведения под этими воздействиями.}
		\item \define{Функционирование}{тип динамики системы, при котором изменения не затрагивают структуру системы: одни элементы заменяются другими, эквивалентными; параметры могут меняться без изменения структуры.}
		\item \define{Рост}{наращивание состава системы}
		\item \define{Спад}{уменьшение состава системы}
		\item \define{Развитие}{качественное улучшение существенных свойств}
		\item \define{Деградация}{качественное ухудшение существенных свойств}	
		\item \define{Жизненный цикл}{последовательность периодов спадов и подъемов в истории системы}
		\item \define{Эмерджентность}{объединение частей в систему, порождающее качественно новые свойства, не сводящиеся к свойствам частей, не выводящиеся из свойств частей, присущие только самой системе и существующие только пока система составляет одно целое.}
		\item \define{Ингерентность}{согласованность, приспособленность к окружающей среде в рамках некой функции}
		\item \define{Цель субъективная}{цель, представляемая конструктором системы, ограниченная только желаниями и фантазией субъекта.}
		\item \define{Цель объективная}{цель, представленная как реальное будущее состояние, существующая в рамках законов природы.}
	\end{itemize}}
\subsection{Модели и моделирование}
\question{Покажите, что познавательная и преобразовательня деятельности субъекта невозможны без моделирования}{Познавательная деятельность требует наличие желаемого результата, т.е. модели того, чего нет, и описания алгоритма планируемых для достижения действий.\\Познавательная деятельность представляет из себя составление модели окружающего мира.}
\question{Опишите алгоритм анализа и перечислите, какие модели он порождает}{Процедура анализа состоит в последовательном выполнении следующих операций:
	\begin{itemize}
		\item Сложно расчленить на более мелкие, предположительно более простые, части.
		\item Дать объяснение полученным фрагментам.
		\item Объединить объяснение частей в объяснение целого.
	\end{itemize}
Если какая-то часть системы остается непонятной, операция декомпозиции повторяется.\\Продуктами анализа являются модель состава и структуры системы, модель черного ящика для каждого элемента системы.}
\question{Опишите алгоритм синтеза и укажите, какие модели он порождает. Какая из них непосредственно описывает исследуемый объект (явление)?}{Процедура синтеза состоит в последовательном выполнении следующих операций:
	\begin{itemize}
		\item Выделение метасистемы.
		\item Рассмотрение состава и структуры метасистемы.
		\item Объяснение роли искомой системы в метасистеме через ее связи с другими подсистемами.
	\end{itemize}
Продуктами синтеза являются модели состава и структуры метасистемы, модель черного ящика системы.}
\question{Что такое модель?}{
	\begin{itemize}
		\item Модель есть средство осуществления любой деятельности субъекта.
		\item Модель есть форма существования знаний.
		\item Модель есть системное отображение оригинала.
	\end{itemize}}
\question{Что такое <<абстрактная модель>>? Кроме языковых, какие еще примеры абстрактных моделей вы можете привести?}{Абстрактная модель --- модель, представляющая собой инструмент мышления; её компонентами являются понятия, а не физические элементы. В качестве абстрактных моделей могут выступать схемы, изображения, диаграммы, макеты, алгоритмы, математические описания, etc.}
\question{Чем вызвано многообразие языков?}{Любая групповая деятельность требует большей точности и определенности терминов, по сравнению с разговорными языками, что приводит к появлению языков профессиональных, снимающих неопределенность за счет дополнительной информации. Поскольку, большая часть различных групповых активностей между собой не пересекается, нет ничего удивительного в том, что разные группы требуют разной дополнительной информации от языка, что и способствует существованию различных профессиональных языков, актуальных только для небольшой, конкретной сферы деятельности.}
\question{Какова простейшая абстрактная модель разнообразия окружающей нас реальности?}{Классификация представляет собой простейшую абстрактную модель разнообразия окружающей реальности, разделяя все объекты на классы по каким-либо признакам.}
\question{Чем отличается искусственная и естественная классификации?}{Искусственная классификация подразумевает разделение исходя из поставленной цели --- на столько классов и с такими границами, как это диктуется целью. Естественная классификация подразумевает деление на основе существующих природных группировок/кластеров.}
\question{Что называется <<реальной моделью>>? Приведите три типа реальных моделей (классификацию по происхождения подобия модели оригиналу).}{Реальной моделью может выступать какой-либо материальный предмет, используемый в качестве модели. Классификация по происхождению подобия между оригиналом и моделью, приводит к трем типам реальных моделей:
	\begin{itemize}
		\item Модели \note{прямого подобия} созданы с помощью либо непосредственного взаимодействия (следы, печать), либо цепочки таких взаимодействий (фотография, макет здания).
		\item Модели \note{косвенного подобия} или аналогии. Похожесть, аналогичность двух явлений объясняется совпадением закономерностей, которым они подчиняются. Моделями косвенного подобия являются: аналоговые ЭВМ, исторические параллели, подопытные животные в медицине.\\Следует осторожно пользоваться аналогиями, поскольку, кроме совпадающих закономерностей, у разных явлений есть и несовпадающие.
		\item Модели \note{условного подобия}. Соответствие такой модели и оригинала устанавливается в результате соглашения между ее пользователями и носит условный характер. Они успешно работают до тех пор, пока известны и соблюдаются договоры об их значении. Примеры: деньги --- модели стоимости, буквы --- модели звуков.
	\end{itemize}}
\question{Чем отличается использование познавательных и прагматических моделей?}{Познавательные модели нацелены на получение информации о материальном мире, потому подвержены изменениям вследствие получения новых знаний и терпимы к отличающимся и противоречивым мнениям. Для познавательных моделей, понятия истинности и адекватности выступают синонимами.\\Прагматические модели обслуживают процессы преобразования реальности в соответствии с целями субъекта, а потому имеют директивный, единственно верный характер. В качестве примера можно рассмотреть религию или мораль. Кроме того, адекватная прагматическая модель не обязательно должно быть истинной --- ложь так же может выступать инструментом достижения цели.}
\question{Почему в любой модели есть, кроме истинного, и (обязательно и неизбежно) неистинное содержание?}{Истинное содержание является чем-то общим и для модели, и для оригинала, благодаря чему модель может служить заменителем оригинала. В то же время, т.к. модель так же является системой, она обязательно будет обладать каким-либо собственными свойствами --- неистинным содержанием, которое не будет иметь отношения к оригиналу.}
\question{Какое качество модели называется адекватностью?}{Адекватностью называют степень успешности модели в достижении цели субъекта. Для познавательных моделей, адекватность и истинность являются синонимами. Для прагматических моделей, ложные модели могут быть адекватными.}
\question{Что является окружающей средой для модели?}{Окружающей средой для искомой модели выступает культура (мир моделей) пользователя.}
\question{Дайте определения следующих терминов:}{
	\begin{itemize}
		\item \define{Анализ}{разложение системы на составные, более простые, части с целью объяснения целого на основе объяснения частей.}
		\item \define{Синтез}{рассмотрение состава и структуры метасистемы, с целью объяснения роли системы на основе ее связей с подсистемами метасистемы.}
		\item \define{Модель абстрактная}{модель, представляющая собой инструмент мышления; её компонентами являются понятия, а не физические элементы.}
		\item \define{Модель языковая}{слово, представляющее собой какое-либо понятие.}
		\item \define{Модель реальная}{материальный предмет, используемый в качестве модели.}
		\item \define{Классификация искусственная}{разделение на классы на основе поставленной цели. Количество и границы классов задаются целью.}
		\item \define{Классификация естественная}{разделение на классы на основе существующих природных группировок.}
		\item \define{Модели познавательные}{модели, обслуживающие процессы получения о внешнем мире.}
		\item \define{Модели прагматические}{модели, обслуживающие процессы преобразования реальности в соответствии с целями субъекта.}
		\item \define{Адекватность модели}{степень успешности модели в достижении цели.}
		\item \define{Культура (субъекта, организации, нации --- любой социальной системы)}{мир моделей, сформировавшийся в процессе жизнедеятельности системы.}
	\end{itemize}}
\end{document}
