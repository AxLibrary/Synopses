\documentclass{article}

\usepackage[T2A]{fontenc}
\usepackage[utf8]{inputenc}
\usepackage[english,russian]{babel}
\usepackage[left=2cm,right=2cm,top=2cm,bottom=2cm,bindingoffset=0cm]{geometry}
\usepackage{sectsty}
\allsectionsfont{\centering}

\newcommand{\important}[1]{\textbf{#1}}

\begin{document}
\title{Ф.П. Тарасенко --- Прикладной системный анализ\\(конспект книги)}
\author{Tass}
\date{2017/10/10}
\maketitle

\tableofcontents
\newpage

\section*{\flushright Введение.\\Как возник системный анализ}

Почему накопление и обобщение опыта решения проблем началось (и продолжается) в рамках каждой отдельной профессии?
\\
Для решения проблем могут потребоваться различные профессиональные знания. Это создаёт впечатление, что проблемы различных специальностей уникальны, что и приводит к накоплению соответствующего опыта в рамках отдельных профессий.
\\
Почему, несмотря на громадное разнообразие проблем, технология (совокупность приёмов) их решения практически одинакова в случае успеха и различается в случае неудач?
\\
Идея универсального алгоритма действий по решению проблем опирается на понятие всеобщей системности, олицетворяющей единство и общность законов мироздания.
\\
Сформулируйте основные отличия прикладного системного анализа от традиционных наук.
\\
Прикладной системный анализ нацелен на решение конкретной проблемы, а не на поиск общих закономерностей; для решения проблемы могут понадобиться знания любой из традиционных наук.
\\
Почему прикладной системный анализ можно назвать над-дисциплинарной и меж-дисциплинарной область деятельности как в теоретической, так и в практической его сфере?
\\
В теоретической сфере, прикладной системный анализ может требовать использования знаний из различных областей традиционных наук. В практической деятельности, аналитик направляет коллектив участников ситуации, являющихся специалистами в требуемой области знания.
\end{document}