\documentclass{article}
% Кодировка, поддержка русского языка
\usepackage[T2A]{fontenc}
\usepackage[utf8]{inputenc}
\usepackage[english,russian]{babel}
% Отступы от края страницы
\usepackage{geometry}
\geometry{left=4cm}
\geometry{right=4cm}
\geometry{top=2cm}
\geometry{bottom=2cm}
\geometry{bindingoffset=0cm}
% Расстояние между элементами списка, отступ слева
\usepackage{enumitem}
\setlist{nosep,leftmargin=*}

\newcommand{\note}[1]{\textit{#1}}
\newcommand{\important}[1]{\textbf{#1}}	
\renewcommand{\section}[2]{
	\vspace{6em}
	\begin{flushright}
	\Large
	\baselineskip=0.5\baselineskip
	\textbf{#1}
	\\
	\rule[0.5\baselineskip]{\textwidth}{0.15pt}
	\\
	\textbf{#2}
	\end{flushright}
	}
\renewcommand{\subsection}[1]{
	\vspace{2em}
	\begin{flushright}
		\large
		\textbf{#1}
	\end{flushright}
	}
\renewcommand{\title}[2]{
	\begin{center}
		\LARGE
		\baselineskip=0.5\baselineskip
		\textbf{#1}
		\\
		\rule[0.5\baselineskip]{0.7\textwidth}{0.15pt}
		\\
		\textbf{#2}
		\\\baselineskip=2\baselineskip(конспект)		
	\end{center}
	}
\newcommand{\define}[2]{
	\textbf{#1} --- #2
	}
\newcommand{\marked}[2]{
	\begin{flushright}\textbf{!}\hspace{2ex}\vline\hspace{2ex}
		\begin{minipage}{0.9\textwidth}
			\define{#1}{#2}
		\end{minipage}
	\end{flushright}
	}
\newcommand{\question}[2]{
	\begin{flushright}
		Q:\hspace{2ex}\vline\hspace{2ex}
		\begin{minipage}{0.9\textwidth}
			\large
			\textit{#1}
		\end{minipage}
	\end{flushright}
	\begin{center}
		\begin{minipage}{0.95\textwidth}
			#2
		\end{minipage}
	\end{center}
	}
\begin{document}
\title{Ф.П. Тарасенко}{Прикладной системный анализ}
\section{Введение}{Как возник системный анализ}
Любая деятельность человека состоит в решении постоянно возникающих \note{проблем}. Для их решения могут потребоваться самые различные специфические знания. Это создало впечатление, что проблемы различных специальностей уникальны, что привело к накоплению и обобщению опыта решения проблем в рамках каждой профессии отдельно.\\
Было замечено, что вероятность успеха повышается, если следовать одним и тем же советам, независимо от природы проблемы. Эта идея опирается на понятие всеобщей системности, олицетворяющей единство и общность законов мироздания.\\
Идея формирования общеупотребительной методики решения проблем была доведена до создания специальной технологии --- \important{прикладного системного анализа}. Данная дисциплина нацелена на решение конкретной проблемы, а не на поиск общих закономерностей. В теоретической сфере, прикладной системный анализ может требовать знаний из различных областей традиционных наук. В практической деятельности аналитик направляет коллектив участников ситуации, являющихся специалистами в требуемой области.
\section{Глава 1}{Проблема и способы ее решения}
\marked{Проблемная ситуация}{некоторое реальное стечение обстоятельств, положений вещей, которым кто-то недоволен и хотел бы изменить.}
\marked{Проблема}{Субъективное отрицательное отношение субъекта к реальности}
В понятиях проблемы и проблемной ситуации неразрывно связаны два аспекта --- объективный (наличие реальной ситуации) и субъективной (негативная оценка реальности субъектом). Отличие понятий в том, на чем делается акцент.
\marked{Решение проблемы}{какие-либо действия, призванные уменьшить или совсем снять недовольство субъекта.}
Все способы решения проблем можно разделить на две группы:
\begin{itemize}
	\item Воздействовать на субъект с целью уменьшить его недовольство, не изменяя реальности.
	\item Изменить реальность так, чтобы недовольство субъекта ослабло.
\end{itemize}
\subsection{Способы влияния на субъект}
Существует три возможности изменить к лучшему отношение субъекта к реальности, не изменяя самой реальности:
\begin{itemize}
	\item Сообщение субъекту дополнительной информации о ситуации, среди которой может оказаться информация позитивного характера. Может осуществляться в виде \note{обучения} субъекта.\\Дополнительная информация не обязательно должна быть правдивой, но обязательно должна быть положительной.\\Возможно сокрытие правды, либо отфильтрованная полуправда.
	\item Изменение восприятия данной реальности субъектом. Формы воздействия могут отличаться: психические (гипноз, пропаганда, реклама), физические (воздействие акустических, электрических, магнитных полей), химические (психотропные медикаменты, наркотики, алкоголь).
	\item Прекращение взаимодействия субъекта с ситуацией. Субъект может быть повышен, направлен в отпуск, переведен в другой отдел, уволен.
\end{itemize}
\subsection{Вмешательство в реальность}
В реальной ситуации участвуют не только недовольный субъект, но и многие другие субъекты, которые оценивают эту же ситуацию со своих позиций. Всякое изменение ситуации в результате вмешательства, будет замечено и оценено всеми ее участниками.\\
Субъект существует в реальной физической среде и подвержен её воздействиям. В отличии от объекта, субъект не только подчинён природным закономерностям, но и наделен способностью \note{оценивать} свои взаимодействия со средой: что-то ему может нравиться, а может и нет. Таким образом, все оценки имеют сугубо субъективный, индивидуальный характер.
\note{Всякий раз, когда в вашем присутствии прозвучит любое оценочное слово} (хорошо --- плохо, полезно --- вредно и т.п.), \note{насторожитесь и задайте вопрос <<В каком смысле?>>}. Оценки не бывают объективны и если вы хотите понять истинный смысл сказанного, надо выяснить, какие критерии применяет оценивающий.
\subsection{Три типа идеологий вмешательства}
\note{Правильным} считается поведение, максимально согласующееся с принятой субъектом идеологией. Именно идеология и определяет, что плохо, а что хорошо. Идеологии могут быть различными, отличаясь определением того, \note{какое отношение к другим субъектам считать правильным}. Явные различия можно провести между тремя типами идеологий:
\begin{itemize}
	\item Первый тип идеологии --- <<принцип приоритета меньшинства>>. Этот принцип приводит к тому, чтобы осуществить вмешательство, угодное клиенту, а интересы других участников не принимаются во внимание. Жизненные примеры реализации --- диктатура, иерархическая организация, эгоизм, и т.д.
	\item Второй тип идеологии --- <<принцип приоритета группы>>. Согласно ей среди участников ситуации, кроме клиента, есть другие субъекты, не менее важные и ценные. Вмешательство должно проводиться с учётом интересов всех <<наших>>. Примеры --- расизм, национализм, фашизм, коммунизм, и т.д.
	\item Третий тип --- <<принцип приоритета каждого>>. В основе лежит два постулата: нет ни одного одинакового субъекта; все субъекты равноценны и равноправны. Правильным, моральным признается только улучшающее вмешательство.
\end{itemize}
\marked{Улучшающее вмешательство}{изменение проблемной ситуации, которое положительно оценивается хотя бы одним из ее участников и неотрицательно --- всеми остальными.}
\marked{Прикладной системный анализ}{теория и практика проектирования и реализации улучшающих вмешательств. Методика решения проблем реальной жизни без создания новых проблем.}
\subsection{Четыре типа вмешательств}
Выделяют четыре типа способов решения проблем:
\begin{itemize}
	\item \important{ABSOLUTION} --- \note{невмешательство} в расчёте на то, что естественный ход событий приведёт к разрешению проблемы. Невмешательство обладает одним из признаков улучшающего вмешательства: при этом никому не становится хуже. Чтобы стать <<улучшающим вмешательством>>, необходимо, чтобы события действительно вели к разрешению проблемы, а выбор подобного поведения может быть обусловлен тем, что любые предлагаемые вмешательства приводят к худшим результатам. Примеры: поведения врача при невозможности исцелению пациента, действия сапёра при встрече с незнакомым взрывным устройством.
	\item \important{RESOLUTION} --- \note{частичное вмешательство}, снижающее неудовлетворённость, ослабляющее остроту проблемы, но не устраняющее ее полностью. Обычно применяется при дефиците ресурсов. Примеры: распределение по жребию или очереди.
	\item \important{SOLUTION} --- \note{оптимальное решение}, наилучшее в данных условиях. 
	\item \important{DISSOLUTION} --- вмешательство, заканчивающееся полным исчезновением проблемы и непоявлением новых проблем. Условия и ограничения рассматриваются не как незыблемо фиксированные, а как подлежащие изменениям с целью поиска новых, недопустимых ранее вариантов, среди которых могут оказаться гораздо более эффективные, чем ранее оптимальные.	
\end{itemize}
\marked{Оптимальность}{сочетание наилучших параметров по заданным \note{критериям} в рамках существующих \note{условий} и ограничений.}
\subsection{Еще о прикладном системном анализе}
Процесс решения проблемы не может быть выполнен лишь самим системным аналитиком. Построение модели проблемной ситуации требует информации, которой обладают сами ее участники. Их вовлечение в процесс работы над проблемой является обязательным элементом технологии. Более того, воплощать разработанное вмешательство так же будут они.\\
Выполнение какой-либо работы собственными усилиями является самой эффективной формой обучения. Таким образом, в прикладном системно анализе оказывается естественно встроенным обучение самому системному анализу.
\section{Глава 2}{Понятие системы}
\subsection{Статические свойства системы}
\note{Статическими свойствами} называются особенности конкретного состояния системы.
\marked{Целостность}{первое свойство}
\define{Целостность}{существование системы как чего-то единого, целого, обособленного, отличающегося от всего остального; факт внешней различимости в среде.}
\marked{Открытость}{второе свойство}
\define{Открытость}{связность системы и окружающей среды, обмен между ними любыми видами ресурсов.}\\
Связи системы со средой имеют направленный характер: по одним среда влияет на систему (\note{входы}), по другим система оказывает влияние на среду (\note{выходы}). Перечень входов и выходов системы называют \note{моделью черного ящика}. В этой модели отсутствует информация о внутренних особенностях системы.\\
В модели должны быть отражены все связи, существенные для достижения цели. Оценку важности той или иной связи может дать только субъект. Возможны четыре типа ошибок при построении модели.
\begin{itemize}
	\item \note{Ошибка первого рода} происходит, когда субъект расценивает связь как существенную и принимает решение о включении ее в модель, хотя на самом деле она несущественна и могла бы быть неучитываемой.
	\item \note{Ошибка второго рода} происходит, когда субъект принимает решение, что данная связь несущественна и не должна быть включена в модель.
	\item \note{Ошибкой третьего рода} принято считать последствия незнания о существовании какой-либо связи. Если связь существенна, испытываемые трудности будут соответствовать ошибке второго рода. Ошибку третьего рода труднее исправить: необходимо добывать новые знания.
	\item \note{Ошибка четвертого рода} может возникнуть при неверном отнесении известной и признанной существенной связи к числу входов или выходов. 
\end{itemize}
Следствием открытости систем является очевидность \note{всеобщей взаимосвязи и взаимозависимости в природе}. Между любыми двумя системами обязательно существует, и ее можно отыскать, длинная или короткая цепочка систем, связывающая их: выход каждой системы является входом другой.
\marked{Внутренняя неоднородность}{третье свойство}
Описание внутренней неоднородности системы сводится к обособлению относительно однородных участков, проведения границ между ними, выделения \note{частей системы}. Выделенные крупные части тоже не однородны, что позволяет выделить еще более мелкие части и получить иерархический список частей системы --- \note{модель состава системы}.\\
Трудности построения модели состава можно представить тремя положениями:
\begin{itemize}
	\item Целое можно делить на части по-разному, в зависимости от того, что требуется для достижения цели. При этом, можно \note{различать} нужные для цели части, но не следует \note{разделять их}.
	\item Количество частей в модели состава зависит от того, на каком уровне остановить дробление системы. Части на конечных ветвях получающегося иерархического дерева называют \note{элементами}. Прекращение декомпозиции производится на разных уровнях, в зависимости от обстоятельств и того, что считать \note{элементарным}.
	\item Любая система является частью какой-либо большей системы. Мета-систему тоже можно делить на подсистемы по-разному. Таким образом, \note{внешняя граница системы имеет относительный, условный характер}. Определение границ системы производится с учетом целей субъекта.
\end{itemize}
\marked{Структурированность}{четвертое свойство}
Части системы не независимы и не изолированы друг с другом, они связаны и взаимодействуют между собой. При этом свойства системы в целом существенно зависят от того, как именно взаимодействуют ее части. Перечень \note{существенных связей между элементами системы} называется \note{моделью структуры системы}.
\begin{itemize}
	\item Модель структуры определяется после выбора модели состава и зависит от нее. При этом, модель структуры вариабельна даже при зафиксированном составе из-за возможности по-разному определить существенность связей.
	\item Каждый элемент системы представляет собой <<черный ящик>>, соответственно могут привнести ошибки, полученные при определении своих входов и выходов.
\end{itemize}
\subsection{Динамические свойства системы}
Динамическими свойствами именуются особенности изменений со временем внутри системы и вне ее. О любых изменениях мы имеем возможность говорить в терминах перемен в статических моделях системы.
\marked{Функциональность}{пятое свойство}
Процессы на выходах системы, рассматриваются как ее \note{функции}.\define{Функции системы}{это ее поведение во внешней среде; изменения, производимые системой в окружающей среде; результаты ее деятельности.} Из множественности выходов следует множественность функций. Субъект, использующий систему, будет \note{оценивать} ее функции и \note{упорядочивать} их по отношению к своим потребностям.
\marked{Стимулируемость}{шестое свойство}
Процессы на входах системы называются \note{стимулами}.\define{Стимулируемость}{подверженность системы воздействиям извне и изменение ее поведения под этими воздействиями.}
\marked{Изменчивость системы со временем}{изменение значений внутренних параметров, состава и структуры системы и любых их комбинаций в .}
\note{Функционированием} называется тип динамики системы, при котором изменения не затрагивают структуру системы: одни элементы заменяются другими, эквивалентными; параметры могут меняться без изменения структуры.\\
Изменения могут носить преимущественно \note{количественный} характер. Наращивание состава системы называют \note{ростом}. Обратные изменения называют \note{спадом}.\\
При \note{качественных} изменения системы происходит изменение ее существенных свойств. Если такие изменения идут в позитивном направлении, они называются \note{развитием}. Утрату или ослабление полезных свойств именуют \note{деградацией}. Развитая система добивается более высоких результатов с теми же ресурсами.\\
Рост есть увеличение в размерах и численности; развитие --- увеличение компетентности, т.е. знаний. Объемность --- результат роста, компетентность --- развития. Недостаток материальных ресурсов может ограничить рост, но не развитие; развитие извне не ограничено.\\
Существует внутреннее ограничение на развитие. Развитие есть результат \note{обучения}, но обучение нельзя осуществить для и вместо обучаемого. Таким образом, \note{развитие возможно только как саморазвитие}.\\
Монотонные изменения не могут длиться вечно, в истории системы можно усмотреть периоды спада и подъема, последовательность которых образует индивидуальный \note{жизненный цикл} системы. При построении описания жизненного цикла, особое внимание необходимо обратить на непрерывность его траектории, так как история проектируемой системы закончится на первом же пробеле в описании ее жизненного цикла.
\marked{Существование в изменяющейся среде}{восьмое свойство}
Изменяется не только сама система, но и окружающая ее среда. Необходимость существовать в подобном окружении имеет множество последствий для самой системы, начиная с необходимости ее приспособления к внешним переменам, до различных других реакций системы.\\
Чем сильнее внешние изменения, тем активнее должны проводиться внутренние. И хотя важными средствами остаются прогнозирование и обучение, более эффективными считаются выработка иммунитета к неподконтрольным изменениям и усиление контроля над остальными.


\question{Что мы называем синтетическими свойствами систем? Перечислите четыре таких свойства.}{?}
\question{Какое из статически свойств системы обеспечивает существование эмерджентных свойства системы?}{?}
\question{Что называется субъективной целью системы?}{?}
\question{Почему не любая субъективная цель достижима?}{?}
\question{Дайте определение следующим понятиям:}{
\begin{itemize}
	\item \define{целостность системы}{?}
	\item \define{открытость системы}{?}
	\item \define{черный ящик}{?}
	\item \define{ошибка первого (второго, третьего, четвертого) рода}{?}
	\item \define{модель состава системы}{?}
	\item \define{подсистема}{?}
	\item \define{элемент системы}{?}
	\item \define{модель структуры системы}{?}
	\item \define{функция системы}{?}
	\item \define{стимулируемость систем}{?}
	\item \define{функционирование}{?}
	\item \define{рост (спад)}{?}
	\item \define{развитие (деградация)}{?}
	\item \define{жизненный цикл}{?}
	\item \define{эмерджентность}{?}
	\item \define{ингерентность}{?}
	\item \define{цель субъективная}{?}
	\item \define{цель объективная}{?}
\end{itemize}}

\newpage
\section{Приложение}{Вопросы для Anki}
\subsection{Введение}
\question{Почему накопление и обобщение опыта решения проблем началось (и продолжается) в рамках каждой отдельной профессии?}{Для решения проблем могут потребоваться различные профессиональные знания. Это создаёт впечатление, что проблемы различных специальностей уникальны, что и приводит к накоплению соответствующего опыта в рамках отдельных профессий.}
\question{Почему, несмотря на громадное разнообразие проблем, технология (совокупность приёмов) их решения практически одинакова в случае успеха и различается в случае неудач?}{Идея универсального алгоритма действий по решению проблем опирается на понятие всеобщей системности, олицетворяющей единство и общность законов мироздания.}
\question{Сформулируйте основные отличия прикладного системного анализа от традиционных наук.}{Прикладной системный анализ нацелен на решение конкретной проблемы, а не на поиск общих закономерностей; для решения проблемы могут понадобиться знания любой из традиционных наук.}
\question{Почему прикладной системный анализ можно назвать над-дисциплинарной и меж-дисциплинарной область деятельности как в теоретической, так и в практической его сфере?}{В теоретической сфере, прикладной системный анализ может требовать использования знаний из различных областей традиционных наук. В практической деятельности, аналитик направляет коллектив участников ситуации, являющихся специалистами в требуемой области знания.}
\subsection{Проблема и способы ее решения}
\question{Поясните различия между понятиями "проблемная ситуация" и "проблема"}{Термин \note{"проблемная ситуация"} выделяет объективный компонент, реальную ситуацию; \note{"проблема"} --- субъективное отношение к реальной ситуации субъекта.}
\question{Что значит "решить проблему"?}{Решение проблемы объединяет в себе какие-либо действия, призванные уменьшить или совсем снять недовольство субъекта.}
\question{Какие три способа воздействия на субъект без изменения реальности могут (при определённых условиях) привести к решению его проблемы? Каковы эти условия?}{
	\begin{itemize}
		\item Сообщение субъекту дополнительной информации о ситуации, среди которой может оказаться информация позитивного характера. Может осуществляться в виде \note{обучения} субъекта.\\Дополнительная информация не обязательно должна быть правдивой, но обязательно должна быть положительной.\\Возможно сокрытие правды, либо отфильтрованная полуправда.
		\item Изменение восприятия данной реальности субъектом. Формы воздействия могут отличаться: психические (гипноз, пропаганда, реклама), физические (воздействие акустических, электрических, магнитных полей), химические (психотропные медикаменты, наркотики, алкоголь).
		\item Прерывание взаимодействия субъекта с ситуацией. Субъект может быть повышен, направлен в отпуск, переведён в другой отдел, уволен.
	\end{itemize}}
\question{Каково основное отличие субъекта от объекта?}{Субъект существует в реальной физической среде и подвержен её воздействиям. В отличие от объекта, субъект не только подчинён природным закономерностям, но и наделен способностью \note{оценивать} свои взаимодействия со средой: что-то ему может нравиться, а может и не нравиться. Таким образом, все оценки имеют сугубо субъективный, индивидуальный характер.}
\question{Как определить смысл оценки, выраженной неким субъектом?}{\note{Всякий раз, когда в вашем присутствии прозвучит любое оценочное слово} (хорошо --- плохо, полезно --- вредно и т.п.), \note{насторожитесь и задайте вопрос <<В каком смысле?>>}. Оценки не бывают объективны и если вы хотите понять истинный смысл сказанного, надо выяснить, какие критерии применяет оценивающий.}
\question{Почему при вмешательстве в реальность с целью решения проблемы приходится опираться на какую-то идеологию?}{\note{Правильным} считается поведение, максимально согласующееся с принятой субъектом идеологией. Именно идеология и определяет, что плохо, а что хорошо.}
\question{Воспроизведите классификацию идеологий на три типа. Каково основное отличие между ними?}{Основное отличие между идеологиями --- это определение того, \note{какое отношение к другим субъектам считать правильным}.
	\begin{itemize}
		\item Первый тип идеологии --- <<принцип приоритета меньшинства>>. Этот принцип приводит к тому, чтобы осуществить вмешательство, угодное клиенту, а интересы других участников не принимаются во внимание. Жизненные примеры реализации --- диктатура, иерархическая организация, эгоизм, и т.д.
		\item Второй тип идеологии --- <<принцип приоритета группы>>. Согласно ей среди участников ситуации, кроме клиента, есть другие субъекты, не менее важные и ценные. Вмешательство должно проводиться с учётом интересов всех <<наших>>. Примеры --- расизм, национализм, фашизм, коммунизм, и т.д.
		\item Третий тип --- <<принцип приоритета каждого>>. В основе лежит два постулата: нет ни одного одинакового субъекта; все субъекты равноценны и равноправны. Правильным, моральным признается только улучшающее вмешательство.
	\end{itemize}}
\question{Назовите четыре типа улучшающих вмешательств.}{
	\begin{itemize}
		\item \important{ABSOLUTION} --- \note{невмешательство} в расчёте на то, что естественный ход событий приведёт к разрешению проблемы. Невмешательство обладает одним из признаков улучшающего вмешательства: при этом никому не становится хуже. Чтобы стать <<улучшающим вмешательством>>, необходимо, чтобы события действительно вели к разрешению проблемы, а выбор подобного поведения может быть обусловлен тем, что любые предлагаемые вмешательства приводят к худшим результатам. Примеры: поведения врача при невозможности исцелению пациента, действия сапёра при встрече с незнакомым взрывным устройством.
		\item \important{RESOLUTION} --- \note{частичное вмешательство}, снижающее неудовлетворённость, ослабляющее остроту проблемы, но не устраняющее ее полностью. Обычно применяется при дефиците ресурсов. Примеры: распределение по жребию или очереди.
		\item \important{SOLUTION} --- \note{оптимальное решение}, наилучшее в данных условиях. 
		\item \important{DISSOLUTION} --- вмешательство, заканчивающееся полным исчезновением проблемы и непоявлением новых проблем. Условия и ограничения рассматриваются не как незыблемо фиксированные, а как подлежащие изменениям с целью поиска новых, недопустимых ранее вариантов, среди которых могут оказаться гораздо более эффективные, чем ранее оптимальные.
	\end{itemize}}
\question{Оптимальность обеспечивается только при совокупном соблюдении двух требований. Каковы эти требования?}{\note{Оптимальный} значит \note{<<наилучший в данных условиях>>}. Первым требованием является определение, по какому критерию или критериям будут упорядочиваться сравниваемые варианты, т.е. в каком смысле мы будем употреблять термин <<наилучший>>. Вторым требованием является определение, в рамкам каких ограничений будут выбираться сравниваемые варианты.}
\question{Каков важный результат прикладного системного анализа конкретной проблемы, кроме решения самой проблемы?}{Выполнять процесс системного анализа будут сами работники фирмы клиента, т.к. для построения модели проблемной ситуации необходима информация, которой обладают только ее участники; воплощать разработанное вмешательство так же будут именно они. В выполнение какой-либо работы собственными усилиями является самой эффективной формой обучения этой деятельности. Таким образом, в прикладном системном анализе оказывается естественно встроенным, неотъемлемым элементом обучение самому системному анализу.}
\question{Дайте определение следующим понятиям:}{
	\begin{itemize}
		\item \define{Проблемная ситуация}{некоторое реальное стечение обстоятельств, положение вещей, которым кто-то недоволен, неудовлетворён и хотел бы изменить.}
		\item \define{Оценка}{мнение, суждение субъекта о чем-либо.}
		\item \define{Проблема}{субъективное отрицательное отношение субъекта к реальности.}
		\item \define{Решение проблемы}{какие-либо действия, призванные уменьшить или совсем снять недовольство субъекта.}
		\item \define{Вмешательство}{изменение проблемной ситуации, снижающее недовольство клиента.}
		\item \define{Улучшающее вмешательство}{изменение проблемной ситуации, которое положительно оценивается хотя бы одним из ее участников и неотрицательно --- всеми остальными.}
		\item \define{Прикладной системный анализ}{теория и практика проектирования и реализации улучшающих вмешательств. Методика решения проблем реальной жизни без создания новых проблем.}
		\item \define{Оптимальность}{сочетание наилучших параметров по заданным критериям в рамках существующих ограничений.}
		\item \define{"Твёрдая проблема"}{хорошо структурированная, допускающая построение количественных математических моделей. Например, многие инженерные и научные проблемы.}
		\item \define{"Мягкая проблема"}{неструктурированные, описанные на естественном языке ситуации, не относящиеся к точным наукам.}
	\end{itemize}}
\question{Что такое статические свойства систем? Перечислите четыре статических свойства.}{\note{Статическими свойствами} называются особенности конкретного состояния системы.
	\begin{itemize}
		\item \define{Целостность}{существование системы как чего-то единого, целого, обособленного, отличающегося от всего остального; факт внешней различимости в среде.}
		\item \define{Открытость}{связность системы и окружающей среды, обмен между ними любыми видами ресурсов.}
		\item \define{Внутренняя неоднородность}{различимость частей системы.}
		\item \define{Структурированность}{Наделенность любой системы определенной структурой.}
	\end{itemize}}
\question{Как из открытости систем вытекает факт всеобщей взаимосвязанности в природе?}{Следствием открытости систем является очевидность \note{всеобщей взаимосвязи и взаимозависимости в природе}. Между любыми двумя системами обязательно существует, и ее можно отыскать, длинная или короткая цепочка систем, связывающая их: выход каждой системы является входом другой.}
\question{Что называется моделью черного ящика? Назовите четыре рода ошибок, которые можно совершить при построении модели черного ящика.}{Перечень входов и выходов системы называют \note{моделью черного ящика}. В этой модели отсутствует информация о внутренних особенностях системы.
	\begin{itemize}
		\item \note{Ошибка первого рода} происходит, когда субъект расценивает связь как существенную и принимает решение о включении ее в модель, хотя на самом деле она несущественна и могла бы быть неучитываемой.
		\item \note{Ошибка второго рода} происходит, когда субъект принимает решение, что данная связь несущественна и не должна быть включена в модель.
		\item \note{Ошибкой третьего рода} принято считать последствия незнания о существовании какой-либо связи. Если связь существенна, испытываемые трудности будут соответствовать ошибке второго рода. Ошибку третьего рода труднее исправить: необходимо добывать новые знания.
		\item \note{Ошибка четвертого рода} может возникнуть при неверном отнесении известной и признанной существенной связи к числу входов или выходов. 
	\end{itemize}}
\question{Что называется моделью состава системы? Каковы три трудности ее построения?}{\define{Модель состава системы}{иерархический список частей системы.}\\Трудности построения модели состава можно представить тремя положениями:
	\begin{itemize}
		\item Целое можно делить на части по-разному, в зависимости от того, что требуется для достижения цели. При этом, можно \note{различать} нужные для цели части, но не следует \note{разделять их}.
		\item Количество частей в модели состава зависит от того, на каком уровне остановить дробление системы. Части на конечных ветвях получающегося иерархического дерева называют \note{элементами}. Прекращение декомпозиции производится на разных уровнях, в зависимости от обстоятельств и того, что считать \note{элементарным}.
		\item Любая система является частью какой-либо большей системы. Мета-систему тоже можно делить на подсистемы по-разному. Таким образом, \note{внешняя граница системы имеет относительный, условный характер}.
	\end{itemize}}
\question{При каких предположения можно говорить о наличии частей у системы?}{?}
\question{Как определяется граница системы?}{\note{Внешняя граница системы имеет относительный, условный характер}. Определение границ системы производится с учетом целей субъекта.}
\question{Что называется моделью структуры системы? В чем трудности ее построения?}{Перечень \note{существенных связей между элементами системы} называется \note{моделью структуры системы}.
	\begin{itemize}
		\item Модель структуры определяется после выбора модели состава и зависит от нее. При этом, модель структуры вариабельна даже при зафиксированном составе из-за возможности по-разному определить существенность связей.
		\item Каждый элемент системы представляет собой <<черный ящик>>, соответственно могут привнести ошибки, полученные при определении своих входов и выходов.
	\end{itemize}}	
\question{Что такое динамические свойства систем? Перечислите их (все четыре).}{\define{Динамические свойства}{особенности изменений со временем внутри системы и вне ее.}
	\begin{itemize}
		\item Функциональность
		\item Стимулируемость
		\item Изменчивость системы со временем
		\item Существование в изменяющейся среде
	\end{itemize}}
\question{Поясните различие между ростом и развитием системы.}{Рост подразумевает наращивание состава системы, увеличение в размерах и численности. Развитие --- существенно изменение свойств системы в позитивном направлении. Рост системы может быть ограничен недостатком ресурсов, тогда как развитие является результатом обучения и требует внутреннего желания}
\end{document}