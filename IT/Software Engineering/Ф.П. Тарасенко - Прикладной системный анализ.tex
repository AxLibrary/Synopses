\documentclass{article}
% Кодировка, поддержка русского языка
\usepackage[T2A]{fontenc}
\usepackage[utf8]{inputenc}
\usepackage[english,russian]{babel}
% Отступы от края страницы
\usepackage{geometry}
\geometry{left=2cm}
\geometry{right=2cm}
\geometry{top=2cm}
\geometry{bottom=2cm}
\geometry{bindingoffset=0cm}
% Вложенные списки
\usepackage[ampersand]{easylist}
\ListProperties(Hide=100,Hang=true,Progressive=3.5ex,Style*=$\triangleright$ )

\newcommand{\note}[1]{\textit{#1}}
\newcommand{\important}[1]{\textbf{#1}}
\newcommand{\enquote}[1]{,,#1''}
\renewcommand{\section}[2]{
	\vspace{6em}
	\begin{flushright}
	\Large
	\baselineskip=0.5\baselineskip
	\textbf{#1}
	\\
	\rule[0.5\baselineskip]{\textwidth}{0.15pt}
	\\
	\textbf{#2}
	\end{flushright}
	}
\renewcommand{\subsection}[1]{
	\vspace{2em}
	\begin{flushright}
		\large
		\textbf{#1}
	\end{flushright}
	}
\renewcommand{\title}[2]{
	\begin{center}
		\LARGE
		\baselineskip=0.5\baselineskip
		\textbf{#1}
		\\
		\rule[0.5\baselineskip]{0.7\textwidth}{0.15pt}
		\\
		\textbf{#2}
		\\\baselineskip=2\baselineskip(конспект)		
	\end{center}
	}
\newcommand{\define}[2]{
	\textbf{#1} --- #2
	}
\begin{document}
\title{Ф.П. Тарасенко}{Прикладной системный анализ}
\section{Введение}{Как возник системный анализ}
\begin{easylist}
& Деятельность человека состоит в решении \note{проблем}.
& Для решения могут потребоваться различные знания.
&& Создает впечатление, что проблемы различных специальностей уникальны.
&& Накопление опыта решения проблем в рамках каждой профессии отдельно.
& Вероятность успеха повышается, если следовать одним и тем же советам.
&& Не зависит от природы проблемы. 
&& Опирается на всеобщую системность, единство и общность законов мироздания.
& Общеупотребительная методика развита до \important{прикладного системного анализа}.
&& Нацелен на решение конкретной проблемы. 
&& В теоретической сфере, прикладной системный анализ может требовать знаний из различных областей традиционных наук. 
&& В практической деятельности аналитик направляет коллектив участников ситуации, являющихся специалистами в требуемой области.
\end{easylist}
\section{Глава 1}{Проблема и способы ее решения}
\begin{easylist}
& \define{Проблемная ситуация}{некоторое реальное положение вещей, которым кто-то недоволен и хотел бы изменить.}
& \define{Проблема}{Субъективное отрицательное отношение субъекта к реальности}
& В понятиях проблемы и проблемной ситуации неразрывно связаны два аспекта
&& Объективный --- наличие реальной ситуации.
&& Субъективный --- негативная оценка реальности субъектом.
&& Отличие понятий в том, на чем делается акцент.
& \define{Решение проблемы}{какие-либо действия, призванные уменьшить или совсем снять недовольство субъекта.}
& Все способы решения проблем можно разделить на две группы:
&& Воздействовать на субъект с целью уменьшить его недовольство, не изменяя реальности.
&& Изменить реальность так, чтобы недовольство субъекта ослабло.
\end{easylist}
\subsection{Способы влияния на субъект}
\begin{easylist}
& Сообщение субъекту дополнительной информации о ситуации. 
&& Обязательно должна быть положительной.
&& Не обязательно должна быть правдивой.
&& Может осуществляться в виде \note{обучения} субъекта.
&& Возможно сокрытие правды, либо отфильтрованная полуправда.
& Изменение восприятия данной реальности субъектом.
&& Воздействия --- гипноз, наркотики, алкоголь и т.п.
& Прекращение взаимодействия субъекта с ситуацией.
&& Отпуск, перевод в другой отдел, увольнение.
\end{easylist}
\subsection{Вмешательство в реальность}
\begin{easylist}
& В ситуации участвуют не только недовольный субъект, но и другие субъекты.
&& Субъекты оценивают ситуацию со своих позиций.
&& Изменение ситуации в результате вмешательства, будет замечено и оценено ее участниками.
& Субъект существует в реальной физической среде и подвержен её воздействиям.
&& Субъект наделен способностью \note{оценивать} свои взаимодействия со средой.
&& Все оценки имеют субъективный, индивидуальный характер.
& Любая оценка требует уточнения.
&& Оценки не бывают объективны.
&& Необходимо выяснить критерии оценки, чтобы понять смысл.
\end{easylist}
\subsection{Три типа идеологий вмешательства}
\begin{easylist}
& Правильное поведение максимально согласуется с принятой субъектом \note{идеологией}. 
& Идеология пределяет, что плохо, а что хорошо.
&& Отличаются определением \note{какое отношение к другим субъектам считать правильным}. 
& Можно выделить три типа.
&& \note{Принцип приоритета меньшинства.}
&&& Приводит к тому, чтобы осуществить вмешательство, угодное клиенту.
&&& Интересы других участников не принимаются во внимание.
&&& Пример: диктатура, иерархическая организация, эгоизм, и т.д.
&& \note{Принцип приоритета группы.}
&&& Среди участников ситуации, кроме клиента, есть другие ценные субъекты.
&&& Вмешательство должно проводиться с учётом интересов всех \enquote{наших}.
&&& Пример: расизм, национализм, фашизм, коммунизм, и т.д.
&& \note{Принцип приоритета каждого.}
&&& Нет ни одного одинакового субъекта.
&&& Все субъекты равноценны и равноправны.
&&& Правильным, моральным признается только улучшающее вмешательство.
& \define{Улучшающее вмешательство}{изменение проблемной ситуации, которое положительно оценивается хотя бы одним из ее участников и неотрицательно всеми остальными.}
& \define{Прикладной системный анализ}{теория и практика проектирования и реализации улучшающих вмешательств. Методика решения проблем реальной жизни без создания новых проблем.}
\end{easylist}
\subsection{Четыре типа вмешательств}
\begin{easylist}
& \important{ABSOLUTION} --- невмешательство.
&& Расчет на то, что естественный ход событий приведет к разрешению проблемы.
&& Обладает признаком улучшающего вмешательства --- никому не становится хуже.
&& События должны вести к разрешению проблемы
&& Предлагаемые вмешательства приводят к худшим результатам.
&& Пример: поведение врача при невозможности исцелению пациента, действия сапера при встрече с незнакомым взрывным устройством.
& \important{RESOLUTION} --- частичное вмешательство.
&& Снижает неудовлетворенность, ослабляет остроту проблемы.
&& Не устраняет проблему полностью.
&& Обычно применяется при дефиците ресурсов.
&& Примеры: распределение по жребию или очереди.
& \important{SOLUTION} --- оптимальное решение.
&& \define{Оптимальность}{сочетание наилучших параметров по заданным \note{критериям} в рамках существующих \note{условий} и ограничений.}
&& Ограничения могут быть неизвестны из-за нехватки информации о проблеме.
&& Оптимальность может выступать недостижимым идеалом.
& \important{DISSOLUTION} --- полное исчезновение проблемы и не появление новых проблем.
&& Условия и ограничения рассматриваются как подлежащие изменениям.
&& Поиск новых, недопустимых ранее вариантов, более эффективных, чем ранее оптимальные.	
\end{easylist}
\subsection{Еще о прикладном системном анализе}
\begin{easylist}
& Процесс решения проблемы не может быть выполнен лишь самим системным аналитиком.
& Вовлечение участников ситуации является обязательным.
&& Обладают необходимой информацией.
&& Будут воплощать разработанное впешательство.
& Выполнение работы собственными усилиями --- эффективная форма обучения. 
&& В прикладном системном анализе естественно встроенно обучение системному анализу.
\end{easylist}
\section{Глава 2}{Понятие системы}
\subsection{Статические свойства системы}
\begin{easylist}
& \note{Статическими свойствами} называются особенности конкретного состояния системы.
& Выделяют четрые статических свойства
&& Целостность
&& Открытость
&& Внутренняя неоднородность
&& Структурированность
& \define{Целостность}{существование системы как чего-то единого, целого, обособленного, отличающегося от всего остального; факт внешней различимости в среде.}
& \define{Открытость}{связность системы и окружающей среды, обмен между ними любыми видами ресурсов.}
& Связи системы со средой имеют направленный характер.
&& Через \note{выходы} система влияет на среду, через \note{входы} среда влияет на систему.
& \note{Модель черного ящика} --- перечень входов и выходов системы.
&& Содержит конечный список связей, у реальной системы их число не ограничено.
&& Не содержит информации о внутренних особенностях системы.
&& Должна содержать связи, существенные для достижения цели.
&& Оценку важности связи может дать только субъект.
& Субъект может ошибаться при оценке важности. Возможны четыре типа ошибок.
&& \note{Ошибка первого рода}: несущественная связь включена в модель.
&& \note{Ошибка второго рода}: существенная связь не включен в модель.
&& \note{Ошибкой третьего рода}: последствия незнания о существовании связи.
&&& Для существенной связи соответствует ошибке второго рода.
&&& Для исправления необходимы новые знания.
&& \note{Ошибка четвертого рода}: неверная трактовка существенной связи, как входа или выхода.
& Все системы связаны между собой.
&& Между двумя системами всегда существует цепочка связей.
&& Выход одной системы является входом другой.
&& Прямая и обратная цепи, как правило, различны.
& \define{Внутренняя неоднорость}{сводится к выделению \note{частей системы}.}
&& Обособление относительно однородных участков, проведение границ между ними.
&& Выделенные части тоже неоднородны, можно выделить еще более мелкие.
&& \note{Модель состава системы} --- иерархический список частей.
& Построение модели состава сопряжено с трудностями.
&& Целое можно делить на части по-разному.
&&& Зависит от того, что требуется для достижения цели. 
&&& Необходимо \note{различать} нужные части, но не следует \note{разделять их}.
&& Количество частей зависит от того, на каком уровне остановить дробление.
&&& \note{Элементы} --- части на конечных ветвях иерархического дерева.
&&& Уровень декомпозиции зависит от того, что считать \note{элементарным}.
&& Внешняя граница системы имеет относительный, условный характер.
&&& Любая система является частью большей системы.
&&& Мета-систему можно делить на подсистемы по-разному.
&&& Определение границ системы происходит с учетом целей субъекта.
& \define{Структурированность}{наделенность системы определенной структурой.}
& Части системы связаны и взаимодействуют между собой.
&& Свойства системы зависят от взаимодействия частей.
&& \note{Модель структуры системы} --- перечень существенных связей между элементами системы.
&&& Модель структуры определяется после выбора модели состава и зависит от нее.
&&& Модель вариабельна из-за возможности по-разному определить существенность связей.
&&& Элементы могут привнести ошибки определения своих входов и выходов.
\end{easylist}
\subsection{Динамические свойства системы}
\begin{easylist}
& \define{Динамические свойства}{особенности изменений со временем внутри системы и вне ее.}
&& Любые изменения можно рассмотреть как перемены в статических моделях системы.
& Выделяют четыре динамических свойства.
&& Функциональность.
&& Стимулируемость.
&& Изменчивость системы со временем.
&& Существование в изменяющейся среде.
& \define{Функции}{изменения, производимые системой в окружающей среде; результаты ее деятельности.}
&& Процессы на выходах системы рассматриваются как ее функции.
&& Субъект, использующий систему, \note{оценивает} и \note{упорядочивает} функции согласно потребностям.
& \define{Стимулируемость}{подверженность системы воздействиям извне и изменение ее поведения под этими воздействиями.}
&& Воздействия на входах системы называются \note{стимулами}.
& \define{Изменчивость системы со временем}{изменение значений внутренних параметров, состава и структуры системы и любых их комбинаций.}
& Изменения могут носить различный характер.
&& \note{Функционирование} --- изменения не затрагивают структуру системы.
&&& Элементы заменяются эквивалентными.
&&& Внутренние параметры меняются без изменения структуры.
&&& Пример: работа часов, городского транспорта.
&& Изменение состава системы носит \note{количественный} характер.
&&& Так же изменяется структура состава.
&&& Определенное время не влияет на свойства системы.
&&& Наращивание системы --- \note{рост}.
&&& Рост происходит за счет потребления ресурсов.
&&& Пример: расширение мусорной свалки, кладбища.
&&& Обратные росту изменения --- \note{спад}.
&& Изменения существенных свойст носят \note{качественный} характер.
&&& \note{Развитие} --- изменения в позитивном направлении.
&&& Развитие происходит за счет усвоения и использования информации.
&&& Возможно появление новых функций.
&&& Развитая система эффективней использует ресурсы.
&&& Обратный развитию процесс --- \note{деградация}.
& Развитие является результатом \note{обучения}.
&& Обучение нельзя осуществить для и вместо обучаемого
&& \note{Развитие возможно только как саморазвитие}.
& Последовательность изменений образует индивидуальный \note{жизненный цикл} системы.
&& Непродуманность будущих этапов приводит к краху.
& \important{Существование в изменяющейся среде} для системы выглядит как непрерывное изменение окружающей среды.
&& Имеет множество различных последствий для системы.
&& Сами изменения постоянно меняются.
&& Активность внутренних изменений зависит от силы изменений внешних.
&& Важны прогнозирование и обучение.
&& Эффективны:
&&& Выработка иммунитета к неподконтрольным изменениям.
&&& Усиление контроля над прочими изменениями.
\end{easylist}
\subsection{Синтетические свойства системы}
\begin{easylist}
& \define{Синтетические свойства}{обобщающие, интегральные свойства, делающие упор на взаимодействия со средой, целостность.}
& Выделяют четыре синтетических свойства.
&& Эмерджентность.
&& Неразделимость на части.
&& Ингерентность.
&& Целесообразность.
& \define{Эмерджентность}{порождение качественно новых свойств объединением частей в систему.}
&& Эмерджентные свойства не объяснимы через свойства отдельных частей системы.
&& Источником является структура системы.
&&& Разные структуры для идентичного состава дадут разные свойства.
&& Существуют неэмерджентные свойства, одинаковые со свойствами частей.
&& В искусственных системах, структура определяет эмерджентность.
&& В естественных системах, эмерджентность определяет структуру.
&&& Живой организм определяет смысл внутренних органов.
&& Взаимодействие частей важнее их работы по отдельности.
& Велика практическая важность \important{Неразделимости на части}.
&& Следствие эмерджентности.
&& При изъятии части, меняется состав и структура системы, её свойства.
&& Часть вне системы может иметь иные свойства.
&&& Свойства объекта проявляются во взаимодействии с окружающими объектами.
& \define{Ингерентность}{согласованность, приспособленность к окружающей среде}.
&& Привязана к функции, по которой производится оценка согласованности.
&& В естественных системах повышается путем естественного отбора.
&& В искусственных системах зависит от конструктора системы.
& \define{Целесообразность}{подчиненность всего в системе поставленной цели}.
&& Фундаментально для искусственных систем.
&& Цель редко однозначно определяет состав и структуру системы.
&&& Реализовать требуемую функцию можно различными путями.
& \note{Цель} --- желаемые будущие состояния системы в соответствующий момент времени.
&& Между исходным и требуемомым состояниями находятся промежуточные.
&& Каждому состоянию соответствует свой момент времени.
&& Целью является траектория движения через все промежуточные состояния.
& Цель естественных систем --- \note{будущее реальное состояние}.
& \enquote{Образ желаемого будущего} и \enquote{реальное будущее} отличаются как цели.
&& \enquote{Образ желаемого будущего} --- \note{субъективная цель}.
&&& Цель искуственных систем.
&&& Ограничены воображением.
&& \enquote{Реальное будущее} --- \note{объективная цель}.
&&& Цель естественных систем.
&&& Ограничены законами природы и доступными ресурсами.
& Важно установить реализуемость субъективной цели до начала попыток реализации.
&& Субъективная цель должна принадлежать к объективным целям.
&& Осуществимы лишь цели, могущие стать реальностью.
& \note{Идеал} --- заведомо недостижимая цель, не считаемая недостойной стремления к ней.
&& Привлекателен.
&& Допускает приблжение к себе.
&& Пример: гармонически развитая личность, стремление самосовершенствоваться.
\end{easylist}
\subsection{Системная картина мира}
\begin{easylist}
& \note{Думай глобально, действуй локально}.
&& Уделяй основное внимание индивидуальным особенностям конкретной системы.
&& Понимай всеобщую системность мира.
\end{easylist}
\section{Глава 3}{Модели и моделирование}
\subsection{Моделирование --- неотъемлемая часть любой деятельности}
\begin{easylist}
& Возможные виды деятельности --- познание мира и его преобразование.
& Любая деятельность возможна только благодаря \note{моделям}.
&& \define{Модель}{система для обеспечения взаимодействия между субъектом и реальностью.}
& Преобразование мира требует наличия цели до начала работы, т.е. модель того, чего пока нет.
&& Описание последовательности промежуточных действий.
&& Преобразовательная деятельность невозможна без моделирования.
& Результат познания --- информация о внешней среде.
&& Должен быть зафиксирован в виде определенной модели.
&& Модель --- форма существования знания.
\end{easylist}
\subsection{Анализ и синтез как методы построения моделей}
\begin{easylist}
& Выделяют два метода познания.
&& Аналитический --- как устроена система.
&& Синтетический --- как система взаимодействует со средой.
& Процедура анализа.
&& Сложное расчленить на мелкие, более простые части.
&& Дать объяснение полученным фрагментам.
&& Объединить объяснение частей в объяснение целого.
& Если какая-то часть остается непонятной, операция декомпозиции повторяется.
& Нелья разрывать связи частей.
&& Нарушает эмерджентные свойства системы.
&& Анализ --- различение частей, а не разбиение на части.
& Продукты анализа.
&& Модель состава системы.
&& Модель структуры система
&& Модель черного ящика для каждого элемента системы.
& Процедура синтеза.
&& Выделение метасистемы.
&& Рассмотрение состава и структуры метасистемы.
&& Объяснение роли искомой системы в метасистеме через ее связи с другими подсистемами.
& Продукты синтеза.
&& Модель состава метасистемы.
&& Модель структуры метасистемы.
&& Модель черного ящика системы.
& Анализ и синтез не противоположны.
&& Дополняют друг друга.
&& В анализе есть синтетический компонент.
&& В синтезе есть аналитический компонент.
\end{easylist}
\subsection{Что такое модель?}
\begin{easylist}
& Модель.
&& Средство осуществления любой деятельности субъекта.
&& Форма существования знаний.
&& Системное отображение оригинала.
\end{easylist}
\subsection{Аналитический подход к понятию модели}
\begin{easylist}
& Модели можно разделить на две группы.
&& Абстрактные --- средства мышления.
&& Реальные --- материальные средства.
& Абстрактные модели могут быть воплощены средствами языка для передачи другим субъектам.
& Язык --- универсальное средство моделирования.
&& Возможно за счет расплывчатости смысла слов.
& Групповая деятельность требует профессионального языка.
&& Такие языки более точны, нежели разговорный.
&& Уменьшение неопределнности возможно только за счет дополнительной информации.
&& Максимальный предел точности --- язык математики.
&& Слово --- элементарная языковая модель.
& Существует спектр языков разной степени определенности.
&& Каждому соответствует спектр моделей разной точности.
& Особенность системного анализа --- развить описание проблемной ситуации в сторону большей точности.
&& Важно движение в сторону уточнения, пока точности не хватит для решения проблемы.
\end{easylist}
\subsection{Классификация --- простейшая абстрактная модель разнообразия реальности}
\begin{easylist}
& Описать мир конечыми фразами можно только упрощенно.
&& Выделив схожие объекты и рассматривая их как одинаковые, можно получить \note{класс}.
&& Оставшиеся вне класса объекты, можно объединить в другие классы.
&& Мир можно описать конечным множеством различных классов.
& Слова языка --- названия некоторых классов.
& Классификация --- простейшая абстрактная модель разнообразия действительности.
&& Идентификация объекта --- определение его класса.
& Построение моделей субъектом --- творческий процесс.
&& Множественность характеристик для классификации.
&& Определение понятий \enquote{сильных} и \enquote{слабых} различий.
& Классификация --- только модель разнообразия реальности.
&& Всегда есть объект, который нельзя однозначно отнести к какому-либо классу.
\end{easylist}
\subsection{Искусственная и естественная классификации}
\begin{easylist}
& Искусственная классификация делит на классы исходя из поставленной цели.
&& Иногда называют произвольной.
& Естественная классификация исходит из существующих природных группировок.
& Классификация лежит в основе более сложных абстрактных моделей.
&& Увеличение числа классов.
&& Введение новых соотношений между классами.
\end{easylist}
\subsection{Реальные модели}
\begin{easylist}
& Реальные модели можно разделить на три группы.
&& Модели прямого подобия.
&&& Пример: След, печать, макет здания.
&& Модели косвенного подобия.
&&& Основаны на совпадении закономерностей, которым подчиняются явления.
&&& Пример: исторические параллели, подопытные животные.
&&& Осторожней с аналогиями, закономерности могут не совпадать.
&& Модели условного подобия.
&&& Соответствие модели и оригинала --- результат соглашения.
&&& Работают пока соблюдаются договоры о значении.
&&& Пример: деньги --- модель стоимости, буквы --- модель звуков.
\end{easylist}
\subsection{Синтетический подход к понятию модели}
\begin{easylist}
& Цель моделирования определяется неким субъектом.
&& Множественность целей ведет к множественности моделей одного и того же оригинала.
& Модель никогда не тождественна оригиналу.
&& Вся информация об оригинале не всегда нужна.
&& Не вся информация об оригинале познана.
&& У модели есть собственные свойства.
&& Истинная информация --- общее у модели и оригинала.
& Познавательные модели обслуживают процессы получения информации.
&& Не претендуют на окончательность --- всегда есть непознанное.
&& Изменяются при получении новых знаний.
&& Терпимо относятся к отличающимся и противоречащим мнениям.
& Прагматические модели обслуживают процесс преобразования реальности.
&& Отображают несуществующее, но желаемое.
&& Носят декларативный характер.
&& Реальность \enquote{подгоняется} под модель.
\end{easylist}
\subsection{Понятие адекватности}
\begin{easylist}
& Разные модели обеспечивают разную степень успешности в достижении цели.
&& Адекватность модели.
&& Адекватные модели позволяют достигнуть успеха.
&& Недекватные модели не обеспечивают успеха.
& Для познавательных моделей, адекватность и истинность --- синонимы.
& Ложные прагматические модели могут быть адекватными.
&& Допускется достижение цели с помощью лжи.
\end{easylist}
\subsection{Согласованность модели с культурой}
\begin{easylist}
& Модель должна быть согласована с окружающей средой.
&& Нельзя прочитать книгу на незнакомом языке.
&& Ингеретность модели культуре --- необходимое требование для моделирования.
& Степень ингерентности может изменяться
&& Возрастать: обучение пользователя, создание адаптера.
&& Убывать: забывание, уничтожение культуры.
\end{easylist}
\subsection{Иерархия моделей}
\begin{easylist}
& Уровни разработанности сведений по версии Р. Акоффа.
&& Данные --- \enquote{что?}
&&& Описание результатов измерений.
&&& Протоколы экспериментов.
&&& Сырые данные.
&& Информация --- \enquote{состав?}
&&& Результат первичной обработки данных.
&&& Упорядочение, классификация, структуризация.
&& Знание --- \enquote{структура?}
&&& Вторичная обработка данных.
&&& Выявление связей и закономерностей.
&& Понимание --- \enquote{почему?}
&&& Объяснение выявленных закономерностей.
&&& Построение теорий, дающих объяснение закономерностям.
&& Мудрость --- \enquote{зачем?}
&&& Сведения о том, зачем это необходимо.
&&& Хорошо ли это.
&&& Надо ли продолжать.
&&& Подход с точки зрения этики и эстетики.
& Ценность любого уровня существенного превосходит ценность предыдущего.
\end{easylist}
\subsection{Заключение}
\begin{easylist}
& Деятельность субъекта возможно только благодаря моделированию.
& Модель есть отображение оригинала.
&& Целевое.
&& Абстрактное или реальное.
&& Упрощенное.
&& Имеющее как истинное, так и ложное содержание.
&& Адекватное цели.
&& Ингерентное культуре пользователя.
\end{easylist}
\section{Глава 4}{Управление}
\begin{easylist}
& Управление --- целенаправленное воздействие на систему.
\end{easylist}
\subsection{Аналитический подход к управлению: пять компонентов управления}
\begin{easylist}
& Первый компонент управления --- объект управления, управляемая система.
&& Обозначим систему как $S$.
&& Выход как $Y(t)$.
&& Управляемый вход как $U(t)$, неуправляемый и наблюдаемый как $V(t)$.
&& Выход $Y(t)$ --- результат преобразования $U(t)$ и $V(t)$.
& Второй компонент --- цель управления.
&& Обозначим желаемое состояние, как $(T^*,Y^*)$.
&& Не стоит забывать, что в цель входит желаемый путь к ней --- $Y^*(t)$.
& Управляющее воздействие --- третий компонент управления.
&& Входы и выходы связаны как $Y(t) = S[V(t),U(T)]$.
&& Возможно существует управляющее воздействие $U^*(t)$, дающее $Y^*(t)$.
&& Поиск необходимого воздействия на самой системе часто неразумен.
& Четвертый компонент управления --- модель системы.
&& Предназначена для поиска необходимого управляющего воздействия.
&& Таким образом, $Y^*(t) = S_m[V(t),U^*_m(t)]$.
& Пятый компонент --- блок управления, управляющее устройство.
&& Может быть подсистемой или внешней системой.
& Два первых обязательных шага управления.
&& Найти на модели системы нужное управляющее воздействие.
&& Выполнить найденное воздействие на системе.
\end{easylist}
\subsection{Этап нахождения нужного управления}
\begin{easylist}
& Управление тем \enquote{лучше}, чем ближе выход системы $Y(t)$ к цели $Y^*(t)$.
&& При поиске на модели, оценивается выход модели $Y_m(t)$ по отношению к цели.
& Для измеримых численно выходов, вводится числовой критерий.
&& Равен нулю при совпадении сравниваемых функций.
&& Возрастает при различии.
& Для целей, заданных нечисловым способом, вводятся измеримые характеристики близости.
\end{easylist}
\subsection{Синтетический подход к управлению: семь типов управления}
\begin{easylist}
& При управлении возможны различные исходы, требующие различных действий.
&& Порождает семь типов управления.
& \important{Управление простой системой} или программное управление.
&& Воздействие на модель совпадает с воздействием на систему, т.е. модель адекватна.
&& Назовем такую систему \important{простой}.
&& Простота системы --- следствие адекватности модели.
&& Управляющее воздействие --- программа.
&& Пример: исправные бытовые приборы, различные автоматы.
& \important{Управление сложной системой}.
&& Поведение модели отличается от поведения системы.
&& Система не подчиняется управлению и является \important{сложной}.
&& Причина сложности - неадекватность модели.
&&& Вызвана недостаточной информацией об управляемом объекте.
&& Управление сводится к получению информации и совершенствованию модели.
&&& Если вся доступная информация использована, единственный способ --- эксперимент.
&& Алгоритм управления сложностью системы.
&&& Выбрать управляющее воздействие.
&&& Использовать управляющее воздействие на модели.
&&& Проверить данное воздействие на системе.
&&& Сравнить результаты.
&&& Скоректировать модель в сторону большего соответствия системе.
&&& Повторить алгоритм.
&& Коррекция модели возможна за счет изменения параметров.
&& Иногда, сложную систему можно превратить в простую за конечное число шагов.
&& Сложность некоторых систем исчерпать невозможно.
&&& Очень сложные системы. Пример: экономика, общество.
&& Алгоритм получил название \enquote{метод проб и ошибок}.
&&& Воздействие --- проба, расхождение между моделью и системой --- ошибка.
&& \enquote{Метод тыка} --- поиск воздействия на самой системе.
& \important{Управление по параметрам}.
&& Система может отклоняться от желаемой траектории со временем.
&& Внесение поправок в модель может оказаться нецелесообразным.
&& Можно внести изменить параметры системы, без изменения структуры.
&& Корректирующее устройство --- регулятор.
&&& Метод управления --- регулирование.
&&& Отрицательная обратная связь --- уменьшение отклонение.
&&& Положительная обратная связь --- увеличение отклонения.
&& Алгоритм управления по параметрам.
&&& Установить отклонение от желаемой траектории.
&&& Вывести и применить управляющее воздействие.
& \important{Управление по структуре}.
&& Систему не всегда можно вернуть на целевую траекторию изменением параметров.
&& Изменив структуру системы, создав новую, можно достигнуть цели другим путем.
&& Новая система может быть создана из частей старой или новых элементов.
&& Пример: сброс балласта с воздушного шара, пристройка к зданию.
& \important{Управление по целям}.
&& Иногда никакая комбинация наличных элементов не дает достижения цели.
&& Остается возможность изменить цель.
&& Обнаружение недостижимости цели --- повод отказаться от нее.
&& \enquote{Идеалы} --- недостижимы, но допускают неограниченное приближение.
&& Смена цели --- болезненный процесс.
&&& Чем выше уровень цели, тем тяжелее.
& \important{Управление большими системами}.
&& Время на поиск управляющего воздействия может быть ограничено.
&& Для поиска воздействия на \important{большую систему} при адекватной модели не хватает времени.
&& В противном случае, система --- \important{малая}.
&& Эффективнейший способ управления --- ускорить моделирование.
&&& Превращает систему в малую.
&& Альтернатива --- отказ от оптимального решения, в пользу более своевременного.
&&& Упрощение модели для получения слабого, но быстрого воздействия.
&&& Принятие первого удовлетворяющего результата моделирования.
& \important{Управление при отсутствии информации о конечной цели}.
&& Иногда конечная цель может быть неизвестна или непостижима.
&& Можно дать субъективное априорное определение конечной цели.
&&& Далее действия по предыдущим схемам.
&&& Пример: Управление крупными социальными системами. В чем смысл жизни?
&& Альтернатива --- выбор оптимального направления следующего шага.
&&& Выбор делается после исследования окружения.
&&& Итеративный процесс.
&&& Пример: эволюция и естественный отбор.
\end{easylist}
\subsection{Выводы}
\begin{easylist}
& Сложные системы и большие системы --- разные вещи.
&& Разные причины возникновения.
&& Разные способы решения.
&& Данные качества могут сочетаться во всех 4х вариантах.
\end{easylist}
\section{Глава 5}{Этапы системного анализа}
\begin{easylist}
& Переход из состояния проблемной ситуации к конечной цели осуществляется пошагово.
& Условия успеха системного исследования.
&& Доступ к любой необходимой информации.
&& Участие первых лиц --- обязательных участников проблемной ситуации.
&& Отказ от требований сформулировать заранее необходимый результат.
&&& Улучшающих вмешательств много, заранее они неизвестны.
& Путь от постановки проблемы до ее решения можно поделить на множество этапов.
&& Линейное прохождение этапов не всегда желательно.
&& Алгоритм решения проблемы может содержать циклы и возвраты.
&& Является другим представлением метода проб и ошибок.
&& Решение проблемы --- преодоление сложностей.
& Операции системного анализа зависят от используемых парадигм и метафор.
& Современный системный анализ акцентируется на:
&& Целостности, эмерджентности системы.
&&& Недопустимость рассмотрения улучшения части, цель --- улучшение всей системы.
&& Вхождение системы как части в другие системы.
&&& Необходимость учета целостности охватывающей метасистемы.
&& Взаимосвязь системы с другими системами.
&&& Рассмотрение проблемной ситуации с нескольких точек зрения.
\end{easylist}
\subsection{Этап первый. Фиксация проблемы}
\begin{easylist}
& Задача этапа --- сформулировать и зафиксировать проблему.
& Формулировка вырабатывается клиентом.
&& Важно не повлиять, не исказить его мнение.
& Нельзя сразу приступать к решению проблемы.
&& Модель ситуации клиента неадекватна.
&&& Не владеет полной информацией о ситуации.
&& Первоначальное определение проблемы не точно или не верно.
&&& Симптомы часто принимаются за проблему.
&& Возможно, следует устранить не проблему клиента, а некую другую.
&&& Пример: решение ряда проблем студентов требует решения проблем преподавателей.
& Фиксация проблемы клиента --- лишь отправная точка.
& Важно документальное оформление работ.
&& Человеческая память несовершенна.
&& Нарастание объемов информации.
&& Изменения окружающей обстановки со временем.
\end{easylist}
\subsection{Этап второй. Диагностика проблемы}
\begin{easylist}
& Задача этапа --- определить, к какому типу относится проблема.
&& Воздействие на самого недовольного субъекта.
&& Вмешательство в реальность.
& Ошибка может привести к неверным действиям и нанести вред.
&& Постановка диагноза более искусство, чем наука.
\end{easylist}
\subsection{Этап третий. Составление списка стейкхолдеров}
\begin{easylist}
& Список участников проблемной ситуации должен быть полным.
&& Невозможно учесть интересы того, кто неизвестен.
& Составить полный список --- невозможно.
&& Из-за открытости систем, все взаимосвязано.
&& В ситуации, так или иначе, участвует вся Вселенная.
& Разобьем Вселенную на два класса --- непосредственные участники и все остальные.
&& Непосредственные участники проблемной ситуации --- стейкхолдеры (stakeholders).
& Основной принцип выявления стейкхолдеров --- обладание нужной информацией.
&& Участники ситуации могут не обладать специальными знаниями.
&& Возможна необходимость участия экспертов по различным вопросам.
&& Некомпетентные, но прямые участники необходимы для оценивания и критики.
&&& Внешний наблюдатель, умеющий выслушивать других и наблюдать.
&&& Внутренний наблюдаель, с опытом работы в системе.
& Список стейкхолдеров --- модель черного черного ящика проблемной ситуации.
&& Актуальны ошибки первого, второго и третьего родов.
&& Необходимо принять меры для их устранения.
& Часто следует включать \enquote{безмолвных стейкхолдеров}.
&& Не только субъекты, но и другие участники ситуации.
&& Будущие поколения --- их еще нет, но их интересы необходимо учесть.
&& Прошлые поколения --- их уже нет, но их интересы представлены культурой.
&& Окружающая среда --- вмешательство не улучшающее, если вредит среде.
& \enquote{ПИРС} --- Пользователи, Исполнители, Руководители, Собственники.
& \enquote{Подсказка европейской комиссии}.
&& Что нужно знать? Чьи мнения и опыт были бы полезны?
&& Кто будет принимать решения по проекту?
&& Кто предполагается быть исполнителем этих решений?
&& Чья активная поддержка существенна для успеха проекта?
&& Кто имеет право быть участником проекта?
&& Кто может воспринять проект как угрозу?
& На более поздних этапах, может оказаться, что кто-то из стейкхолдеров пропущен.
&& Возвращение к этапу и пополнение списка.
\end{easylist}
\subsection{Этап четвертый. Выявление проблемного месива}
\begin{easylist}
& Проблемное месиво --- перечень субъективных оценок существующей реальности стейкхолдерами.
&& Входящие суждения не являются независимыми, они переплетены.
&& Проблема клиента --- ядро месива.
& Проблемное месиво может быть выявлено коллективным обсуждением стейкхолдеров.
&& Каждый стейкхолдер формулирует осознаваемые им проблемы.
&& Затем проводится обсуждение.
& Ограниченность описания реальности субъектом возможно избежать.
&& Предложить посмотреть на ситуацию с других точек зрения.
&& Существуют различные методики.
& Всё происходящее со временем будет претерпевать изменения.
&& Необходимо опираться не только на статическую модель, но и на динамическую.
& Проблемное месиво --- совокупность взаимосвязанных угроз и возможностей для организации.
&& Определяет, как организация довела бы себя до краха, если продолжила бы действовать так же.
& Иногда, формулирование проблемного месива можно свести к выявлению проблем стейкхолдеров.
&& При большей сложности ситуации, может потребоваться более детальное рассмотрение.
&&& Выполнение анализа системы.
&&& Выполнение анализа препятствий.
&&& Определение сценариев возможно будущего.
& Индивидуальные представления о проблемной ситуации --- исходные данные.
&& Извлечение сведений выражается в построении более содержательных моделей.
&& Первичная обработка дает описание, названное \enquote{информацией}.
&&& Группы проблем при статическом описании.
&&& Сценарии (опорные проекции) при динамическом описании.
&& Может потребоваться учет взаимосвязей между компонентами месива.
&& Некоторые технологии не требуют дальнейшей структуризации.
& Методика \enquote{идеализированного проектирования} Акоффа.
&& Выявление проблем и их последствий нужно для осознания необходимости перемен.
&& Проектирование перемен исходит из предположения, что существующая система исчезла.
&& Работа направлена на проектирование желательных особенностей.
& Лучший источник информации о стейкхолдере --- он сам.
&& Проблема доступности стейкхолдера.
& Всех стейкхолдеров можно разделить на обязательных и желательных участников.
&& Первые лица проблемосодержащей и проблеморазрешающей систем --- обязательны.
&& Остальные стейкхолдеры --- желательны.
&& При недоступности стейкхолдера, надо найти как можно более лучшего его представителя.
&& Для группы-стейкхолдера, выбирают компетентного представителя.
&&& Альтернатива --- методы прикладной социологии и статистики.
\end{easylist}
\subsection{Этап пятый. Определение конфигуратора}
\begin{easylist}

\end{easylist}
\newpage
\ListProperties(Hide=100,Hang=true,Progressive=3.5ex,Style*=$\triangleright$ ,Style1*=$\Rightarrow$ )
\section{Приложение}{Вопросы для Anki}
\subsection{Введение}
\begin{easylist}
& Почему накопление и обобщение опыта решения проблем началось (и продолжается) в рамках каждой отдельной профессии?
&& Поскольку для решения проблемы зачастую необходимы глубоко профессиональные знания, создалось впечатление, что проблемы в каждой области знания уникальны. Поэтому накопление и обощение опыта решения проблем началось в рамках каждой профессии отдельно.
& Почему, несмотря на громадное разнообразие проблем, технология (совокупность приемов) их решения практически одинакова в случае успеха и различается в случае неудач?
&& Идея универсального алгоритма действий по решению проблем опирается на понятие всеобщей системности. Ошибиться же можно множеством различных путей.
& Сформулируйте основные отличия прикладного системного анализа от традиционных наук.
&& Прикладной системный анализ нацелен на решение конкретной проблемы; для решения проблемы могут понадобиться знания любой из традиционных наук; системный анализ выполняется участниками проблемной ситуации.
& Почему прикладной системный анализ можно назвать над-дисциплинарной и меж-дисциплинарной область деятельности как в теоретической, так и в практической его сфере?
&& Необходимость использования знаний из различных сфер деятельности; привлечением к системному анализу соответствующих специалистов, задействованных в проблемной ситуации.
\end{easylist}
\subsection{Проблема и способы ее решения}
\begin{easylist}
& Поясните различия между понятиями \enquote{проблемная ситуация} и \enquote{проблема}.
&& \enquote{Проблемная ситуация} выделяет объективный компонент, реальную ситуацию; \enquote{проблема} --- субъективное отношение к реальной ситуации субъекта.
& Что значит \enquote{решить проблему}?
&& Решение проблемы объединяет в себе какие-либо действия, призванные уменьшить или совсем снять недовольство субъекта.
& Какие три способа воздействия на субъект без изменения реальности могут (при определенных условиях) привести к решению его проблемы? Каковы эти условия?
&& Сообщение субъекту дополнительной информации позитивного характера о ситуации. Может происходить в виде обучения, быть неправдивой или отфильтрованной полуправдой.
&& Изменение восприятия данной реальности субъектом, например, с помощью алкоголя или наркотков.
&& Прекращение взаимодействия субъекта с ситуацией. Субъект может быть повышен, направлен в отпуск, переведен в другой отдел, уволен.
& Каково основное отличие субъекта от объекта?
&& Субъект существует в реальной физической среде и наделен способностью оценивать свои взаимодействия со средой: что-то ему может нравиться, а может и не нравиться.
& Как определить смысл оценки, выраженной неким субъектом?
&& Так как оценки не бывают объективны, для понимания смысла необходимо уточнить критерии оценки.
& Почему при вмешательстве в реальность с целью решения проблемы приходится опираться на какую-то идеологию?
&& Правильным считается поведение, максимально согласующееся с принятой субъектом идеологией. Именно идеология и определяет, что плохо, а что хорошо.
& Воспроизведите классификацию идеологий на три типа. Каково основное отличие между ними?
&& Основное отличие между идеологиями --- определение того, какое отношение к другим субъектам считать правильным.
&&& \enquote{Принцип приоритета меньшинства}: осуществить вмешательство, угодное клиенту, интересы других участников не принимаются во внимание. Примеры: диктатура, иерархическая организация, эгоизм.
&&& \enquote{Принцип приоритета группы}: среди участников ситуации есть важные субъекты, помимо клиента. Вмешательство проводится с учётом интересов \enquote{наших}. Примеры: расизм, национализм, фашизм, коммунизм.
&&& \enquote{Принцип приоритета каждого}: все субъекты равноценны и равноправны, правильным признается только улучшающее вмешательство.
& Назовите четыре типа улучшающих вмешательств.
&& \important{ABSOLUTION} --- невмешательство. Пример: действия сапера при встрече с незнакомым взрывным устройством.
&& \important{RESOLUTION} --- частичное вмешательство. Пример: распределение по жребию или очереди.
&& \important{SOLUTION} --- оптимальное решение.
&& \important{DISSOLUTION} --- вмешательство с изменением условий, заканчивающееся полным исчезновением проблемы.
& Оптимальность обеспечивается при совокупном соблюдении двух требований. Каковы эти требования?
&& Оптимальный значит \enquote{наилучший в данных условиях}. Первым требованием является определение, критерия сравнения варианта. Вторым требованием является определение ограничений.
& Каков важный результат прикладного системного анализа конкретной проблемы, кроме решения самой проблемы?
&& Важный результат прикладного системного анализа - обучение системному анализу в процессе.
& Дайте определение следующим понятиям:
&& \define{Проблемная ситуация}{некоторое реальное стечение обстоятельств, положение вещей, которым кто-то недоволен, неудовлетворён и хотел бы изменить.}
&& \define{Оценка}{мнение, суждение субъекта о чем-либо.}
&& \define{Проблема}{субъективное отрицательное отношение субъекта к реальности.}
&& \define{Решение проблемы}{какие-либо действия, призванные уменьшить или совсем снять недовольство субъекта.}
&& \define{Вмешательство}{изменение проблемной ситуации, снижающее недовольство клиента.}
&& \define{Улучшающее вмешательство}{изменение проблемной ситуации, которое положительно оценивается хотя бы одним из ее участников и неотрицательно --- всеми остальными.}
&& \define{Прикладной системный анализ}{теория и практика проектирования и реализации улучшающих вмешательств. Методика решения проблем реальной жизни без создания новых проблем.}
&& \define{Оптимальность}{сочетание наилучших параметров по заданным критериям в рамках существующих ограничений.}
&& \define{\enquote{Твёрдая проблема}}{хорошо структурированная, допускающая построение количественных математических моделей. Например, многие инженерные и научные проблемы.}
&& \define{\enquote{Мягкая проблема}}{неструктурированные, описанные на естественном языке ситуации, не относящиеся к точным наукам.}
\end{easylist}
\subsection{Понятие системы}
\begin{easylist}
& Что такое статические свойства систем?
&& Статическими свойствами называются особенности конкретного состояния системы.
& Перечислите статические свойства.
&& Целостность
&& Открытость
&& Внутренняя неоднородность
&& Структурированность
& Как из открытости систем вытекает факт всеобщей взаимосвязанности в природе?
&& Открытость подразумевает связность системы и окружающей среды, обмен между ними любыми видами ресурсов
&& Исходя из этого, можно сказать, что между любыми двумя система обязательно существует цепочка систем, связывающаю их.
&& Выход каждой системы связан со входами другой системы.
& Что называется моделью черного ящика?
&& Моделью черного ящика называют перечень входов и выходов системы.
&& В этой модели отсутствует информация о внутренних особенностях системы.
& Назовите четыре рода ошибок, которые можно совершить при построении модели черного ящика.
&& Ошибка первого рода: несущественная связь включена в модель.
&& Ошибка второго рода: существенная связь не включен в модель.
&& Ошибкой третьего рода: последствия незнания о существовании связи. Для существенной связи соответствует ошибке второго рода.
&& Ошибка четвертого рода: неверная трактовка существенной связи, как входа или выхода.
& Что называется моделью состава системы?
&& Моделю состава системы называют иерархический список частей системы.
& Каковы трудности построения модели состава системы?
&& Целое можно делить на части по-разному.
&& Количество частей зависит от того, на каком уровне остановить дробление.
&& Внешняя граница системы имеет относительный, условный характер.
& При каких предположения можно говорить о наличии частей у системы?
&& ?
& Как определяется граница системы?
&& Так как граница системы имеет относительный, условный характер, определение следует проводить с учетом целей субъекта.
& Что называется моделью структуры системы?
&& Моделью структуры называют перечень существенных связей между элементами системы.
& В чем трудности построения модели структуры системы?
&& Модель вариабельна из-за возможности по-разному определить существенность связей.
&& Элементы могут привнести ошибки определения своих входов и выходов.
& Что такое динамические свойства системы?
&& Динамическими свойствами называют особенности изменений со временем внутри системы и вне ее.
& Перечислите динамические свойства.
&& Функциональность
&& Стимулируемость
&& Изменчивость системы со временем
&& Существование в изменяющейся среде
& Поясните различие между ростом и развитием системы.
&& Рост подразумевает наращивание системы за счет потребления ресурсов.
&& Развитие подразумевает изменение существенных свойств в позитивном направлении, за счет усвоения информации в процессе обучения.
& Что такое синтетические свойства системы?
&& Синтетическими называют обобщающие, интегральные свойства, делающие упор на взаимодействия со средой, целостность.
& Перечислите синтетические свойства.
&& Эмерджентность.
&& Неразделимость на части.
&& Ингерентность.
&& Целесообразность.
& Какое из статических свойств системы обеспечивает существование эмерджентных свойства системы?
&& Структурированность, так как источником эмерджентных свойств является структура системы.
& Что называется субъективной целью системы?
&& Субъективной целью называют цель, представляемую конструктором искусственной системы.
& Почему не любая субъективная цель достижима?
&& Достижение субъективной цели может быть невозможно в том случае, если данная цель находится за пределами законов природы, т.е. не принадлежит к объективным целям.
& Дайте определение следующим понятиям:
&& \define{Целостность системы}{существование системы как чего-то единого, целого, обособленного, отличающегося от всего остального; факт внешней различимости в среде.}
&& \define{Открытость системы}{связность системы и окружающей среды, обмен между ними любыми видами ресурсов.}
&& \define{Черный ящик}{совокупность входов и выходов системы.}
&& \define{Ошибка первого рода}{несущественная связь включена в модель.}
&& \define{Ошибка второго рода}{существенная связь не включен в модель.}
&& \define{Ошибкой третьего рода}{последствия незнания о существовании связи.}
&& \define{Ошибка четвертого рода}{неверная трактовка существенной связи, как входа или выхода.}
&& \define{Внутренняя неоднорость системы}{возможность выделения частей системы, проведения границ между ними.}
&& \define{Модель состава системы}{иерархический список частей системы.}
&& \define{Подсистема}{система, являющаяся частью более общей системы.}
&& \define{Элемент системы}{части на конечных ветвях иерархического дерева, считаемые простейшими и неделимыми в рамках заданной декомпозиции.}
&& \define{Структурированность}{наделенность системы определенной структурой.}
&& \define{Модель структуры системы}{перечень существенных связей между элементами системы.}
&& \define{Функция системы}{изменения, производимые системой в окружающей среде; результаты ее деятельности.}
&& \define{Стимулируемость}{подверженность системы воздействиям извне и изменение ее поведения под этими воздействиями.}
&& \define{Функционирование}{тип динамики системы, при котором изменения не затрагивают структуру системы: одни элементы заменяются эквивалентными; параметры меняются без изменения структуры.}
&& \define{Рост}{наращивание состава системы.}
&& \define{Спад}{уменьшение состава системы.}
&& \define{Развитие}{качественное улучшение существенных свойств.}
&& \define{Деградация}{качественное ухудшение существенных свойств.}
&& \define{Жизненный цикл}{последовательность изменений в истории системы.}
&& \define{Эмерджентность}{порождение качественно новых свойств объединением частей в систему.}
&& \define{Ингерентность}{согласованность, приспособленность к окружающей среде в рамках некой функции.}
&& \define{Цель субъективная}{образ желаемого будущего, ограниченный только желаниями и фантазией субъекта.}
&& \define{Цель объективная}{реальное будущее состояние, существующее в рамках законов природы.}
\end{easylist}
\subsection{Модели и моделирование}
\begin{easylist}
& Покажите, что познавательная деятельность субъекта невозможна без моделирования.
&& Познавательная деятельность представляет из себя составление модели окружающего мира.
& Покажите, что преобразовательная деятельность субъекта невозможна без моделирования
&& Преобразовательная деятельность требует наличие желаемого результата, т.е. модели того, чего нет, и описания алгоритма планируемых для достижения действий.
& Опишите алгоритм анализа.
&& Сложное расчленить на мелкие, более простые части.
&& Дать объяснение полученным фрагментам.
&& Объединить объяснение частей в объяснение целого.
&& Если какая-то часть остается непонятной, операция декомпозиции повторяется.
& Перечислите, какие модели он порождает анализ
&& Модель состава системы.
&& Модель структуры система
&& Модель черного ящика для каждого элемента системы.
& Опишите алгоритм синтеза.
&& Выделение метасистемы.
&& Рассмотрение состава и структуры метасистемы.
&& Объяснение роли искомой системы в метасистеме через ее связи с другими подсистемами.
& Какие модели порождает синтез? Какая из них описывает исследуемый объект (явление)?
&& Модель состава метасистемы.
&& Модель структуры метасистемы.
&& Модель черного ящика системы.
&& Модель черного ящика системы непосредственно описывает исследуемый объект
& Что такое модель?
&& Средство осуществления любой деятельности субъекта.
&& Форма существования знаний.
&& Системное отображение оригинала.
& Что такое \enquote{абстрактная модель}? Кроме языковых, какие еще примеры абстрактных моделей вы можете привести?
&& Абстрактная модель --- модель, представляющая собой инструмент мышления; её компонентами являются понятия, а не физические элементы.
&& В качестве абстрактных моделей могут выступать схемы, изображения, диаграммы, макеты, алгоритмы, математические описания, etc.
& Чем вызвано многообразие языков?
&& Групповая деятельность требует профессионального языка.
&&& Такие языки более точны, нежели разговорный.
&&& Позволяют уменьшенить неопределнность за счет дополнительной информации.
& Какова простейшая абстрактная модель разнообразия окружающей нас реальности?
&& Классификация представляет собой простейшую абстрактную модель разнообразия окружающей реальности, разделяя все объекты на классы по каким-либо признакам.
& Чем отличается искусственная и естественная классификации?
&& Искусственная классификация делит на классы исходя из поставленной цели.
&& Естественная классификация исходит из существующих природных группировок.
& Что называется \enquote{реальной моделью}?
&& Реальной моделью может выступать какой-либо материальный предмет, используемый в качестве модели.
& Приведите три типа реальных моделей (классификацию по происхождения подобия модели оригиналу).
&& Модели прямого подобия.
&&& Пример: След, печать, макет здания.
&& Модели косвенного подобия.
&&& Основаны на совпадении закономерностей, которым подчиняются явления.
&&& Пример: исторические параллели, подопытные животные.
&&& Осторожней с аналогиями, закономерности могут не совпадать.
&& Модели условного подобия.
&&& Соответствие модели и оригинала --- результат соглашения.
&&& Работают пока соблюдаются договоры о значении.
&&& Пример: деньги --- модель стоимости, буквы --- модель звуков.
& Чем отличается использование познавательных и прагматических моделей?
&& Познавательные модели обслуживают процессы получения информации.
&&& Не претендуют на окончательность --- всегда есть непознанное.
&&& Изменяются при получении новых знаний.
&&& Терпимо относятся к отличающимся и противоречащим мнениям.
&& Прагматические модели обслуживают процесс преобразования реальности.
&&& Отображают несуществующее, но желаемое.
&&& Носят декларативный характер.
&&& Реальность \enquote{подгоняется} под модель.
& Почему в любой модели есть, кроме истинного, и (обязательно и неизбежно) неистинное содержание?
&& Истинное содержание является чем-то общим и для модели, и для оригинала, благодаря чему модель может служить заменителем оригинала. В то же время, т.к. модель так же является системой, она обязательно будет обладать каким-либо собственными свойствами --- неистинным содержанием, которое не будет иметь отношения к оригиналу.
& Какое качество модели называется адекватностью?
&& Адекватностью называют степень успешности модели в достижении цели субъекта.
& Что является окружающей средой для модели?
&& Окружающей средой для искомой модели выступает культура (мир моделей) пользователя.
& Дайте определения следующих терминов:
&& \define{Анализ}{разложение системы на составные, более простые, части с целью объяснения целого на основе объяснения частей.}
&& \define{Синтез}{рассмотрение состава и структуры метасистемы, с целью объяснения роли системы на основе ее связей с подсистемами метасистемы.}
&& \define{Модель абстрактная}{модель, представляющая собой инструмент мышления; её компонентами являются понятия, а не физические элементы.}
&& \define{Модель языковая}{слово, представляющее собой какое-либо понятие.}
&& \define{Модель реальная}{материальный предмет, используемый в качестве модели.}
&& \define{Классификация искусственная}{разделение на классы на основе поставленной цели. Количество и границы классов задаются целью.}
&& \define{Классификация естественная}{разделение на классы на основе существующих природных группировок.}
&& \define{Модели познавательные}{модели, обслуживающие процессы получения о внешнем мире.}
&& \define{Модели прагматические}{модели, обслуживающие процессы преобразования реальности в соответствии с целями субъекта.}
&& \define{Адекватность модели}{степень успешности модели в достижении цели.}
&& \define{Культура (субъекта, организации, нации --- любой социальной системы)}{мир моделей, сформировавшийся в процессе жизнедеятельности системы.}
\end{easylist}
\subsection{Управление}
\begin{easylist}
& Какие пять составляющих обеспечивают выполнение процесса управления?
&& Для выполнения процесса управления необходимы: объект управления; цель управления; управляющее воздействие; модель системы и блок управления.
& При каких условиях поиск управляющего воздействия на самой системе является неразумным, неприемлемым?
&& Поиск управляющего воздействия на самой системе сводится к перебору возможных управляющих воздействий.
&& В случае, если множество управляющих воздействий будет достаточно велико или каждое неверное решение будет причинять какие-либо потери, работа напрямую с системой становится нецелесообразной.
& Что называется простой системой? В чем причина простоты?
&& В случае, если поведение самой системы и ее модели для одного и того же управляющего воздействия совпадают, такую систему можно назвать простой.
&& Простота системы есть следствие адекватности модели.
&& Так же можно сказать, что простота системы характеризует полноту знаний о системе.
& Какую систему называют сложной? Какова причина сложности?
&& Сложной называется та система, у которой поведение самой системы и ее модели для одного и того же управляющего воздействия отличаются.
&& Сложность системы есть следствие неадекватности модели.
&& Так же можно сказать, что сложность системы характеризует недостаток знаний о системе.
& Опишите алгоритм метода проб и ошибок. Какими особенностями он обладает?
&& Метод проб и ошибок предназначен для получения дополнительной информации о системе.
&& Представляет из себя рекурсивный алгоритм, на каждой итерации которого:
&&& На основе модели выбирается оптимальное управляющее воздействие.
&&& Происходит проба полученного управляющего воздействия на системе.
&&& На основе полученных данных совершенствуется модель.
& Чем отличается метод проб и ошибок от \enquote{метода тыка}?
&& Метод проб и ошибок предполагает поиск необходимого управляющего воздействия на основе модели системы, тогда как \enquote{метод тыка} предполагает поиск воздействия на самой системе.
& Перечислите, какие функции выполняет регулятор.
&& Регулятор выполеняет следующие функции:
&&& Держит в памяти опорную траекторию, ведущую к цели.
&&& Следит за реальной траекторией движения системы.
&&& Ищет различие между опорной и реальной траекториями.
&&& Вычисляет на основе модели корректирующее, дополнительное управляющее воздействие.
&&& Выполняет полученное управляющее воздействие на системе, возвращая ее на опорную траекторию.
& В чем состоит управление по структуре?
&& Управление по структуре предполагает изменение структуры существующей системы, с использованием как частей исходной системы, так и привлечением новых частей из вне.
&& Управление по структуре целесообразно в тех случаях, когда отклонение системы от целевой траектории настолько велико, что не позволяет вернуть систему назад изменением параметров.
& В чем состоит управление по целям? При каких условиях применим этот тип управления?
&& Управление по целям предполагает смену цели в сторону понижения предъявляемых требований и переориентирования на достижимые сроки.
&& Позволяет снизить требования к ресурсам, необходимым для достижения цели.
&& Данный тип управления целесообразен в тех случаях, когда другие типы управления не позволяют достигнуть цели при существующих ресурсах.
& Что такое \enquote{большая система}? Каковы варианты управления ею?
&& Большой системой является та система, для воздействия на которую достаточно информационного ресурса, но недостаточно времени.
&& Управлять данной системой можно:
&&& Путем превращения ее в малую системе, за счет ускорения процесса моделирования.
&&& За счет отказа от оптимального решения в пользу первого удовлитворительного варианта.
&&& Упрощением модели, с целью получения более слабого варианта, требующего меньше времени для вычисления.
& Придумайте примеры систем, которые были бы одновременно: малой и простой, малой и сложной, большой и простой, большой и сложной.
&& Малая-простая: практически любой бытовой прибор, например, электрочайник.
&& Малая-сложная:
&& Большая-простая: шахматы, го.
&& Большая-сложная:
\end{easylist}
\subsection{Этапы системного анализа}
\begin{easylist}
& Почему необходимо документально зафиксировать проблему клиента?
&& Человеческая память несовершенна.
&& Нарастание объемов информации.
&& Изменения окружающей обстановки со временем.
& Почему не следует приступать к решению проблемы сразу после ее фиксации?
&& Модель ситуации клиента неадекватна.
&&& Не владеет полной информацией о ситуации.
&& Первоначальное определение проблемы не точно или не верно.
&&& Симптомы часто принимаются за проблему.
&& Возможно, следует устранить не проблему клиента, а некую другую.
&&& Пример: решение ряда проблем студентов требует решения проблем преподавателей.
& Попробуйте сформулировать соображения, которые помогли бы вам сделать выбор между тем, нужно ли воздействовать на субъект или надо вмешиваться в саму проблемную ситуацию.
&& ???
& Кто такие \enquote{стейкхолдеры}?
&& Под термином \enquote{стейкхолдер} понимают прямых участников проблемной ситуации.
& Значит ли то, что в дальнейшем мы будем учитывать интересы только \enquote{стейкхолдеров}, т.е. что интересы \enquote{нестейкхолдеров} вообще никак не будут учтены?
&& Нет, идея улучшающего вмешательства предполагает, что вмешательство никому не сделает хуже.
& Запомни ли вы подсказки, способствующие составлению более полного списка \enquote{стейкхолдеров}?
& Список стейкхолдеров --- модель черного черного ящика проблемной ситуации.
&& Актуальны ошибки первого, второго и третьего родов.
&& Необходимо принять меры для их устранения.
& Часто следует включать \enquote{безмолвных стейкхолдеров}.
&& Не только субъекты, но и другие участники ситуации.
&& Будущие поколения --- их еще нет, но их интересы необходимо учесть.
&& Прошлые поколения --- их уже нет, но их интересы представлены культурой.
&& Окружающая среда --- вмешательство не улучшающее, если вредит среде.
& \enquote{ПИРС} --- Пользователи, Исполнители, Руководители, Собственники.
& \enquote{Подсказка европейской комиссии}.
&& Что нужно знать? Чьи мнения и опыт были бы полезны?
&& Кто будет принимать решения по проекту?
&& Кто предполагается быть исполнителем этих решений?
&& Чья активная поддержка существенна для успеха проекта?
&& Кто имеет право быть участником проекта?
&& Кто может воспринять проект как угрозу?
& Что называется \enquote{проблемным месивом}?
&& Проблемное месиво --- перечень субъективных оценок существующей реальности стейкхолдерами.
& Что является динамическим вариантом проблемного месива?
&& ???
& Почему не следует решать проблему клиента в отрыве от проблемного месива?
&& Проблемное месиво представляет собой неделимую систему, содержащим проблему.
&& Решение в отрыве от проблемного месива может не являться улучшающим вмешательством.
& Что значит \enquote{работать с проблемным месивом как с целым}?
&& ???
& Как решаются трудности, возникающие при недоступности части стейкхолдеров?
&& В случае недоступности стейкхолдера, нужно найти как можно более лучшего его представителя.
&& Желательными стейкхолдерами можно пренебречь.
\end{easylist}
\end{document}
