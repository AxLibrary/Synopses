\documentclass{article}
% Кодировка, поддержка русского языка
\usepackage[T2A]{fontenc}
\usepackage[utf8]{inputenc}
\usepackage[english,russian]{babel}
% Отступы от края страницы
\usepackage{geometry}
\geometry{left=2cm}
\geometry{right=2cm}
\geometry{top=2cm}
\geometry{bottom=2cm}
\geometry{bindingoffset=0cm}
% Вложенные списки
\usepackage[ampersand]{easylist}
\ListProperties(Hide=100,Hang=true,Progressive=3.5ex,Style*=$\triangleright$ )

\newcommand{\note}[1]{\textit{#1}}
\newcommand{\important}[1]{\textbf{#1}}
\newcommand{\enquote}[1]{,,#1''}
\renewcommand{\section}[2]{
	\vspace{6em}
	\begin{flushright}
	\Large
	\baselineskip=0.5\baselineskip
	\textbf{#1}
	\\
	\rule[0.5\baselineskip]{\textwidth}{0.15pt}
	\\
	\textbf{#2}
	\end{flushright}
	}
\renewcommand{\subsection}[1]{
	\vspace{2em}
	\begin{flushright}
		\large
		\textbf{#1}
	\end{flushright}
	}
\renewcommand{\title}[2]{
	\begin{center}
		\LARGE
		\baselineskip=0.5\baselineskip
		\textbf{#1}
		\\
		\rule[0.5\baselineskip]{0.7\textwidth}{0.15pt}
		\\
		\textbf{#2}
		\\\baselineskip=2\baselineskip(конспект)		
	\end{center}
	}
\newcommand{\define}[2]{
	\textbf{#1} --- #2
	}
\begin{document}
\title{Ф.П. Тарасенко}{Прикладной системный анализ}
\section{Введение}{Как возник системный анализ}
\begin{easylist}
& Деятельность человека состоит в решении \note{проблем}.
& Для решения могут потребоваться различные знания.
&& Создает впечатление, что проблемы различных специальностей уникальны.
&& Накопление опыта решения проблем в рамках каждой профессии отдельно.
& Вероятность успеха повышается, если следовать одним и тем же советам.
&& Не зависит от природы проблемы. 
&& Опирается на всеобщую системность, единство и общность законов мироздания.
& Общеупотребительная методика развита до \important{прикладного системного анализа}.
&& Нацелен на решение конкретной проблемы. 
&& В теоретической сфере, прикладной системный анализ может требовать знаний из различных областей традиционных наук. 
&& В практической деятельности аналитик направляет коллектив участников ситуации, являющихся специалистами в требуемой области.
\end{easylist}
\section{Глава 1}{Проблема и способы ее решения}
\begin{easylist}
& \define{Проблемная ситуация}{некоторое реальное положение вещей, которым кто-то недоволен и хотел бы изменить.}
& \define{Проблема}{Субъективное отрицательное отношение субъекта к реальности}
& В понятиях проблемы и проблемной ситуации неразрывно связаны два аспекта
&& Объективный --- наличие реальной ситуации.
&& Субъективный --- негативная оценка реальности субъектом.
&& Отличие понятий в том, на чем делается акцент.
& \define{Решение проблемы}{какие-либо действия, призванные уменьшить или совсем снять недовольство субъекта.}
& Все способы решения проблем можно разделить на две группы:
&& Воздействовать на субъект с целью уменьшить его недовольство, не изменяя реальности.
&& Изменить реальность так, чтобы недовольство субъекта ослабло.
\end{easylist}
\subsection{Способы влияния на субъект}
\begin{easylist}
& Сообщение субъекту дополнительной информации о ситуации. 
&& Обязательно должна быть положительной.
&& Не обязательно должна быть правдивой.
&& Может осуществляться в виде \note{обучения} субъекта.
&& Возможно сокрытие правды, либо отфильтрованная полуправда.
& Изменение восприятия данной реальности субъектом.
&& Воздействия --- гипноз, наркотики, алкоголь и т.п.
& Прекращение взаимодействия субъекта с ситуацией.
&& Отпуск, перевод в другой отдел, увольнение.
\end{easylist}
\subsection{Вмешательство в реальность}
\begin{easylist}
& В ситуации участвуют не только недовольный субъект, но и другие субъекты.
&& Субъекты оценивают ситуацию со своих позиций.
&& Изменение ситуации в результате вмешательства, будет замечено и оценено ее участниками.
& Субъект существует в реальной физической среде и подвержен её воздействиям.
&& Субъект наделен способностью \note{оценивать} свои взаимодействия со средой.
&& Все оценки имеют субъективный, индивидуальный характер.
& Любая оценка требует уточнения.
&& Оценки не бывают объективны.
&& Необходимо выяснить критерии оценки, чтобы понять смысл.
\end{easylist}
\subsection{Три типа идеологий вмешательства}
\begin{easylist}
& Правильное поведение максимально согласуется с принятой субъектом \note{идеологией}. 
& Идеология пределяет, что плохо, а что хорошо.
&& Отличаются определением \note{какое отношение к другим субъектам считать правильным}. 
& Можно выделить три типа.
&& \note{Принцип приоритета меньшинства.}
&&& Приводит к тому, чтобы осуществить вмешательство, угодное клиенту.
&&& Интересы других участников не принимаются во внимание.
&&& Пример: диктатура, иерархическая организация, эгоизм, и т.д.
&& \note{Принцип приоритета группы.}
&&& Среди участников ситуации, кроме клиента, есть другие ценные субъекты.
&&& Вмешательство должно проводиться с учётом интересов всех \enquote{наших}.
&&& Пример: расизм, национализм, фашизм, коммунизм, и т.д.
&& \note{Принцип приоритета каждого.}
&&& Нет ни одного одинакового субъекта.
&&& Все субъекты равноценны и равноправны.
&&& Правильным, моральным признается только улучшающее вмешательство.
& \define{Улучшающее вмешательство}{изменение проблемной ситуации, которое положительно оценивается хотя бы одним из ее участников и неотрицательно всеми остальными.}
& \define{Прикладной системный анализ}{теория и практика проектирования и реализации улучшающих вмешательств. Методика решения проблем реальной жизни без создания новых проблем.}
\end{easylist}
\subsection{Четыре типа вмешательств}
\begin{easylist}
& \important{ABSOLUTION} --- невмешательство.
&& Расчет на то, что естественный ход событий приведет к разрешению проблемы.
&& Обладает признаком улучшающего вмешательства --- никому не становится хуже.
&& События должны вести к разрешению проблемы
&& Предлагаемые вмешательства приводят к худшим результатам.
&& Пример: поведение врача при невозможности исцелению пациента, действия сапера при встрече с незнакомым взрывным устройством.
& \important{RESOLUTION} --- частичное вмешательство.
&& Снижает неудовлетворенность, ослабляет остроту проблемы.
&& Не устраняет проблему полностью.
&& Обычно применяется при дефиците ресурсов.
&& Примеры: распределение по жребию или очереди.
& \important{SOLUTION} --- оптимальное решение.
&& \define{Оптимальность}{сочетание наилучших параметров по заданным \note{критериям} в рамках существующих \note{условий} и ограничений.}
&& Ограничения могут быть неизвестны из-за нехватки информации о проблеме.
&& Оптимальность может выступать недостижимым идеалом.
& \important{DISSOLUTION} --- полное исчезновение проблемы и не появление новых проблем.
&& Условия и ограничения рассматриваются как подлежащие изменениям.
&& Поиск новых, недопустимых ранее вариантов, более эффективных, чем ранее оптимальные.	
\end{easylist}
\subsection{Еще о прикладном системном анализе}
\begin{easylist}
& Процесс решения проблемы не может быть выполнен лишь самим системным аналитиком.
& Вовлечение участников ситуации является обязательным.
&& Обладают необходимой информацией.
&& Будут воплощать разработанное впешательство.
& Выполнение работы собственными усилиями --- эффективная форма обучения. 
&& В прикладном системном анализе естественно встроенно обучение системному анализу.
\end{easylist}
\section{Глава 2}{Понятие системы}
\subsection{Статические свойства системы}
\begin{easylist}
& \note{Статическими свойствами} называются особенности конкретного состояния системы.
& Выделяют четрые статических свойства
&& Целостность
&& Открытость
&& Внутренняя неоднородность
&& Структурированность
& \define{Целостность}{существование системы как чего-то единого, целого, обособленного, отличающегося от всего остального; факт внешней различимости в среде.}
& \define{Открытость}{связность системы и окружающей среды, обмен между ними любыми видами ресурсов.}
& Связи системы со средой имеют направленный характер.
&& Через \note{выходы} система влияет на среду, через \note{входы} среда влияет на систему.
& \note{Модель черного ящика} --- перечень входов и выходов системы.
&& Содержит конечный список связей, у реальной системы их число не ограничено.
&& Не содержит информации о внутренних особенностях системы.
&& Должна содержать связи, существенные для достижения цели.
&& Оценку важности связи может дать только субъект.
& Субъект может ошибаться при оценке важности. Возможны четыре типа ошибок.
&& \note{Ошибка первого рода}: несущественная связь включена в модель.
&& \note{Ошибка второго рода}: существенная связь не включен в модель.
&& \note{Ошибкой третьего рода}: последствия незнания о существовании связи.
&&& Для существенной связи соответствует ошибке второго рода.
&&& Для исправления необходимы новые знания.
&& \note{Ошибка четвертого рода}: неверная трактовка существенной связи, как входа или выхода.
& Все системы связаны между собой.
&& Между двумя системами всегда существует цепочка связей.
&& Выход одной системы является входом другой.
&& Прямая и обратная цепи, как правило, различны.
& \define{Внутренняя неоднорость}{сводится к выделению \note{частей системы}.}
&& Обособление относительно однородных участков, проведение границ между ними.
&& Выделенные части тоже неоднородны, можно выделить еще более мелкие.
&& \note{Модель состава системы} --- иерархический список частей.
& Построение модели состава сопряжено с трудностями.
&& Целое можно делить на части по-разному.
&&& Зависит от того, что требуется для достижения цели. 
&&& Можно \note{различать} нужные части, но не следует \note{разделять их}.
&& Количество частей зависит от того, на каком уровне остановить дробление.
&&& \note{Элементы} --- части на конечных ветвях иерархического дерева.
&&& Уровень декомпозиции зависит от того, что считать \note{элементарным}.
&& Внешняя граница системы имеет относительный, условный характер.
&&& Любая система является частью большей системы.
&&& Мета-систему можно делить на подсистемы по-разному.
&&& Определение границ системы происходит с учетом целей субъекта.
& \define{Структурированность}{наделенность системы определенной структурой.}
& Части системы связаны и взаимодействуют между собой.
&& Свойства системы зависят от взаимодействия частей.
&& \note{Модель структуры системы} --- перечень существенных связей между элементами системы.
&&& Модель структуры определяется после выбора модели состава и зависит от нее.
&&& Модель вариабельна из-за возможности по-разному определить существенность связей.
&&& Элементы могут привнести ошибки определения своих входов и выходов.
\end{easylist}
\subsection{Динамические свойства системы}
\begin{easylist}
& \define{Динамические свойства}{особенности изменений со временем внутри системы и вне ее.}
&& Любые изменения можно рассмотреть как перемены в статических моделях системы.
& Выделяют четыре динамических свойства.
&& Функциональность.
&& Стимулируемость.
&& Изменчивость системы со временем.
&& Существование в изменяющейся среде.
& \define{Функции}{изменения, производимые системой в окружающей среде; результаты ее деятельности.}
&& Процессы на выходах системы рассматриваются как ее функции.
&& Субъект, использующий систему, \note{оценивает} и \note{упорядочивает} функции согласно потребностям.
& \define{Стимулируемость}{подверженность системы воздействиям извне и изменение ее поведения под этими воздействиями.}
&& Воздействия на входах системы называются \note{стимулами}.
& \define{Изменчивость системы со временем}{изменение значений внутренних параметров, состава и структуры системы и любых их комбинаций.}
& Изменения могут носить различный характер.
&& \note{Функционирование} --- изменения не затрагивают структуру системы.
&&& Элементы заменяются эквивалентными.
&&& Внутренние параметры меняются без изменения структуры.
&&& Пример: работа часов, городского транспорта.
&& Изменение состава системы носит \note{количественный} характер.
&&& Так же изменяется структура состава.
&&& Определенное время не влияет на свойства системы.
&&& Наращивание системы --- \note{рост}.
&&& Рост происходит за счет потребления ресурсов.
&&& Пример: расширение мусорной свалки, кладбища.
&&& Обратные росту изменения --- \note{спад}.
&& Изменения существенных свойст носят \note{качественный} характер.
&&& \note{Развитие} --- изменения в позитивном направлении.
&&& Развитие происходит за счет усвоения и использования информации.
&&& Возможно появление новых функций.
&&& Развитая система эффективней использует ресурсы.
&&& Обратный развитию процесс --- \note{деградация}.
& Развитие является результатом \note{обучения}.
&& Обучение нельзя осуществить для и вместо обучаемого
&& \note{Развитие возможно только как саморазвитие}.
& Последовательность изменений образует индивидуальный \note{жизненный цикл} системы.
&& Непродуманность будущих этапов приводит к краху.
% TODO HERE 
& \define{Существование в изменяющейся среде}{восьмое свойство}
& Изменяется не только сама система, но и окружающая ее среда. Необходимость существовать в подобном окружении имеет множество последствий для самой системы, начиная с необходимости ее приспособления к внешним переменам, до различных других реакций системы.
& Чем сильнее внешние изменения, тем активнее должны проводиться внутренние. И хотя важными средствами остаются прогнозирование и обучение, более эффективными считаются выработка иммунитета к неподконтрольным изменениям и усиление контроля над остальными.
\end{easylist}
\subsection{Синтетические свойства системы}
\begin{easylist}
& \note{Синтетическими} называют обобщающие, собирательные, интегральные свойства, делающие упор на взаимодействия системы со средой, на целостность.
& \define{Эмерджентность}{девятое свойство}
& Объединение частей в систему порождает у системы качественно новые свойства, не сводящиеся к свойствам частей, не выводящиеся из свойств частей, присущие только самой системе и существующие только пока система составляет одно целое. Например, ни одна из частей самолета летать не может, а самолет летает. Эмерджентные свойства определяются \note{структурой} системы.
&& У системы есть и не-эмерджентные свойства, одинаковые со свойствами ее частей. При \note{фрактальном} построении системы, части обладают свойствами системы в целом.
&& Система выступает как единое целое потому, что она является носителем эмерджентного свойства.
&& Для появления нового качества достаточно объединить хотя бы два элемента.
&& В естественных системах эмерджентность определяет, какие части должны быть соединены и как они должны взаимодействовать. Живой организм определяет смысл скелета, сердца и печени.
&& Действие системы больше зависит от того, как ее части взаимодействуют, чем от того, как они действуют сами по себе.
& \define{Неразделимость на части}{десятое свойство}
& При изъятии из системы некоторой части изменяется состав системы, а значит и ее структура. Это будет уже \note{другая} система, с отличающимися свойствами.
& Часть в системе и вне ее - это не одно и то же. Изменяются ее свойства, т.к. свойства объекта проявляются во взаимодействиях с окружающими его объектами.
& \define{Ингерентность}{одиннадцатое свойство}
& Система тем более ингерентна, чем лучше она согласована, приспособлена к окружающей среде. Ингерентность привязана к некоторой конкретной функцией.
& В естественных системах ингерентность повышается путем естественного отбора. В искусственных системах она должна быть особой заботой конструктора.
& \define{Целесообразность}{двенадцатое свойство системы}
& \define{Целесообразность}{подчиненность всего поставленной цели.}Цель определяет, какое эмерджентное свойство будет обеспечивать реализацию цели. \note{Система есть средство достижения цели}.
& Под целью стоит понимать \note{желаемые будущие состояние системы в различные моменты времени}. Конечная цель --- \note{будущее реальное состояние}. <<Цель как образ желаемого будущего>> и <<цель как реальное будущее>> --- не одно и то же; первое --- \note{субъективная цель}, а второе --- \note{объективная цель}. Субъективные цели есть порождение воображения, тогда как объективные --- результат проявления законов природы.  Ограничения на мысленные конструкции гораздо слабее ограничений на возможные реальные события, поэтому достижима не всякая субъективная цель, но лишь так, которая принадлежит к числу объективных целей.
& Важно установить реализуемость субъективной цели до начала попыток реализовать ее. Заведомо недостижимые цели, не считаемые недостойными стремления к ним, называются \note{идеалами}.
\end{easylist}
\subsection{Системная картина мира}
\begin{easylist}
& Системное видение мира состоит в том, чтобы, понимая его всеобщую системность, приступить к рассмотрению конкретной системы, уделяя основное внимание ее индивидуальным особенностям: \note{думай глобально, действуй локально}
\end{easylist}
\section{Глава 3}{Модели и моделирование}
\subsection{Моделирование --- неотъемлемая часть любой деятельности}
\begin{easylist}
& Все возможные виды деятельности можно разбить на два типа: познание мира и преобразование его. Любая деятельность субъекта становится возможной только благодаря \note{моделям} --- системам, предназначенным для обеспечения взаимодействия между субъектом и реальностью.
& Что бы человек не делал, до начала самой работы он должен определить цель, т.е. модель того, чего пока нет. Для достижения конечного результата необходимо выполнить определенную последовательность промежуточных действий, для чего их нужно \note{описать}. Таким образом, преобразовательная деятельность невозможна без моделирования.
& Конечный результат познания должен быть зафиксирован, описан, представлен в виде определенной модели. Конечной целью познания является построение моделей интересующей нас части мира. \note{Модель есть форма существования знания}.
\end{easylist}
\subsection{Анализ и синтез как методы построения моделей}
\begin{easylist}
& Процедура анализа состоит в последовательном выполнении следующих операций:
&& Сложно расчленить на более мелкие, предположительно более простые, части.
&& Дать объяснение полученным фрагментам.
&& Объединить объяснение частей в объяснение целого.
& Если какая-то часть системы остается непонятной, операция декомпозиции повторяется. Самая серьезная ловушка анализа - опасность разорвать связи частей, нарушив эмерджентные свойства системы. Таким образом, правильный анализ должен осуществлять различение частей, а не разбиение на части.
& Продуктами анализа являются модель состава и структуры системы, модель черного ящика для каждого элемента системы.
Процедура синтеза состоит в последовательном выполнении следующих операций:
&& Выделение метасистемы.
&& Рассмотрение состава и структуры метасистемы.
&& Объяснение роли искомой системы в метасистеме через ее связи с другими подсистемами.
& Продуктами синтеза являются модели состава и структуры метасистемы, модель черного ящика системы.
& Анализ и синтез не противоположны, а дополняют друг друга. В анализе есть синтетический компонент, а в синтезе --- аналитический.
\end{easylist}
\subsection{Что такое модель?}
\begin{easylist}
& Модель есть средство осуществления любой деятельности субъекта.
& Модель есть форма существования знаний.
& Модель есть системное отображение оригинала.
\end{easylist}
\subsection{Аналитический подход к понятию модели}
\begin{easylist}
& Модели подразделяются на \note{абстрактные} --- средства мышления и \note{реальные} --- материальные средства. Абстрактные модели могут быть воплощены средствами языка для передачи другим субъектам. Язык является \note{универсальным} средством моделирования. Одно из свойств, делающих это возможным, является \note{расплывчатость смысла слов}.
& Всякая групповая деятельность требует выработки специального, более точного \note{профессионального} языка. Профессиональные языки более точны, нежели разговорный, за счет большей определенности терминов. Снятие неопределенности возможно только за счет новой, \note{дополнительной информации}. Максимальный предел точности --- язык математики. Элементарная языковая модель --- \note{слово}.
& Существует спектр языков разной степени определенности, которым соответствует спектр моделей разной степени точности. Одна из главных особенностей прикладного системного анализа --- попытка развить описание проблемной ситуации в сторону более точного описания. Важен факт движения в сторону уточнения, пока точности определения условий не хватит для решения проблемы.
\end{easylist}
\subsection{Классификация --- простейшая абстрактная модель разнообразия реальности}
\begin{easylist}
* Описать бесконечно разнообразный мир конечными фразами можно только упрощенно и приблизительно. Выделение мало различающихся объектов в группы, с последующим рассмотрением членов группы как одинаковых между собой, дает \note{класс}. Оставшиеся вне класса объекты так же могут быть объединены между собой в новые классы, что позволит описать мир конечным множеством отличающихся друг от друга классов. Слова языка представляют собой названия некоторых классов. \note{Классификация есть простейшая абстрактная модель разнообразия действительности.}
& \note{Любая} классификация есть только \note{модель} разнообразия реальности. Всегда найдется объект, который нельзя однозначна отнести к тому или иному классу.
\end{easylist}
\subsection{Искусственная и естественная классификации}
\begin{easylist}
& При искусственной классификации разделение на классы производится исходя из поставленной цели --- на столько классов и с такими границами, как это диктуется целью. Искусственную классификацию так же называют \note{произвольной}.\\
Естественная классификация подразумевает классификацию на основе существующих природных группировок, которые как бы напрашиваются быть определенными как классы.
& Классификация лежит в основе более сложных абстрактных моделей, что достигается как увеличением числа классов, так и введением новых соотношений между классами.
\end{easylist}
\subsection{Реальные модели}
\begin{easylist}
& Реальной моделью может выступать какой-либо материальный предмет, используемый в качестве модели. Классификация по происхождению подобия между оригиналом и моделью, приводит к трем типам реальных моделей:
&& Модели \note{прямого подобия} созданы с помощью либо непосредственного взаимодействия (следы, печать), либо цепочки таких взаимодействий (фотография, макет здания).
&& Модели \note{косвенного подобия} или аналогии. Похожесть, аналогичность двух явлений объясняется совпадением закономерностей, которым они подчиняются. Моделями косвенного подобия являются: аналоговые ЭВМ, исторические параллели, подопытные животные в медицине. Следует осторожно пользоваться аналогиями, поскольку, кроме совпадающих закономерностей, у разных явлений есть и несовпадающие.
&& Модели \note{условного подобия}. Соответствие такой модели и оригинала устанавливается в результате соглашения между ее пользователями и носит условный характер. Они успешно работают до тех пор, пока известны и соблюдаются договоры об их значении. Примеры: деньги --- модели стоимости, буквы --- модели звуков.
\end{easylist}
\subsection{Синтетический подход к понятию модели}
\begin{easylist}
& Модель никогда не тождественна оригиналу. Зачастую в этом нет необходимости, т.к. для определенной цели требуется лишь некоторая, а не вся информация об оригинале. Разнообразие целей ведет к множественности моделей для одного и того же оригинала. Модели можно различать по типу целей, например, полезным бывает разделение моделей на \note{познавательные} и \note{прагматические}.
& Оригинал и модель могут различаться между собой. Во-первых, не вся информация об оригинале необходима в модели. \note{Истинная} информация является общей и для модели, и для оригинала. Именно благодаря ей модель может служить заменителем оригинала. Собственные свойства модели, не имеющие отношения к оригиналу, называются \note{ложными}.
& Познавательные модели обслуживают процессы получения информации о внешнем мире и не претендуют на окончательность: всегда остается что-то непознанное. Они подвержены изменениям при присоединении к ним новых знаний. В познавательной практике принято терпимо относиться к отличающимся и даже противоречивым мнениям.
& Прагматические модели обслуживают процессы преобразования реальности в соответствии с целями субъекта, отображая несуществующее, но желаемое, и имеют директивный характер. Это придает им статус <<единственно верных>>, что ярко выражено в религиях, морали, стандартах, etc. В прагматической деятельности реальность <<подгоняется>> под модель.
\end{easylist}
\subsection{Понятие адекватности}
\begin{easylist}
& Разные модели обеспечивают разную степень успешности в достижении цели, это свойство можно назвать \note{степенью их адекватности}. Модели, позволяющие субъекту достигнуть цели, называются \note{адекватными}, а не обеспечивающие успеха --- \note{неадекватными}.
& Для познавательных моделей, целевая принадлежность которых --- накопление истинных знаний, адекватность и истинность являются, по существу, синонимами. Для прагматических моделей достижение цели может быть проще с помощью лжи. Таким образом, ложные модели могут быть адекватными.
\end{easylist}
\subsection{Согласованность модели с культурой}
\begin{easylist}
& Чтобы модель реализовала свою модельную функцию, недостаточно только наличия самой модели. Необходимо, что бы модель была совместима, согласована с окружающей средой. \note{Ингерентность модели культуре} является необходимым требованием для осуществление моделирования.
\end{easylist}
\subsection{Иерархия моделей}
\begin{easylist}
& Модели могут описывать реальное и желаемое состояния рассматриваемой системы с разной степенью подробности. Акофф предложит такую классификацию:
&& \note{Данные} --- (что?) --- описание результатов измерений; исходные, <<сырые>> данные.
&& \note{Информация} --- (состав?) --- результат первичной обработки данных, их упорядочение, классификация, структуризация.
&& \note{Знание} --- (структура?) --- результат вторичной обработки данных; выявления связей и закономерностей между группами, классами данных.
&& \note{Понимание} --- (почему?) --- объяснение выявленных закономерностей, построение теорий, дающих такое объяснение.
&& \note{Мудрость} --- (зачем?) --- сведения о том, зачем это все надо, хорошо ли это, надо ли это продолжать или прекращать --- подход с точки зрения эстетики и этики.
\end{easylist}
\subsection{Заключение}
\begin{easylist}
& Любая деятельность субъекта возможна только благодаря моделированию.
& Модель есть отображение оригинала: целевое; абстрактное или реальное; упрощенное, приближенное; имеющее как истинное, так и ложное содержание; адекватное цели; ингерентное культуре пользователя.
\end{easylist}
\section{Глава 4}{Управление}
\begin{easylist}
& Исходное определение управления --- \note{целенаправленное воздействие на систему}.
\end{easylist}
\subsection{Аналитический подход к управлению: пять компонентов управления}
\begin{easylist}
& Первым компонентом управления является сам \note{объект управления, управляемая система}.
& Вторым обязательным компонентом системы управления является \note{цель управления}.
& \note{Управляющее воздействие} есть третий компонент управления. Тот факт, что входы и выходы системы связаны между собой, позволяется надеяться, что существует такое управляющее воздействие, при котором на выходе реализуется цель.
& Существует два типа управления:
&& Первый состоит в том, чтобы подать на управляемый вход какое-либо воздействие и посмотреть, что получиться. Если цель не достигнута --- подать другое воздействие и т.д. Иногда такой способ может быть единственно возможным, но чаще всего, такое способ управления является неразумным по ряду причин: цена, сроки, etc.
&& Второй подход основан на использовании всей имеющейся информации об управляемом объекте. Это означет, что поиск необходимого воздействия следует осуществлять не на самой системе, а на ее модели.
& Таким образом, \note{модель системы} становится четвертой часть процесса управления. Поиск управления на модели тоже требует потерь, но эти потери несравнимо меньше тех, которые были бы понесены во время поиска нужного управления на самой системе.
& Функция выполнения действий, необходимых для управления, возлагает на пятую составляющую процесса управления --- \note{блок управления, система (подсистема) управления, управляющее устройство}. Блок управления может быть как подсистемой управляемой системы, так и внешней системой.
& Два первых обязательных шага процесса управления:
&& Найти \note{на модели} системы нужное управляющее воздействие.
&& Выполнить это воздействие \note{на системе}.
\end{easylist}
\subsection{Этап нахождения нужного управления}
\begin{easylist}
& Управление тем <<лучше>>, чем ближе выход системы к цели. Поскольку поиск управления проходит на модели, наилучшим придется считать то управление, которое максимально приблизит выход модели к цели.
& Если выходы измеримы численно, то вводится некоторый числовой критерий, который равнялся бы нулю при свопадении сравниваемых функций и возрастал при любом их различии. Для целей, задаваемых нечисловым способом, все-равно вводятся измеримые характеристики близости результата к цели.
\end{easylist}
\subsection{Синтетический подход к управлению: семь типов управления}
\begin{easylist}
& \define{Первый тип управления}{\note{управление простой системой или программное управление}}
& Если подача на вход системы воздействия, обеспечивающего цель на выходе модели, приводит к такому же результату на выходе системы, это означает, что модель оказалась \note{адекватной}. В этом случае, систему будем называть \note{простой}. Простота есть следствие адекватности модели. Управляющее возедйствие в таком случае называется \note{программой}, а данный тип управления --- \note{программным управлением}. Примеры: исправные бытовые приборы, различные автоматы, компьютеры, etc.
& \define{Второй тип управления}{\note{управление сложной системой}}
& Если на найденное на модели управляющее воздействие откликается вовсе не так, как модель, это означает, что модель оказалась \note{неадекватна}. Система не подчиняется управлению и является \note{сложной}. Причиной сложности оказывается неадекватность ее модели, вызванная недостаточностью информации об управляемом объекте.
& Управление сложной системой сводится к добыче недостающей информации и последующем ее использовании для совершенствования, повышения адекватности модели. В случае, если при построении модели была использована вся доступная информаци, единственным источником информации остается эксперимент:
&& На текущей модели некоторым методом определяется управляющее воздействие, которое подается на вход системы
&& За счет полученной информации корректируется модель
&& Алгоритм рекурсивно повторяется до достижения цели.
& Поскольку на каждом шаге итерации будет получаться "не совсем цель", будут понесены определенные потери --- такова цена незнания.
& Каждое управляющее воздействие имеуется \note{пробным воздействием} или \note{пробой}, расхождение между целью и полученным в ходе итерации результатом --- \note{ошибкой}. Сам алгоритм управления сложной системой получил название \note{метод проб и ошибок}. Отличие данного метода от <<метода тыка>> в том, что второй ищет нужное воздействие на самой системе, а не на ее модели.
& Существуют системы, сложность которых исчерпать невозможно, однако их изучение не напросто, а бесконечно. Такие системы иногда называют \note{очень сложными} системами, например: природа, общество, экономика.
& \define{Третий тип управления}{\note{управление по параметрам или регулирование}}
& В случае, если, после подачи управляющего воздействия, система идет по желаемой траектории, отклоняясь от нее, может оказаться, что внесение поправок в модель нецелесообразно. В таком случае, можно внести изменения в систему за счет \note{изменений параметров} системы, без изменения структуры, за счет дополнительного, корректирующего управляющего воздействия.
& Для выполнения данной функции необходимо специальное устройство --- \note{регулятор}. Сам метод управления называется \note{регулированием}; если отклонение от опорной траектории необходимо уменьшить --- \note{управление с обратной отрицательной связью}, увеличить --- \note{управление с положительной обратной связью}.
& \define{Четвертый тип управления}{\note{управление по структуре}}
& В случае, если система настолько быстро и далеко отклоняется от целевой траектории, что не может быть возвращена на нее изменением параметров, приходится признать, что цель недостижима для существующей системы. Но, может быть, она достижима для \note{другой} системы?
& Изменив структуру системы, создав тем самым новую систему, можно по-прежнему достигнуть цели, пусть и другим путем. Такое управление называется \note{управлением по структуру}. Новая система может быть создана из частей (возможно не всех) старой системы или с использованием новых элеметнов. Примеры: сброс балласта с воздушного вара, пристройка к зданию, etc.
& \define{Пятый тип управления}{\note{управление по целям}}
& Могут встретиться случаи, когда потенциал управления по структуре исчерпан - никакая комбинация наличных элементов не обеспечивает достижение поставленной конечной цели. Остается смениь цель, понизив уровень притязаний, переориентироваться на достижимые сроки и/или другие параметры конечного состояния.
& Существуют другие, заведомо недостижимые цели, допускающие неограниченное приближение и достойные стремления к ним --- \note{идеалы}.
& Следует иметь в виду, что смена цели - болезненный процесс, тем более тяжелый, чем более высокого уровня цель приходится менять. Поэтому, метод требует осторожности.
& \define{Шестой тип управления}{\note{управление большими системами}}
& Может оказаться, что время, требующееся для нахождения оптимального решения, превосходит предельно допустимое для исполнения управляющего вмешательства. Сама возможность найти оптимальное решение становится не нужной. Систему, для нахождения оптимального воздействия на которую достаточно информационного ресурса (модель адекватна), но не достаточно времени, назовем \note{большой}, в противном случае --- \note{малой}.
& Самый эффективный способ управления большой системой --- превратить ее в малую, ускорив процесс моделирования. Например: замени моделирующий компьютер более быстродействующим или распараллеливая алгоритм. Такой способ может натолкнуться на непреодолимые трудности. Менее эффективный по качеству управления, но дающий своевременный результат способ  --- принимать первый получившийся удовлетворительный вариант. Для получения слабого, но быстрого решения идут на различные упрощения модели - линеаризация, аппроксимация, округление, etc.
& \define{Седьмой тип управления}{\note{управление при отсутствии информации о конечной цели}}
& Первый способ состоит в том, чтобы дать субъективное, априорное определение конечной цели, а дальше действовать по предыдущим схемам. Наглядный пример: управление крупными социальными системами.
& Другой подход предполагает, что конечная цель все-таки существует и должна существовать траектория продвижения к ней. Она так же неизвестна, но можно попытаться исследовать ближайшую окрестность вокруг текущего состояния и определить наиболее предпочтительное направление следующего шага в пределах этой окрестности. Затем сделать шаг и повторить. Примеры: эволюция и естественный отбор.
\end{easylist}

\begin{easylist}
& Придумайте примеры систем, которые были бы одновременно: малой и простой, малой и сложной, большой и простой, большой и сложной.
&& Малая-простая: практически любой бытовой прибор, например, электрочайник.
&& Малая-сложная:
&& Большая-простая:
&& Большая-сложная:
\end{easylist}

\newpage
\section{Приложение}{Вопросы для Anki}
\subsection{Введение}
\begin{easylist}
& Почему накопление и обобщение опыта решения проблем началось (и продолжается) в рамках каждой отдельной профессии?
&& Поскольку для решения проблемы зачастую необходимы глубоко профессиональные знания, создалось впечатление, что проблемы в каждой области знания уникальны. Поэтому накопление и обощение опыта решения проблем началось в рамках каждой профессии отдельно.
& Почему, несмотря на громадное разнообразие проблем, технология (совокупность приемов) их решения практически одинакова в случае успеха и различается в случае неудач?
&& Идея универсального алгоритма действий по решению проблем опирается на понятие всеобщей системности. Ошибиться же можно множеством различных путей.
& Сформулируйте основные отличия прикладного системного анализа от традиционных наук.
&& Прикладной системный анализ нацелен на решение конкретной проблемы; для решения проблемы могут понадобиться знания любой из традиционных наук; системный анализ выполняется участниками проблемной ситуации.
& Почему прикладной системный анализ можно назвать над-дисциплинарной и меж-дисциплинарной область деятельности как в теоретической, так и в практической его сфере?
&& Необходимость использования знаний из различных сфер деятельности; привлечением к системному анализу соответствующих специалистов, задействованных в проблемной ситуации.
\end{easylist}
\subsection{Проблема и способы ее решения}
\begin{easylist}
& Поясните различия между понятиями \enquote{проблемная ситуация} и \enquote{проблема}.
&& \enquote{Проблемная ситуация} выделяет объективный компонент, реальную ситуацию; \enquote{проблема} --- субъективное отношение к реальной ситуации субъекта.
& Что значит \enquote{решить проблему}?
&& Решение проблемы объединяет в себе какие-либо действия, призванные уменьшить или совсем снять недовольство субъекта.
& Какие три способа воздействия на субъект без изменения реальности могут (при определенных условиях) привести к решению его проблемы? Каковы эти условия?
&& Сообщение субъекту дополнительной информации позитивного характера о ситуации. Может происходить в виде обучения, быть неправдивой или отфильтрованной полуправдой.
&& Изменение восприятия данной реальности субъектом, например, с помощью алкоголя или наркотков.
&& Прекращение взаимодействия субъекта с ситуацией. Субъект может быть повышен, направлен в отпуск, переведен в другой отдел, уволен.
& Каково основное отличие субъекта от объекта?
&& Субъект существует в реальной физической среде и наделен способностью оценивать свои взаимодействия со средой: что-то ему может нравиться, а может и не нравиться.
& Как определить смысл оценки, выраженной неким субъектом?
&& Так как оценки не бывают объективны, для понимания смысла необходимо уточнить критерии оценки.
& Почему при вмешательстве в реальность с целью решения проблемы приходится опираться на какую-то идеологию?
&& Правильным считается поведение, максимально согласующееся с принятой субъектом идеологией. Именно идеология и определяет, что плохо, а что хорошо.
& Воспроизведите классификацию идеологий на три типа. Каково основное отличие между ними?
&& Основное отличие между идеологиями --- определение того, какое отношение к другим субъектам считать правильным.
&&& \enquote{Принцип приоритета меньшинства}: осуществить вмешательство, угодное клиенту, интересы других участников не принимаются во внимание. Примеры: диктатура, иерархическая организация, эгоизм.
&&& \enquote{Принцип приоритета группы}: среди участников ситуации есть важные субъекты, помимо клиента. Вмешательство проводится с учётом интересов \enquote{наших}. Примеры: расизм, национализм, фашизм, коммунизм.
&&& \enquote{Принцип приоритета каждого}: все субъекты равноценны и равноправны, правильным признается только улучшающее вмешательство.
& Назовите четыре типа улучшающих вмешательств.
&& \important{ABSOLUTION} --- невмешательство. Пример: действия сапера при встрече с незнакомым взрывным устройством.
&& \important{RESOLUTION} --- частичное вмешательство. Пример: распределение по жребию или очереди.
&& \important{SOLUTION} --- оптимальное решение.
&& \important{DISSOLUTION} --- вмешательство с изменением условий, заканчивающееся полным исчезновением проблемы.
& Оптимальность обеспечивается при совокупном соблюдении двух требований. Каковы эти требования?
&& Оптимальный значит \enquote{наилучший в данных условиях}. Первым требованием является определение, критерия сравнения варианта. Вторым требованием является определение ограничений.
& Каков важный результат прикладного системного анализа конкретной проблемы, кроме решения самой проблемы?
&& Важный результат прикладного системного анализа - обучение системному анализу в процессе.
& Дайте определение следующим понятиям:
&& \define{Проблемная ситуация}{некоторое реальное стечение обстоятельств, положение вещей, которым кто-то недоволен, неудовлетворён и хотел бы изменить.}
&& \define{Оценка}{мнение, суждение субъекта о чем-либо.}
&& \define{Проблема}{субъективное отрицательное отношение субъекта к реальности.}
&& \define{Решение проблемы}{какие-либо действия, призванные уменьшить или совсем снять недовольство субъекта.}
&& \define{Вмешательство}{изменение проблемной ситуации, снижающее недовольство клиента.}
&& \define{Улучшающее вмешательство}{изменение проблемной ситуации, которое положительно оценивается хотя бы одним из ее участников и неотрицательно --- всеми остальными.}
&& \define{Прикладной системный анализ}{теория и практика проектирования и реализации улучшающих вмешательств. Методика решения проблем реальной жизни без создания новых проблем.}
&& \define{Оптимальность}{сочетание наилучших параметров по заданным критериям в рамках существующих ограничений.}
&& \define{\enquote{Твёрдая проблема}}{хорошо структурированная, допускающая построение количественных математических моделей. Например, многие инженерные и научные проблемы.}
&& \define{\enquote{Мягкая проблема}}{неструктурированные, описанные на естественном языке ситуации, не относящиеся к точным наукам.}
\end{easylist}
\subsection{Понятие системы}
\begin{easylist}
& Что такое статические свойства систем? Перечислите четыре статических свойства.
&& \note{Статическими свойствами} называются особенности конкретного состояния системы.
&&& \define{Целостность}{существование системы как чего-то единого, целого, обособленного, отличающегося от всего остального; факт внешней различимости в среде.}
&&& \define{Открытость}{связность системы и окружающей среды, обмен между ними любыми видами ресурсов.}
&&& \define{Внутренняя неоднородность}{различимость частей системы.}
&&& \define{Структурированность}{Наделенность любой системы определенной структурой.}
& Как из открытости систем вытекает факт всеобщей взаимосвязанности в природе?
&& Следствием открытости систем является очевидность \note{всеобщей взаимосвязи и взаимозависимости в природе}. Между любыми двумя системами обязательно существует, и ее можно отыскать, длинная или короткая цепочка систем, связывающая их: выход каждой системы является входом другой.
& Что называется моделью черного ящика? Назовите четыре рода ошибок, которые можно совершить при построении модели черного ящика.
&& Перечень входов и выходов системы называют \note{моделью черного ящика}. В этой модели отсутствует информация о внутренних особенностях системы.
&&& \note{Ошибка первого рода} происходит, когда субъект расценивает связь как существенную и принимает решение о включении ее в модель, хотя на самом деле она несущественна и могла бы быть неучитываемой.
&&& \note{Ошибка второго рода} происходит, когда субъект принимает решение, что данная связь несущественна и не должна быть включена в модель.
&&& \note{Ошибкой третьего рода} принято считать последствия незнания о существовании какой-либо связи. Если связь существенна, испытываемые трудности будут соответствовать ошибке второго рода. Ошибку третьего рода труднее исправить: необходимо добывать новые знания.
&&& \note{Ошибка четвертого рода} может возникнуть при неверном отнесении известной и признанной существенной связи к числу входов или выходов. 
& Что называется моделью состава системы? Каковы три трудности ее построения?
&& \define{Модель состава системы}{иерархический список частей системы.}
&& Трудности построения модели состава можно представить тремя положениями:
&&& Целое можно делить на части по-разному, в зависимости от того, что требуется для достижения цели. При этом, можно \note{различать} нужные для цели части, но не следует \note{разделять их}.
&&& Количество частей в модели состава зависит от того, на каком уровне остановить дробление системы. Части на конечных ветвях получающегося иерархического дерева называют \note{элементами}. Прекращение декомпозиции производится на разных уровнях, в зависимости от обстоятельств и того, что считать \note{элементарным}.
&&& Любая система является частью какой-либо большей системы. Мета-систему тоже можно делить на подсистемы по-разному. Таким образом, \note{внешняя граница системы имеет относительный, условный характер}.
& При каких предположения можно говорить о наличии частей у системы?
&& ?
& Как определяется граница системы?
&& \note{Внешняя граница системы имеет относительный, условный характер}. Определение границ системы производится с учетом целей субъекта.
& Что называется моделью структуры системы? В чем трудности ее построения?
&& Перечень \note{существенных связей между элементами системы} называется \note{моделью структуры системы}.
&& Модель структуры определяется после выбора модели состава и зависит от нее. При этом, модель структуры вариабельна даже при зафиксированном составе из-за возможности по-разному определить существенность связей.
&& Каждый элемент системы представляет собой <<черный ящик>>, соответственно могут привнести ошибки, полученные при определении своих входов и выходов.
& Что такое динамические свойства систем? Перечислите их (все четыре).
&& \define{Динамические свойства}{особенности изменений со временем внутри системы и вне ее.}
&&& Функциональность
&&& Стимулируемость
&&& Изменчивость системы со временем
&&& Существование в изменяющейся среде
& Поясните различие между ростом и развитием системы.
&& Рост подразумевает наращивание состава системы, увеличение в размерах и численности. Развитие --- существенно изменение свойств системы в позитивном направлении. Рост системы может быть ограничен недостатком ресурсов, тогда как развитие является результатом обучения и требует внутреннего желания
& Что мы называем синтетическими свойствами систем? Перечислите четыре таких свойства.
&& Синтетическими называют обобщающие, собирательные, интегральные свойства, делающие упор на взаимодействия системы со средой, на целостность.
&&& \define{Эмерджентность}{объединение частей в систему, порождающее качественно новые свойства, не сводящиеся к свойствам частей, не выводящиеся из свойств частей, присущие только самой системе и существующие только пока система составляет одно целое}
&&& \important{Неразделимость на части}
&&& \define{Ингерентность}{согласованность, приспособленность к окружающей среде в рамках некой функции}
&&& \define{Целесообразность}{подчиненность всего поставленной цели}
& Какое из статических свойств системы обеспечивает существование эмерджентных свойства системы?
&& ?
& Что называется субъективной целью системы?
&& Субъективной целью называют цель, представляемую конструктором системы, ограниченная только желаниями и фантазией субъекта.
& Почему не любая субъективная цель достижима?
&& Достижение субъективной цели может быть невозможно в том случае, если данная цель находится за пределами законов природы, т.е. не принадлежит к объективным целям.
& Дайте определение следующим понятиям:
&& \define{Целостность системы}{существование системы как чего-то единого, целого, обособленного, отличающегося от всего остального; факт внешней различимости в среде.}
&& \define{Открытость системы}{связность системы и окружающей среды, обмен между ними любыми видами ресурсов.}
&& \define{Черный ящик}{совокупность входов и выходов системы}
&& \define{Ошибка первого рода}{оценка связи как существенной и включение ее в модель, хотя на самом деле она несущественна и могла бы быть неучитываемой.}
&& \define{Ошибка второго рода}{оценка связь как несущественной и не включение ее в модель.}
&& \define{Ошибкой третьего рода}{незнание о существовании какой-либо связи. Если связь существенна, испытываемые трудности будут соответствовать ошибке второго рода. Ошибку третьего рода труднее исправить: необходимо добывать новые знания.}
&& \define{Ошибка четвертого рода}{неверное отнесение известной и признанной существенной связи к числу входов или выходов.}
&& \define{Модель состава системы}{иерархический список частей системы.}
&& \define{Подсистема}{система, являющаяся частью более общей системы.}
&& \define{Элемент системы}{часть системы, считаемая простейшей и неделимой, в рамках заданной декомпозиции.}
&& \define{Модель структуры системы}{перечень существенных связей между элементами системы.}
&& \define{Функция системы}{процесс на выходе системы.}
&& \define{Стимулируемость}{подверженность системы воздействиям извне и изменение ее поведения под этими воздействиями.}
&& \define{Функционирование}{тип динамики системы, при котором изменения не затрагивают структуру системы: одни элементы заменяются другими, эквивалентными; параметры могут меняться без изменения структуры.}
&& \define{Рост}{наращивание состава системы}
&& \define{Спад}{уменьшение состава системы}
&& \define{Развитие}{качественное улучшение существенных свойств}
&& \define{Деградация}{качественное ухудшение существенных свойств}	
&& \define{Жизненный цикл}{последовательность периодов спадов и подъемов в истории системы}
&& \define{Эмерджентность}{объединение частей в систему, порождающее качественно новые свойства, не сводящиеся к свойствам частей, не выводящиеся из свойств частей, присущие только самой системе и существующие только пока система составляет одно целое.}
&& \define{Ингерентность}{согласованность, приспособленность к окружающей среде в рамках некой функции}
&& \define{Цель субъективная}{цель, представляемая конструктором системы, ограниченная только желаниями и фантазией субъекта.}
&& \define{Цель объективная}{цель, представленная как реальное будущее состояние, существующая в рамках законов природы.}
\end{easylist}
\subsection{Модели и моделирование}
\begin{easylist}
& Покажите, что познавательная и преобразовательня деятельности субъекта невозможны без моделирования
&& Познавательная деятельность требует наличие желаемого результата, т.е. модели того, чего нет, и описания алгоритма планируемых для достижения действий.\\Познавательная деятельность представляет из себя составление модели окружающего мира.
& Опишите алгоритм анализа и перечислите, какие модели он порождает
&& Процедура анализа состоит в последовательном выполнении следующих операций:
&&& Сложно расчленить на более мелкие, предположительно более простые, части.
&&& Дать объяснение полученным фрагментам.
&&& Объединить объяснение частей в объяснение целого.
&& Если какая-то часть системы остается непонятной, операция декомпозиции повторяется.\\Продуктами анализа являются модель состава и структуры системы, модель черного ящика для каждого элемента системы.
& Опишите алгоритм синтеза и укажите, какие модели он порождает. Какая из них непосредственно описывает исследуемый объект (явление)?
&& Процедура синтеза состоит в последовательном выполнении следующих операций:
&&& Выделение метасистемы.
&&& Рассмотрение состава и структуры метасистемы.
&&& Объяснение роли искомой системы в метасистеме через ее связи с другими подсистемами.
&& Продуктами синтеза являются модели состава и структуры метасистемы, модель черного ящика системы.
& Что такое модель?
&& Модель есть средство осуществления любой деятельности субъекта.
&& Модель есть форма существования знаний.
&& Модель есть системное отображение оригинала.
& Что такое <<абстрактная модель>>? Кроме языковых, какие еще примеры абстрактных моделей вы можете привести?
&& Абстрактная модель --- модель, представляющая собой инструмент мышления; её компонентами являются понятия, а не физические элементы. В качестве абстрактных моделей могут выступать схемы, изображения, диаграммы, макеты, алгоритмы, математические описания, etc.
& Чем вызвано многообразие языков?
&& Любая групповая деятельность требует большей точности и определенности терминов, по сравнению с разговорными языками, что приводит к появлению языков профессиональных, снимающих неопределенность за счет дополнительной информации. Поскольку, большая часть различных групповых активностей между собой не пересекается, нет ничего удивительного в том, что разные группы требуют разной дополнительной информации от языка, что и способствует существованию различных профессиональных языков, актуальных только для небольшой, конкретной сферы деятельности.
& Какова простейшая абстрактная модель разнообразия окружающей нас реальности?
&& Классификация представляет собой простейшую абстрактную модель разнообразия окружающей реальности, разделяя все объекты на классы по каким-либо признакам.
& Чем отличается искусственная и естественная классификации?
&& Искусственная классификация подразумевает разделение исходя из поставленной цели --- на столько классов и с такими границами, как это диктуется целью. Естественная классификация подразумевает деление на основе существующих природных группировок/кластеров.
& Что называется <<реальной моделью>>? Приведите три типа реальных моделей (классификацию по происхождения подобия модели оригиналу).
&& Реальной моделью может выступать какой-либо материальный предмет, используемый в качестве модели. Классификация по происхождению подобия между оригиналом и моделью, приводит к трем типам реальных моделей:
&&& Модели \note{прямого подобия} созданы с помощью либо непосредственного взаимодействия (следы, печать), либо цепочки таких взаимодействий (фотография, макет здания).
&&& Модели \note{косвенного подобия} или аналогии. Похожесть, аналогичность двух явлений объясняется совпадением закономерностей, которым они подчиняются. Моделями косвенного подобия являются: аналоговые ЭВМ, исторические параллели, подопытные животные в медицине.\\Следует осторожно пользоваться аналогиями, поскольку, кроме совпадающих закономерностей, у разных явлений есть и несовпадающие.
&&& Модели \note{условного подобия}. Соответствие такой модели и оригинала устанавливается в результате соглашения между ее пользователями и носит условный характер. Они успешно работают до тех пор, пока известны и соблюдаются договоры об их значении. Примеры: деньги --- модели стоимости, буквы --- модели звуков.
& Чем отличается использование познавательных и прагматических моделей?
&& Познавательные модели нацелены на получение информации о материальном мире, потому подвержены изменениям вследствие получения новых знаний и терпимы к отличающимся и противоречивым мнениям. Для познавательных моделей, понятия истинности и адекватности выступают синонимами.\\Прагматические модели обслуживают процессы преобразования реальности в соответствии с целями субъекта, а потому имеют директивный, единственно верный характер. В качестве примера можно рассмотреть религию или мораль. Кроме того, адекватная прагматическая модель не обязательно должно быть истинной --- ложь так же может выступать инструментом достижения цели.
& Почему в любой модели есть, кроме истинного, и (обязательно и неизбежно) неистинное содержание?
&& Истинное содержание является чем-то общим и для модели, и для оригинала, благодаря чему модель может служить заменителем оригинала. В то же время, т.к. модель так же является системой, она обязательно будет обладать каким-либо собственными свойствами --- неистинным содержанием, которое не будет иметь отношения к оригиналу.
& Какое качество модели называется адекватностью?
&& Адекватностью называют степень успешности модели в достижении цели субъекта. Для познавательных моделей, адекватность и истинность являются синонимами. Для прагматических моделей, ложные модели могут быть адекватными.
& Что является окружающей средой для модели?
&& Окружающей средой для искомой модели выступает культура (мир моделей) пользователя.
& Дайте определения следующих терминов:
&& \define{Анализ}{разложение системы на составные, более простые, части с целью объяснения целого на основе объяснения частей.}
&& \define{Синтез}{рассмотрение состава и структуры метасистемы, с целью объяснения роли системы на основе ее связей с подсистемами метасистемы.}
&& \define{Модель абстрактная}{модель, представляющая собой инструмент мышления; её компонентами являются понятия, а не физические элементы.}
&& \define{Модель языковая}{слово, представляющее собой какое-либо понятие.}
&& \define{Модель реальная}{материальный предмет, используемый в качестве модели.}
&& \define{Классификация искусственная}{разделение на классы на основе поставленной цели. Количество и границы классов задаются целью.}
&& \define{Классификация естественная}{разделение на классы на основе существующих природных группировок.}
&& \define{Модели познавательные}{модели, обслуживающие процессы получения о внешнем мире.}
&& \define{Модели прагматические}{модели, обслуживающие процессы преобразования реальности в соответствии с целями субъекта.}
&& \define{Адекватность модели}{степень успешности модели в достижении цели.}
&& \define{Культура (субъекта, организации, нации --- любой социальной системы)}{мир моделей, сформировавшийся в процессе жизнедеятельности системы.}
\end{easylist}
\subsection{Управление}
\begin{easylist}
& Какие пять составляющих обеспечивают выполнение процесса управления?
&& Для выполнения процесса управления необходимы: объект управления; цель управления; управляющее воздействие; модель системы и блок управления.
& При каких условиях поиск управляющего воздействия на самой системе является неразумным, неприемлемым?
&& Поиск управляющего воздействия на самой системе сводиться к перебору возможных управляющих воздействий. В случае, если множество управляющих воздействий будет достаточно велико или каждое неверное решение будет причинять какие-либо потери, работа напрямую с системой становится нецелесообразной.
& Что называется простой системой? В чем причина простоты?
&& В случае, если поведение самой системы и ее модели для одного и того же управляющего воздействия совпадают, такую систему можно назвать простой. Простота системы есть следствие адекватности модели. Так же можно сказать, что простота системы характеризует полноту знаний о системе.
& Какую систему называют сложной? Какова причина сложности?
&& Сложной называется та система, у которой поведение самой системы и ее модели для одного и того же управляющего воздействия отличаются. Сложность системы есть следствие неадекватности модели. Так же можно сказать, что сложность системы характеризует недостаток знаний о системе.
& Опишите алгоритм метода проб и ошибок. Какими особенностями он обладает?
&& Метод проб и ошибок предназначен для получения дополнительной информации о системе. Представляет из себя рекурсивный алгоритм, на каждой итерации которого: на основе модели выбирается оптимальное управляющее воздействие; происходит проба полученного управляющего воздействия на системе; на основе полученных данных совершенствуется модель.
& Чем отличается метод проб и ошибок от <<метода тыка>>?
&& Метод проб и ошибок предполагает поиск необходимого управляющего воздействия на основе модели системы, тогда как <<метод тыка>> предполагает поиск воздействия на самой системе.
& Перечислите, какие функции выполняет регулятор.
&& Регулятор выполеняет следующие функции:
&&& Держит в памяти опорную траекторию, ведущую к цели.
&&& Следит за реальной траекторией движения системы.
&&& Ищет различие между опорной и реальной траекториями.
&&& Вычисляет на основе модели корректирующее, дополнительное управляющее воздействие.
&&& Выполняет полученное управляющее воздействие на системе, возвращая ее на опорную траекторию.
& В чем состоит управление по структуре?
&& Управление по структуре предполагает изменение структуры существующей системы, с использованием как частей исходной системы, так и привлечением новых частей из вне. Управление по структуре целесообразно в тех случаях, когда отклонение системы от целевой траектории настолько велико, что не позволяет вернуть систему назад изменением параметров.
& В чем состоит управление по целям? При каких условиях применим этот тип управления?
&& Управление по целям предполагает смену цели в сторону понижения предъявляемых требований и переориентирования на достижимые сроки; что позволяет снизить требования к ресурсам, необходимым для достижения цели. Данный тип управления целесообразен в тех случаях, когда другие типы управления не позволяют достигнуть цели при существующих ресурсах.
& Что такое <<большая система>>? Каковы варианты управления ею?
&& Большой системой является та система, для воздействия на которую достаточно информационного ресурса, но недостаточно времени. Управлять данной системой можно либо путем превращения ее в малую системе, за счет ускорения процесса моделирования; либо за счет отказа от оптимального решения в пользу первого удовлитворительного варианта; либо упрощением модели, с целью получения более слабого варианта, требующего меньше времени для вычисления.
\end{easylist}
\end{document}
